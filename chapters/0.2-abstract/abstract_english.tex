\chapter*{Abstract}
\chaptermark{Abstract}
\addcontentsline{toc}{chapter}{Abstract}

%Document analysis is an active field of research which can attain a complete understanding of the semantics of a given document. One example of the document understanding process is enabling a computer to understand a comic strip story. 
% In this study we propose a knowledge-driven system that can interact with bottom-up and top-down information to progressively understand the content of a document.
% We model the comic book’s and the image processing domains knowledge for information consistency analysis. In addition, different image processing methods are improved or developed to extract panels, balloons, tails, texts, comic characters and their semantic relations in an unsupervised way.

Born in the 19th century, comics is a visual medium used to express ideas via images, often combined with text or visual information.
It is considered as a sequential art, spread worldwide initially using newspapers, books and magazines.
Nowadays, the development of the new technologies and the World Wide Web is giving birth to a new form of paperless comics that takes advantage of the virtual word freedom.
However, traditional comics still represent an important cultural heritage in many countries.
They have not yet received the same level of attention as music, cinema or literature about their adaptation to the digital format.
Using information technologies with classic comics would facilitate the exploration of digital libraries, faster theirs translations, allow augmented reading, speech playback for the visually impaired etc.

Heritage museums such as the CIBDI (French acronym for International City of Comic books and Images), the Kyoto International Manga Museum and the digitalcomicmuseum.com have already digitized several thousands of comic albums that some are now in the public domain.
Despite the expending market place of digital comics, few researches have been carried out to take advantage of the added value provided by these new media.
Document analysis is the corresponding field of research which is relatively application dependent.
The design process of comics is so typical that their automated analysis may be seen as a niche research field within document analysis, at the intersection of complex background, semi-structured and mixed content documents.

Being at the intersection of several fields combine their difficulties.
In this thesis, we review, highlight and illustrated the challenges in order to give to the reader a good overview about the last research progress in this field and the current issues.
We propose three different approaches for comic book image analysis relying on previous work and novelties.
The first approach is called ``sequential'' because the image content is described in an intuitive way, from simple to complex elements using previously extracted elements to guide further processing.
Simple elements such as panel text and balloon are extracted first, followed by the balloon tail and then the comic character position in the panel by from the direction pointed by the tail.
The second method addresses independent information extraction to recover the main drawback of the first approach: error propagation.
This second method is called ``independent'' because it is composed by several specific extractors for each elements of the image content.
Those extractors can be used in parallel, without needing previous extraction.
Extra processing such as balloon type classification and text recognition are also covered.
The third approach introduce a knowledge-driven system that combines low and high level processing to build a scalable system of comics image understanding.
We built an expert system composed by an inference engine and two models, one for comics domain and an other one for image processing related domain, stored in an ontology.
This expert system combines the benefits of the two first approaches and enables high level semantic description such as the reading order of panels and text, the relations between the speech balloons and their speakers and the comic character identification.

Apart from that, in this thesis we have provided the first public comics image dataset and ground truth to the community along with an overall experimental comparison of all the proposed methods and some of the state-of-the-art methods.
%Furthermore, some dataset models have also been proposed.


% At first glance, the structure of a comic book image may appear easy to extract.
% In practice, the configuration of the page, the size and the shape of the panels
% can be different from one page to the next.
% Even if some conventions were established historically, drawers have a certain liberty.
% With a complex and relatively free layout, the application
% of methods used for other media (e.g. newspapers, magazines)
% is inefficient on comics images.

% Comics being unstructured graphical documents, combine the difficulties of both domains, making the task of content extraction especially challenging.
% On one hand, they differ from classical documents in
% that they comprise complex backgrounds of a graphical nature. On the other hand, they belong to the class of non-structured documents meaning there is no regular structure present for the prediction of content locations and no layout method applicable.

% Image processing has been widely studied during the last decades on grey-scale image and more recently on colour images. We are now able to provide a low level description for simple object recognition purpose. 
% In the case of complex object composed of many regions as comics characters (where each region has its own shape, colour, texture), it is necessary to consider the location of each single region as well. A priori model of character is very tricky to specify because of posture and overlapping variabilities. 
% This thesis will first focus on panel (a single drawing in a comic strip) description by segmentation (colour, texture...) and then relative region location (graph, descriptor...). The second step is to propose an indexation process (statistical analysis, graph comparison...) with the objective of redundant structure retrieval among the frames of the albums. These particular structures may be assimilated as recurrent object and therefore character or special scenery. The user will interact throughout the process to provide relevant feedback to revise the description model in order to reach a  ``high level'' description.
% The last step is passing from “high level” (e.g. it is a character) to semantic representation (e.g. it is Tintin/Asterix). This representation will enrich a system of knowledge representation developed as part of another thesis.






\clearpage\thispagestyle{empty}