\chapter*{R{\'e}sum{\'e}}
\chaptermark{R{\'e}sum{\'e}}
\addcontentsline{toc}{chapter}{R{\'e}sum{\'e}}

Née au 19ème siècle, les bandes dessinées sont utilisées pour l'expression d'idées au travers de séquences d'images, souvent en combinaison avec du texte et des graphiques.
La bande dessinée est considérée comme le neuvième art, l'art séquentiel, diffusé grâce aux progrès de l'imprimerie puis de l'Internet à travers le monde dans les journaux, les livres et les magazines.
De nos jours, le développement grandissant des nouvelles technologies et du World Wide Web (la toile Internet) donne naissance à de nouvelles formes d'expressions s'acquittant du support papier pour profiter de toute la liberté du monde virtuel.
Cependant, la bande dessinée traditionnelle continue a perdurer et représente un patrimoine culturel important dans de nombreux pays.
À la différence de la musique, du cinéma ou encore de la littérature classique, elle n'a pas encore trouvée sont homologue dans l'univers du numérique.
L'utilisation des technologies de l'information et de la télécommunication pourrait faciliter l'exploration de bibliothèques en ligne, accélérer leur traduction et exportation, permettre de faire de la lecture augmentée (enrichissement du contenu lors de la lecture, à la demande et personnalisé) ou encore permettre l'écoute du texte et des bruitages pour les mal-voyants ou les apprenants.

Les organismes de préservation du patrimoine culturel comme le CIBDI à Angoulême (Centre International de la Bande Dessinée et de l'Image), le musée international du manga à Kyoto (Kyoto International Manga Museum) ou encore le site digitalcomicmuseum.com aux États-unis ont déjà numérisé des centaines d'albums dont certain sont du domaine public.
Malgré la part de marché grandissante de la bande dessinée numérique dans les pays développés, peu de recherches ont été menées à ce jour pour valoriser ces contenus au travers des nouvelles technologies.
L'analyse de document est une thématique de recherche qui traite ce genre de problème.
Une de ces particularités est sa dépendance au type de document qui requiert souvent des traitements spécifiques.
Le processus de création d'une bande dessinée est propre à cet art qui peut être considéré comme une niche du domaine de l'analyse de document.
En réalité, cette niche est à l'intersection de plusieurs problématiques de recherche qui compte les documents constitués d'un fond complexe, semi-structurés et avec un contenu varié.

L'intersection entre plusieurs thématiques de recherche combine leurs difficultés.
Dans ce manuscrit de thèse, nous détaillons et illustrons les différents défis scientifiques liés à ces travaux de recherche de manière à donner au lecteur tous les éléments concernant les dernières avancées scientifiques en la matière ainsi que les verrous scientifiques actuels. 
Nous proposons trois approches pour l'analyse d'image de bandes dessinées composé de différents traitements dont certains améliorent des travaux antérieurs et d'autres étant de nouvelles pistes d'exploration.
La première approche est dite ``séquentielle'' car le contenu de l'image est décrit progressivement et de manière intuitive.
% depuis des éléments simple à complexe, en utilisant les éléments simple pour guider la recherche d'élément plus complexe.
Dans cette approche, l'extraction des éléments se succède, en commençant par les plus simples tels que les cases, le texte et les bulles qui servent ensuite à guider l'extraction d'éléments complexes tels que la queue des bulles et les personnages au sein des cases en fonction de la direction pointée par les queues.
La seconde méthode propose des extractions indépendantes les unes des autres de manière à éviter la propagation d'erreur entre les traitements.
Dans cette approche, les différents extracteurs peuvent être utilisés en parallèle puisque qu'ils n'ont pas d'inter-dépendance.
D'autres éléments tel que la classification du type de bulle et la reconnaissance de texte y sont associés.
La troisième approche introduit un système fondé sur une base de connaissance \emph{à priori} du contenu des images de bandes dessinées qui permet d'interagir entre des traitements de bas et haut niveaux pour construire une description sémantique de l'image.
Nous proposons un système expert composé d'un système d'inférence et de deux modèles sous forme d'ontologies, un pour modéliser le domaine de la bande dessinée, et l'autre pour modéliser les traitements d'images associés.
Ce système dirigé par les modèles, combine les avantages des deux approches précédentes et permet une description sémantique de haut niveau pouvant inclure des informations telles que l'ordre de lecture des cases, du texte et des bulles, des relations entre les bulles  et leurs locuteurs ainsi que la distinction entre les personnages.

Dans cette thèse, nous introduisons également la première base d'images de bandes dessinées ainsi que la vérité terrain associée comportant des informations bibliographiques, spatiales et sémantiques.
Cette base d'images annotées a été mise à disposition de la communauté scientifique.
Des expérimentations basées sur les méthodes proposées et une comparaison avec des approches de la littérature sont également détaillées dans ce manuscrit.



\clearpage\thispagestyle{empty}