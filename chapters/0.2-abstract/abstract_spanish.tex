\chapter*{Resumen}
\chaptermark{Resumen}
\addcontentsline{toc}{chapter}{Resumen}

Nacido en el siglo 19, los historietas se utilizan para la expresión de ideas a través de secuencias de imágenes, a menudo en combinación con el texto y los gráficos.
El cómic esta considerado como un noveno arte, arte secuencial, salida con los avances en la impresión y la Internet en todo el mundo en periódicos, libros y revistas.
Hoy en día, el creciente desarrollo de las nuevas tecnologías y la World Wide Web (el lienzo Internet) da lugar a nuevas formas de expresión que lleva el papel a disfrutar de la libertad del mundo virtual.
Sin embargo, el cómic tradicional persiste y es un patrimonio cultural importante en muchos países.
A diferencia de la música, el cine o la literatura clásica, que aún no ha encontrado son homólogos en el mundo digital.
El uso de tecnologías de la información y de las telecomunicaciones podría facilitar la exploración de bibliotecas en línea, la traducción y acelerar su permiso de exportación a la mayor lectura (enriquecimiento de los contenidos durante la reproducción, a la carta y personalizado ) o permitir la escucha de texto y efectos de sonido para los estudiantes con discapacidad visual o allumnos.

Agencias de la preservación del patrimonio cultural como CIBDI en Angouleme (Centro Internacional del Cómic y de imagen), el Museo Internacional de Manga en Kioto (Kyoto International Manga Museum) o el sitio digitalcomicmuseum.com de los Estados Unidos han digitalizado cientos de álbumes, algunos son públicos.
Pese a la creciente cuota de mercado de los cómics digitales en los países desarrollados, poca investigación se ha llevado a cabo hasta la fecha para desarrollar estos contenidos a través de las nuevas tecnologías.
El análisis de documentos es un tema de investigación que se ocupa de este problema. Una de estas características es la dependencia del tipo de documento que a menudo requiere un tratamiento específico.
El proceso de creación de un cómic es exclusivo de este arte que puede ser considerado como un nicho en el campo de análisis de documentos.
En realidad, este nicho está en la intersección de varios documentos de investigación que cuenta consiste en un fondo complejo, contenido semi-estructurada y variada.

La intersección de varias investigaciones combina sus dificultades.
En esta tesis de doctorado, se describen e ilustran los diversos retos científicos de esta investigación con el fin de dar al lector toda la evidencia acerca de los últimos avances científicos en el campo, así como las barreras científicas actuales.
Proponemos tres enfoques de análisis de imagen cómica compuesta por diferentes tratamientos que mejora algunos trabajos previos y otros que son nuevas vías de exploración.
El primer enfoque se denomina ``secuencial'' porque los contenidos de la imagen se describe gradualmente y de manera intuitiva.
Simples artículos como cajas y texto y las burbujas se extraen primero y luego siguen la cola de las burbujas y los personajes de los cuadros de acuerdo a la dirección apuntada por las colas.
El segundo método ofrece extracciones independientes unos de otros a fin de evitar la propagación del error entre aplicaciones, que es la principal desventaja del primer método.
En este enfoque, los diversos extractores se pueden utilizar en paralelo, ya que no tienen la interdependencia.
Otros elementos como la clasificación del tipo de burbuja y el reconocimiento de texto están asociados.
El tercer enfoque introduce un sistema basado en un conocimiento a priori del contenido de las imágenes de dibujos animados que interactúa entre los tratamientos bajos y altos niveles para construir una descripción semántica de la imagen.
Proponemos un sistema experto consiste en un sistema de inferencia y dos modelos de la forma de ontologías, un modelo para el campo de los cómics y el otro para modelar el procesamiento de imágenes asociado.
Este sistema experto combina las ventajas de ambos enfoques anteriores y proporciona un alto nivel de descripción semántica puede incluir información como el orden de lectura de los cuadros, el texto y las burbujas, burbujas relaciones entre habladas y sus altavoces y el distinción entre los caracteres.

Además, se describen los primeros cómics públicas basadas en imágenes y la realidad sobre el terreno que incluye que se han propuesto a la literatura científica, la información espacial y semántica.
Un experimento de todos los métodos propuestos y una comparación de los enfoques de la literatura también se detallan en este manuscrito.

% White page
\clearpage\thispagestyle{empty}