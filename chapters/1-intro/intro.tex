\chapter{Introduction}% maximum 20 pages
\chaptermark{Introduction}
\label{chap:intro}
\graphicspath{{./chapters/1-introduction/figs/}}
% Overview of pattern recognition
\section{The evolution of the ``bandes dessinée''}
Setting/positioning the scenario, place comics in the addressed context, origin of comics/BD until new usages in 2014 created by new technologies.
\cite[p.~215]{McCloud94}

\begin{itemize}
	\item The origin of BD, comics, Manga, other

	\item see Clement's intro
	\item collectif PANIC book
	\item Évolution de la bande dessinée en France depuis quinze ans \url{http://ejournals.library.ualberta.ca/index.php/af/article/download/21313/16112}
	\item \url{http://fr.wikipedia.org/wiki/Bande_dessinee_en_ligne}

	\item \url{http://en.wikipedia.org/wiki/Comics}
	\item European Comic Art \url{http://journals.berghahnbooks.com/eca/}
	\item Webcomics \url{http://www.phdcomics.com} and \url{http://xkcd.com}
	\item Le manga: Une synthèse de référence qui éclaire en image (Eyrolles, 2013) \url{http://books.google.fr/books?id=3MqgAgAAQBAJ&printsec=frontcover#v=onepage&q&f=false}
	\item \url{http://en.wikipedia.org/wiki/Manga}
	\item Etat présent BANDE DESSINEE STUDIES LAURENCE GROVE UNIVERSITY OF GLASGOW \url{http://fs.oxfordjournals.org/content/68/1/78.full.pdf+html}
	\item Définition de la bande dessinée interactive \url{Bande_dessinee_interactive_Tony_Rageul-part1.pdf}
	\item IDPF (ePub3 \url{http://www.figoblog.org/node/2014} and \url{http://idpf.org/idpf-comics-manga-workshop-paris} and \url{http://idpf.org/digital-book-2014}
	\item Public domain \url{http://digitalcomicmuseum.com/}
	\item Thesis of Julien Falgas and Cohn
	\item Writing for Animation, Comics, and Games, chapter 5 and 6 in \url{~/Biblio_LINK/Bandes_dessinées}
	\item De la page à l'écran \url{2010_Boudissa_La_Bande_dessinee_entre_page_et_ecran.pdf}
	\item Check \url{/home/crigau02/Bureau/PhD/News/BD}
	
\end{itemize}

S'il est difficile de définir avec précision la bande dessinée, c'est qu'elle se situe précisément au carrefour de plusieurs moyens d'expression artistique: l'art graphique, l'art cinématographique et la littérature. Elle est tout à la fois dessin, cinéma, écriture, se conjuguant entre eux pour former un art nouveau, doté d'un ensemble de moyens d'expressions extrêmement complet et varié [...]~\cite{duc1997art}

Generic part: mixed content document or semi-structured documents

% Motivation -------------------------------------------------------------------------------------------------------------------------------------------
\section{Motivations}
The needs? numbers? impact on the society?

\begin{itemize}
	\item \url{http://fr.wikipedia.org/wiki/Bande_dessinee#Aspects_.C3.A9conomiques}
	\item La lecture des bandes dessinées en france \url{http://www.culturecommunication.gouv.fr/Politiques-ministerielles/Etudes-et-statistiques/Les-publications/Collections-de-synthese/Culture-etudes-2007-2014/La-lecture-de-bandes-dessinees-CE-2012-2} and \url{2012_Evans_La lecture de bandes dessinées_CE-2012-2.pdf}
	\item The Comics Chronicles (US market) is a free resource for academic research: \url{http://www.comichron.com/} and \url{http://comicsbeat.com/category/sales-charts/}
	\item Intro journal paper 2014
\end{itemize}

Comics or ``bande dessin{\'e}e'' represents an important part of the cultural heritage of many countries, especially in the US~\cite{Stewart2000,IBISWorld2013}, western Europe (particularly France and Belgium)~\cite{Ratier2013}, and Japan~\cite{Japan2013}.
Unfortunately, they did not yet received the same level of attention as music, cinema or literature about their adaptation to the digital format.
Indeed, while the latter entered to the common digital uses via the proliferation of music services and video on demand and quasi-systematic output of books in paper and electronic book formats, millions of works from the imagination of the comics authors still struggled to find an echo on the side of digitized world.
Ancient works could be reused with information technology to explore digital libraries~\cite{Back2001}, assist translators~\cite{borodo2014multimodality}, augmented reading~\cite{Singh2004,Raulet2013Comics}, speech playback for visually impaired~\cite{Brandon2014,Ponsard09}, story analysis, advertising etc.
Nevertheless, the process of conversion and adaptation is not as simple and straightforward as the one established for the digital publication of films and novels.
The comics differ from the latter in that the media itself is intimately linked to the medium.
Indeed, a film can be decomposed into a series images plus a soundtrack. Just watching these images in the right order and at the right frame rate allow to reconstruct the initial content, regardless of the medium. 
In the same way a novel is ultimately a sequence of words.
Reading these words in the correct order, on paper or on a screen does not change neither the content nor the artistic dimension of the work.
% The ``bande dessinée'' is defined as a series of ``paintings and stills deliberately juxtaposed in sequences'' by the reference work in the field of McCloud~\cite{McCloud94}.
The ``bande dessinée'' is defined as juxtaposed sequences of image by McCloud~\cite{McCloud94} and Thomas~\cite{Thomas2010Invisible}.

However, it differs from the films on the form and the spatial positioning of the images.
Where the latter pictures are all of equal size and each new image replaces the previous one, comic panels vary in size and spatial organisation in a limited space (paper sheet).
These two features, added to the fact that the reader has the opportunity to see all the boxes of a
same page, but not those of the next page, are tools at the service of the author to stage the story.
Therefore, changing the medium, the reading surface format or the sequence order involves a modification of the staging that may in some cases be detrimental to the story.

Several companies offer printed to digital format conversion services for comics, to facilitate reading on mobile devices.
However, the conversion process is both tedious because done by hand, and simplistic as it is too often reduced to the successive display of panel interspersed with user selected transitions.
The ideal would be to understand the process used by authors to draw the paper version of the comics and automatically change it in a form adapted to the medium in which the work is read (e.g. smartphone, web page, 3D book). 
It starts with an analysis of  the digitalized paper page to extract the different components (e.g. panel, balloon, text, comic character) and their relations (e.g. read before, said by, addressed to).
Once this initial work is done, it is necessary to reconstruct the story by placing the extracted elements in the initial order to keep the story coherent.



Why is this scientifically relevant?

The analysis of comic images by computer is particularly challenging because they are semi-structured document with mixed content.
Comic document are at the intersection between unstructured (e.g. teaching board~\cite{Oliveira10}, free-form document~\cite{Delaye2014Multi}) and complex background (e.g. advertising poster~\cite{Clavelli09}, real scene~\cite{Weinman09,Epshtein10,Neumann12}) images which are nowadays active fields of research for the community.
Being at the intersection of several fields of research increase the complexity of the problem.
This is one of the reason why the analysis of comics is a recent (in the document analysis history) and not solved field of research.



% Objective of this work, contributions --------------------------------------------------------------------------------------------------------
\section{Objectives and contributions}


% Outline --------------------------------------------------------------------------------------------------------------------------------------
\section{Outlines}
