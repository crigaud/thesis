\chapter{Introduction}% maximum 20 pages
\chaptermark{Introduction}
\label{chap:intro}
\graphicspath{{./chapters/1-introduction/figs/}}
% Overview of pattern recognition
\section{The evolution of the ``bandes dessinée''}
Setting/positioning the scenario, place comics in the addressed context, origin of comics/BD until new usages in 2014 created by new technologies.
\cite[p.~215]{McCloud94}

\begin{itemize}
	\item The origin of BD, comics, Manga, other

	\item see Clement's intro
	\item collectif PANIC book
	\item Évolution de la bande dessinée en France depuis quinze ans \url{http://ejournals.library.ualberta.ca/index.php/af/article/download/21313/16112}
	\item \url{http://fr.wikipedia.org/wiki/Bande_dessinee_en_ligne}

	\item \url{http://en.wikipedia.org/wiki/Comics}
	\item European Comic Art \url{http://journals.berghahnbooks.com/eca/}
	\item Webcomics \url{http://www.phdcomics.com} and \url{http://xkcd.com}
	\item Le manga: Une synthèse de référence qui éclaire en image (Eyrolles, 2013) \url{http://books.google.fr/books?id=3MqgAgAAQBAJ&printsec=frontcover#v=onepage&q&f=false}
	\item \url{http://en.wikipedia.org/wiki/Manga}
	\item Etat présent BANDE DESSINEE STUDIES LAURENCE GROVE UNIVERSITY OF GLASGOW \url{http://fs.oxfordjournals.org/content/68/1/78.full.pdf+html}
	\item Définition de la bande dessinée interactive \url{Bande_dessinee_interactive_Tony_Rageul-part1.pdf}
	\item IDPF (ePub3 \url{http://www.figoblog.org/node/2014} and \url{http://idpf.org/idpf-comics-manga-workshop-paris} and \url{http://idpf.org/digital-book-2014}
	\item Public domain \url{http://digitalcomicmuseum.com/}
	\item Thesis of Julien Falgas and Cohn
	\item Writing for Animation, Comics, and Games, chapter 5 and 6 in \url{~/Biblio_LINK/Bandes_dessinées}
	\item De la page à l'écran \url{2010_Boudissa_La_Bande_dessinee_entre_page_et_ecran.pdf}
	\item Check \url{/home/crigau02/Bureau/PhD/News/BD}
	\item Define comics vocabulary (see Clement's thesis and master student report) for panel, balloon, text, character
	
\end{itemize}

S'il est difficile de définir avec précision la bande dessinée, c'est qu'elle se situe précisément au carrefour de plusieurs moyens d'expression artistique: l'art graphique, l'art cinématographique et la littérature. Elle est tout à la fois dessin, cinéma, écriture, se conjuguant entre eux pour former un art nouveau, doté d'un ensemble de moyens d'expressions extrêmement complet et varié [...]~\cite{duc1997art}

Generic part: mixed content document or semi-structured documents

% Motivation -------------------------------------------------------------------------------------------------------------------------------------------
\section{Motivations}
The needs? numbers? impact on the society?

\begin{itemize}
	\item \url{http://fr.wikipedia.org/wiki/Bande_dessinee#Aspects_.C3.A9conomiques}
	\item La lecture des bandes dessinées en france \url{http://www.culturecommunication.gouv.fr/Politiques-ministerielles/Etudes-et-statistiques/Les-publications/Collections-de-synthese/Culture-etudes-2007-2014/La-lecture-de-bandes-dessinees-CE-2012-2} and \url{2012_Evans_La lecture de bandes dessinées_CE-2012-2.pdf}
	\item The Comics Chronicles (US market) is a free resource for academic research: \url{http://www.comichron.com/} and \url{http://comicsbeat.com/category/sales-charts/}
	\item See motivations in \url{PhD/Publications/Doctoral_school_L3i_first_year_phd_report}
	\item IGS-CP (http://www.igs-cp.fr), a content extraction companies working for digital comics promoting (e.g. espritBD, alterComics), ITEsoft

\end{itemize}

Comics or ``bande dessin{\'e}e'' represents an important part of the cultural heritage of many countries, especially in the US~\cite{Stewart2000,IBISWorld2013}, western Europe (particularly France and Belgium)~\cite{Ratier2013}, and Japan~\cite{Japan2013}.
Unfortunately, they did not yet received the same level of attention as music, cinema or literature about their adaptation to the digital format.
Indeed, while the latter entered to the common digital uses via the proliferation of music services and video on demand and quasi-systematic output of books in paper and electronic book formats, millions of works from the imagination of the comics authors still struggled to find an echo on the side of digitized world.
Ancient works could be reused with information technology to explore digital libraries~\cite{Back2001}, assist translators~\cite{borodo2014multimodality}, augmented reading~\cite{Singh2004,Raulet2013Comics}, speech playback for visually impaired~\cite{Brandon2014,Ponsard09}, story analysis, advertising etc.
Nevertheless, the process of conversion and adaptation is not as simple and straightforward as the one established for the digital publication of films and novels.
The comics differ from the latter in that the media itself is intimately linked to the medium.
Indeed, a film can be decomposed into a series images plus a soundtrack. Just watching these images in the right order and at the right frame rate allow to reconstruct the initial content, regardless of the medium. 
In the same way a novel is ultimately a sequence of words.
Reading these words in the correct order, on paper or on a screen does not change neither the content nor the artistic dimension of the work.
% The ``bande dessinée'' is defined as a series of ``paintings and stills deliberately juxtaposed in sequences'' by the reference work in the field of McCloud~\cite{McCloud94}.
The ``bande dessinée'' is defined as juxtaposed sequences of image by McCloud~\cite{McCloud94} and Thomas~\cite{Thomas2010Invisible}.

However, it differs from the films on the form and the spatial positioning of the images.
Where the latter pictures are all of equal size and each new image replaces the previous one, comic panels vary in size and spatial organisation in a limited space (paper sheet).
These two features, added to the fact that the reader has the opportunity to see all the boxes of a
same page, but not those of the next page, are tools at the service of the author to stage the story.
Therefore, changing the medium, the reading surface format or the sequence order involves a modification of the staging that may in some cases be detrimental to the story.

Several companies offer printed to digital format conversion services for comics, to facilitate reading on mobile devices.
However, the conversion process is both tedious because done by hand, and simplistic as it is too often reduced to the successive display of panel interspersed with user selected transitions.
The ideal would be to understand the process used by authors to draw the paper version of the comics and automatically change it in a form adapted to the medium in which the work is read (e.g. smartphone, web page, 3D book). 
It starts with an analysis of  the digitalized paper page to extract the different components (e.g. panel, balloon, text, comic character) and their relations (e.g. read before, said by, addressed to).
Once this initial work is done, it is necessary to reconstruct the story by placing the extracted elements in the initial order to keep the story coherent.



Why is this scientifically relevant?

The analysis of comic images by computer is particularly challenging because they are semi-structured document with mixed content.
Comic document are at the intersection between unstructured (e.g. teaching board~\cite{Oliveira10}, free-form document~\cite{Delaye2014Multi}) and complex background (e.g. advertising poster~\cite{Clavelli09}, real scene~\cite{Weinman09,Epshtein10,Neumann12}) images which are nowadays active fields of research for the community.
Being at the intersection of several fields of research increase the complexity of the problem.
This is one of the reason why the analysis of comics is a recent (in the document analysis history) and not solved field of research.



% Objective of this work, contributions --------------------------------------------------------------------------------------------------------
\section{Objectives and contributions}
Comics contain many heterogeneous elements that are hard to process in once.
Our objective is to process them separately, from simple to complex, in order to progressively build a complete comics understanding system.
% We decomposed the work into several steps.=
First, we construct a public dataset of comic images and the corresponding ground truth in order to evaluate our work and to give to the community the opportunity to work on identical data in order to make comparable and reproducible research.
Second, we focus on extraction panels, texts, balloons in order to facilitate the extraction of more complex elements by focussing on a region of interest instead of the whole image.
Third, we combine the extraction processes to reconstruct the context and the relations between the elements.
The dataset and this last work are the result of a collaboration with Cl{\'e}ment Gu{\'e}rin, Ph.D. student working on the same project.

To meet the above objectives, we have made the following contributions in this
thesis.

\begin{itemize}
	\item[1)] Dataset and ground truth: The eBDtheque dataset is the first publicly available\footnote{Dataset website:~\url{http://ebdtheque.univ-lr.fr}} dataset and ground truth of comic images.
	Such dataset is important for the community to make comparable, reproductive and growing research.
	The dataset consist in a mixture comic images coming from different albums with the goal of being as representative as possible of the comics diversity.
	The database consists of a hundred pages of various comic book albums including Franco-Belgian ``bande dessinée'', American comics and Japanese mangas. 
	The ground truth contains the spacial position of panels, balloons and text lines, comic characters and their associated semantic annotations, that appear in the images.
	Also, bibliographic information are given such for each image.
	This work is presented in Chapter~\ref{chap:gt}.
	
	\item [2)] Panel extraction: Comics are mixed content documents that require different techniques to extract different element.
	The first particularity of comics is the sequence of panels that we extract using connected component classification and filtering in Chapter~\ref{chap:pe}.
	
	\item [3)] Balloon detection: Balloons or bubbles are key elements in comics,
	they link graphical and textual elements and are part of the comics style. They can have various shapes (e.g. oval, rectangular) and contours (e.g. smooth, wavy, spiky, absent).
	In this work we propose a closed balloon extractor based on the analysis of the blob content, a active contour model to extract open balloons from text line position.
	Both methods are described in Chapter~\ref{chap:be} along with balloon contour classification (smooth, wavy, zigzag) and a tail detector.

	\item [4)] Text localization: Text is of different nature in comics, there are sound effects (onomatopoeias), graphic text (illustration), speech text (dialogues) and narrative text (captions). 
	Speech text represents the majority of the text present in comics~\cite{??}, we propose a adaptive binarisation process from a Minimum Connected Component Thresholding followed by a text/graphic separation based on contrast ratio and then a text line grouping algorithm.
	Finally, an OCR system filters out non text region.
	This work is explained in Chapter~\ref{chap:te}

	
	\item [5)] Comic character detection: Unsupervised comic character extraction is a difficult task as soon as we aim to process heterogeneous comic styles using the same algorithm.
	In this context, learning-based approaches are not reliable unless if we have enough data to train on all comic styles. 
	We first propose a query by example approach that asks the user to select a part of the object he is looking for in one comics image and the system spots other occurrences everywhere in all the pages of the comics album, assuming that they have been digitized under the same conditions.
	Second, we go one step further by refining the comic character location according to the contextual elements (e.g. panel contents, speech balloon position, tail direction) given a region of interest.
	% To do so, we replace the user query by an expert system using knowledge about the comics domain.
	These works are presented in Chapter~\ref{chap:ce}.

	\item [6)] Comics understanding: Making a computer understand the complete story of a comics is a really challenging task, especially because it is even hard for human sometimes.
	Putting comics domain knowledge in an ontology-based framework enable to interact between image processing and semantic information in order to progressively understand the content of a document.
	This work is introduced in Chapter~\ref{chap:hp}.
	% We apply this framework to comics understanding in order to extract panels, balloons, texts, comic characters and their semantic relations in an unsupervised way.


\end{itemize}


% Outline --------------------------------------------------------------------------------------------------------------------------------------
\section{Outlines}

The rest of the thesis is organized as follows:
\begin{itemize}
% \item \ch{chap:gm} presents some definitions, concepts of graph theory, particularly the ones that are necessary for our work to be described later. After that we present a review of the state-of-the-art methods for graph and subgraph matching.
\item \ch{chap:sota} presents a detailed review of the state-of-the-art methods for the analysis of comic images. This chapter details several image processing methods in the four first subsections and then we review the holistic understanding systems that have been applied to document analysis so far and the more advanced application on the market.
% and gives an overview, pros and cons and examples of different categories of methods.

\item \ch{chap:gt} presents the dataset and ground truth of comic images that we provided to the community. The idea of this dataset consists in a mixture images coming from different albums with the goal of being as representative as possible of the comics diversity.
It also describe the indexation structure in a Scalable Vector Graphics (SVG) format.

\item \ch{chap:pe} introduces a fast panel extraction method by classifying the connected components into three classes for panel, text and noise.
This method can also be used for text/graphic separation.   %presents a subgraph matching method based on tensor product graph and a symbol spotting methodology is proposed using that. To cope with the structural errors a new dual graph based representation is proposed which is proved to be effective in graphical documents.

\item \ch{chap:be} introduces both closed and non closed balloon localisation and segmentation methods together with contour classification and tail detection and description. %introduces near convex region adjacency graph (NCRAG) which solves the limitations of the basic region adjacency graph (RAG) in graphical documents.

\item \ch{chap:te} addresses the difficulty of text localization and recognition in comics.
It propose speech text localisation method using connected components alignment and neighbourhood similarity to form text lines.
Text recognition is also addressed and preliminary results are detailed. %the structural errors encountered in the graphical documents. To solve the problem it proposes a hierarchical graph representation which can correct the errors/distortions in hierarchical steps.

\item \ch{chap:ce} presents two methods for comic character detection.
The first method overcomes the difficulties of such non rigid object detection thanks to the proposed descriptor invariant to scale, object deformation, translation and rotation transformations.
The second method takes profit of the contextual information to define the region of interest of comic characters.%presents a unified experimental evaluation of all the proposed methods and comparisons with some state-of-the-art methods. This is done in a symbol spotting experimental framework.

\item \ch{chap:hp} presents a system that combines low and high level processing to build a scalable system of comics image understanding reaching the level of semantic interaction between elements.
This work is a collaboration with Cl{\'e}ment Gu{\'e}rin, Ph.D. student working on data mining applied to comic documents.

\item \ch{chap:conclusions} concludes the thesis and defines the future direction of comics document analysis.

% \item \app{app:datasets} %provides a brief description on the datasets that we have used in this thesis work and \app{app:perf-eval} describes a performance evaluation protocol for symbol spotting systems that we follow for the experimental evaluation.
\end{itemize}

