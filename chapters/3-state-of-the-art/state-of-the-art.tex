\chapter{State-of-the-art} % about 20-30 pages
\chaptermark{State-of-the-art of comic books analysis}
\label{chap:sota}
\graphicspath{{./chapters/3-state-of-the-art/figs/}}

% This thesis work has considered symbol spotting problem as an application of subgraph matching. Symbol spotting has experienced a lot of interests among the graphics recognition community. Here the main task starts by querying a given graphic symbol or model object, usually cropped from a bigger document. The main aim is to search the model object in a target document or set of target documents. In this dissertation we will alternatively name the model object to be queried as \emph{model symbol} or \emph{query symbol} or \emph{pattern} and the corresponding graph representing them as \emph{query graph} or \emph{model graph} or \emph{pattern graph}. And the document or the set of documents where the user intends to find the model symbol as \emph{input document} or \emph{target document} and the corresponding graph as \emph{target graph} or \emph{input graph}. Most of the time the user is interested in getting a ranked list of retrieved zones supposed to contain the queried symbol depending on the similarity or dissimilarity measure. This Chapter contains a review of the state-of-the-art of symbol spotting methods. The major existing research can be classified into five broad families as in~\cite{RusinolThesis2009}, which are listed in Table~\ref{table:sota-ss:relworks}. We review those families as follows:

\section{Layout analysis and panel extraction}
\label{sec:sota:layout_panel}


\section{Balloon segmentation}
\label{sec:sota:balloon_segmentation}


\section{Text extraction and recognition}
\label{sec:sota:text}


\section{Comic character detection}
\label{sec:sota:comic_character}

\section{Holistic understanding} % (fold)
\label{sec:sota:holistic_understanding}

% section holistic_understanding (end)

\section{Existing applications}
\label{sec:sota:applications}


\section{Conclusion}
\label{sec:sota:conclusion}

% To conclude the literature review, some of the challenges of symbol spotting can be highlighted from the above state-of-the-art reviews. First, symbol spotting is concerned with various graphical documents viz. electronic documents, architectural floorplans etc., which in reality suffer from \emph{noise} that may come from various sources such as low-level image processing, intervention of text, etc. So efficiently handling structural noise is crucial for symbol spotting in documents. Second, an example application of symbol spotting is to find any symbolic object from a large amount of documents. Hence, the method should be efficient enough to handle a \emph{large database}. Third, symbol spotting is usually invoked by querying a cropped symbol from some document, which acts as a query to the system. So it implies infinite possibilities of the query symbols, and indirectly \emph{restricts the possibility of training} in the system. Finally, since symbol spotting is related to real-time applications, the method should have a \emph{low computational complexity}. We chose these five important aspects (segmentation, robustness in noise, training free, computational expenses, robustness with a large database) of symbol spotting to specify the advantages and disadvantages of the key research, which is listed in Table~\ref{table:sota-ss:relworkscomp}.

% In the next chapter, we are going to propose the first symbol spotting algorithm. The method is based on the factorization of graph into serialized subgraphs (graph paths) and then organizing those graph paths into hash table. The hash based technique helps to reduce search space considerably and perform the subgraph matching (symbol spotting) faster.