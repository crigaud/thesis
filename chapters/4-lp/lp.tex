\chapter{Comics image processing}
\chaptermark{Comics image processing}
\label{chap:lp}
\graphicspath{{./chapters/4-lp/figs/}}

%Abstract-------------------------------------------------------------------------------------------------------------
%In this chapter we propose a symbol spotting technique in graphical documents. Graphs are used to represent the documents and an error tolerant (sub)graph matching technique is used to detect the symbols in them. We propose a graph serialization to reduce the usual computational complexity of graph matching. Serialization of graphs is performed by computing acyclic graph paths between each pair of connected nodes. Graph paths are one dimensional structures of graphs, handling which is less expensive in terms of computation. At the same time they enable robust localization even in the presence of noise and distortion. Indexing in large graph databases involves a computational burden as well. We utilize a graph factorization approach to tackle this problem. Factorization is intended to create a unified indexed structure over the database of graphical documents. Once graph paths are extracted, the entire database of graphical documents is indexed in hash tables by locality sensitive hashing (LSH) of shape descriptors of the paths. The hashing data structure aims to execute an approximate $k$-NN search in a sub-linear time. We have performed detailed experiments with various datasets of line drawings and the results demonstrate the effectiveness and efficiency of our technique.
%----------------------------------------------------------------------------------------------------------------------
\section{Introduction}
\label{sec:hssg:intro}



\section{Panel extraction}
\label{sec:hssg:meth}


\section{Balloon segmentation}
\label{ssec:hssg:frwrk}


% \subsection{Closed balloons}
% \label{ssec:lp:balloon_closed}


% \subsection{Open balloons}
% \label{ssec:lp:balloon_open}


\section{Text extraction}
\label{ssec:hssg:pthdesc}



\section{Character spotting and identification}
\label{ssec:hssg:vot}


\section{Conclusions}
\label{sec:hssg:concl}

%In this chapter we have proposed a graph based approach for symbol spotting in graphical documents. We represent the documents with graphs where the critical points detected in the vectorized graphical documents are considered as the nodes and the lines joining them are considered as the edges. The document database is represented by the unification of the serialized substructures of graphs. Here the graph substructures are the acyclic graph paths between each pair of connected nodes. The factorized substructures are the one-dimensional (sub)graphs which give efficiency in terms of computation and since they provide a unified representation over the database, the computation is substantially reduced. Moreover, the paths give adaptation to some structural errors in documents with a certain degree of tolerance. We organize the graph paths in hash tables using the LSH technique, this helps to retrieve symbols in real time. We have tested the method on different datasets of various kinds of document images and the results are quite encouraging.

%In the next chapter we are going to propose a subgraph matching algorithm based on tensor product graph (TPG) (see \sect{sec:gm:pg} for details). Continuous optimization is a very popular approach in (sub)graph matching but it mostly works with pairwise measurements. But often pairwise quantifications are not reliable, to remove this problem in \ch{chap:pg} we propose walk based propagation of pairwise similarities to obtain contextual information incorporated in the higher order similarity measures. Then we formulate the subgraph matching problem as a node, edge selection problem in TPG. Also in \ch{chap:pg}, a \emph{dual edge graph} representation is proposed which achieves spatial relationship between the graph paths which was absent in this chapter.