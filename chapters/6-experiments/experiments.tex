\chapter{Experimentations}
\chaptermark{Experimentations}
\label{chap:experimentations}
\graphicspath{{./chapters/6-experiments/figs/}}


% \modif{TODO add total number of annotated elements for each category and report the diversity of style, format, publishing from the IJDAR paper}

% Abstract-----------------------------------------------------
With the analysis and processing of data comes the need of the output results evaluation.
Traditionally, this evaluation is made by validating the results of an algorithm with a ground truth that represents what an ideal output should be\cite{pascal-voc-2012, smeaton2006evaluation, griffinHolubPerona}.
Ideally, such a ground truth is made publicly available so anyone can challenge his own algorithm to the community \cite{lamiroy:inria-00537035}.
This can be applied to any kind of results from image segmentation to classification or information retrieval.
In this chapter we present the dataset and ground truth we provided to evaluate comic book image related works.
We describe the metrics we used to evaluate each contribution: Method A, B and C presented \ch{chap:sequential}, \ch{chap:independent} and \ch{chap:knowledge} respectively and evaluate our work compared to previous works from the literature.
%The evaluation of Method C is decomposed in two parts: the validation and the inference processes.
% Validations are combined with Method A and B in each extraction evaluation to show its benefits.
% The inference part of Method C can not be directly compared to low level information extraction, hence we discuss it in a proper section (Section~\ref{sec:semantic_links_evaluation}).

% compare our work to previous approaches from the literature.

% We evaluate the contributions of the thesis at different levels.
% We first evaluate the sequential, independent and knowledge-driven approaches separately and then we compare them in the global evaluation section.


% \section{Introduction} % (fold)
% \label{sec:ex:introduction}

% TODO

% section introduction (end)

\section{Dataset and ground truth construction} % (fold)
\label{sec:dataset_and_ground_truth_construction}

Being in need of comic books material and an associated ground truth to evaluate our work, we noticed that there is not such dataset publicly available for scientific purpose. 
Therefore, we decided to gather the first publicly available comic books dataset in association with several comic books authors and publishers and to build up the corresponding ground truth according to document analysis and understanding concerns.
% which are image segmentation and semantic analysis.
The comic book images were selected to cover the huge diversity of comic styles with the agreement of the consenting authors and publishers that had the objective to foster innovation in this domain trough academic research (Appendix~\ref{app:dataset}).

% %%%%%%%%%%%%%%%%%%%%%%%%%%%%%%%%%%%%%%%%%%%%%%%%%%
% \begin{figure}[ht]  %trim=l b r t  width=0.5\textwidth,
%    \centering
%   \includegraphics[trim= 4px 0px 0px 0px, clip, width=230px]{comics_diversity.png}
%   \caption[Examples of comic panels that reflect the diversity of comic books of the dataset]{Examples of comic panels that reflect the diversity of comic books of the dataset.}
%   \label{fig:ex:comics_diversity}
%  \end{figure}

% %%%%%%%%%%%%%%%%%%%%%%%%%%%%%%%%%%%%%%%%%%%%%%%%%%

It took almost one year to define which type of comics are the most interesting for researchers, meet and convince comic book authors and publishers, get copyright authorizations for the scientific community, develop a specific annotation tool and finally to hire people to do the ground truth (manual task).

A selection of hundred comic pages were annotated in one day by twenty volunteers affiliated to the L3i lab.
In order to provide a common basis for evaluating research work, the ground truth have been published in 2013~\cite{Guerin2013} and made available to the scientific community through a dedicated website\footnote{\url{http://ebdtheque.univ-lr.fr}}.
It has been enriched in 2014 by adding the location of the principal comic characters and semantic information to the already annotated elements.

The content of the dataset and its terms of use, the ground truth construction protocol and its quality assessment are detailed in the four next sub sections.



% \section{Structure and indexation}
% \label{sec:gt:structure_indexation}
\subsection{Dataset description} % (fold)
\label{sec:dataset_description}

% section dataset_description (end)

Scott McCloud defined comics as ``juxtaposed pictorial and other images in deliberate sequence, intended to convey information and/or to produce an aesthetic response in the viewer''~\cite{mccloud1994understanding}.
This definition is intentionally broad enough to encompass the spectrum of the majority of works produced so far.
The dataset composition should reflects this heterogeneity to give everyone the opportunity
to compare its algorithms to a globally representative dataset of the comics world.
We contacted authors with different comic styles and have selected a corpus of one hundred images, representing single or double comics page.

The images were partly processed by the French company A3DNum\footnote{\url{http://www.a3dnum.fr}} which was commissioned to digitize 14 albums.
Among all the files, scanned at a resolution of 300 dots per inch and encoded in uncompressed Portable Network Graphic (PNG) format, we used 46 pages to integrate the eBDtheque corpus.
The remaining 54 images were selected from webcomics, public domain comics\footnote{\url{http://digitalcomicmuseum.com}} and unpublished artwork with different styles from 72 to 300 dots per inch.
We encoded all the images of the eBDtheque dataset in Joint Photographic Experts Group (JPEG) format with a lossy compression to facilitate file exchange.
%, the non compressed images are available on request.

Hereafter we describe the characteristics of the selected pages and their content.


\paragraph{Albums} % (fold)
 \label{par:albums}
 published between 1905 and 2012.
 29 pages were published before 1953 and 71 after 2000.
 Quality paper, colour saturation and textures related to printing technique changes can vary a lot from one image to another.
 The artworks are mainly from France (81\%), United States (13\%) and Japan (6\%).
 Their styles varies from classical Franco-Belgium ``bandes dessinées'' to Japanese manga through webcomic and American comics.

 % paragraph albums (end)

\paragraph{Pages} % (fold)
\label{par:pages}
 themselves have very diverse characteristics.
Among all, 72 are printed in colours and according to the authors and periods, there are a majority of the tint areas, watercolours and hand-coloured areas.
Among the remaining 28, 16 have are greyscale and 12 are simply black and white.
One album has two versions of each page, one in colour and the other one in black and white.
We have integrated an examples of each of them in order to allow performance comparison of algorithms on the same graphic style by using colour information or not.
Five of the 100 images are double page, others are single page and 20\% are not A4 format.
% We therefore, strictly speaking, 105 and 100 pages not in our database, each with a distinct structure
% paragraph pages (end)

\paragraph{Panels} % (fold)
\label{par:panels}
 contained in the pages are of various shapes.
Although most of them are bounded by a black line, a significant proportion has at least one part of the panel which is indistinguishable from the background of the page (frame less panel).
Two pages consist only of frame less panels, the visual delimitation uses background contrast difference between the panel and image.
Nine images contain overlapping panels, twelve contain only panels without border and several has panels connected by an straddle object.

% paragraph panels (end)

\paragraph{Balloons} % (fold)
\label{par:balloons}
 also contain a great diversity.
Some of them are completely surrounded by a black stroke, some partially and others not at all.
They have a bright background with a rectangular, oval or non geometric shape with ``smooth'', ``wavy'' or ``spiky'' contour in general.
Most of them has a tail pointing towards the speaker, but some do not.
There is text without any surrounding balloons on 33 images of the corpus.
% and 8 of them simply contain no bubble.

% paragraph balloons (end)

\paragraph{Text} % (fold)
\label{par:text}
 is either typewritten (61\% of the image) or handwritten, mainly upper-case.
The text lines contains 12 elements in average (Figure~\ref{fig:ex:textline_lenth_distribution}) and there are more than hundred text lines that are composed by only one letter corresponding to punctuation or single letter words such as ``I'' or ``A''; this is a particularity of comics.

Most pages are from French artworks, where the text is written in French.
Only 13 pages contain English text and 6 images are in Japanese.
Onomatopoeia appears in 18 pages.

    %%%%%%%%%%%%%%%%%%%%%%%%%%%%%%%%%%%%%%%%%%%%%%%%%%%%%%%%
    \begin{figure}[ht]%trim=l b r t  width=0.5\textwidth,  
      \centering
      \includegraphics[trim= 0px 5px 65px 5px, clip, width=0.85\textwidth]{number_of_letter_per_textline.png}
      \caption[Distribution of the number of elements per text lines]{Distribution of the number of elements per text lines.
      }
      \label{fig:ex:textline_lenth_distribution}
    \end{figure}  
    %%%%%%%%%%%%%%%%%%%%%%%%%%%%%%%%%%%%%%%%%%%%%%%%%%%%%%%%

% paragraph text (end)

\paragraph{Comic characters} % (fold)
\label{par:comic_characters}
or protagonist are specific to each album.
They all have eye, harm and leg but at least 50\% are not humanoid, depending on the interpretation.
% paragraph comic_characters (end)

\subsection{Terms of use} % (fold)
\label{sub:term_of_use}
We obtain the minimum rights for sharing and publishing image material from the right holders but we had to make sure the user accept it before using the data.
In collaboration with the intellectual property department of the University of La Rochelle, we established the following:

\textit{In order to use this database, you must firstly agree to the terms.
You may not use the database if you don't accept the terms.
The use of this database is limited to scientific and non-commercial purpose only, in the computer science domain.
For instance, you are allowed to split the images, through the use of segmentation algorithms.
You can also use pieces of this database to illustrate your research in publications and presentations.
Any other use case must be validated by our service.
If you do agree to be bound by all of these Terms of Use, please fill and email the request form and then use the login and password provided to download the selected version below.}

The concerned request form requires the identity, affiliation, address and intended use of the person who wishes to use data.

% subsection term_of_use (end)

\subsection{Ground truth construction} % (fold)
\label{sec:ground_truth_construction}

The ground truth has been defined in accordance to existing formalism in order to fulfil the needs of a large amount of researchers related to comics material.
It integrates low and high level information such as spatial position of the elements in the image, their semantic links and also bibliographic information.

In order to cover a wide range of possible research matters, it has been decided to extract three different type of objects from the corpus: text lines, balloons and panels.
We decided to do this first ground truth by drawing horizontal bounding boxes as close as possible from the feature and including all its pixels.
% in order to proceed a maximum of pages in the allowed time. 
We chose this level of granularity in order to limit the subjectiveness of the person making the annotation.

The comics art is extremely heterogeneous and our dataset voluntarily integrates albums that can be classified as
unconventional.
This leaves room for interpretation on the form which increase annotation variations by different people and decrease the uniformity of the ground truth.
This precision level is used in several, widely used datasets~\cite{pascal-voc-2012, yao2007introduction}.
%{\color{blue}This groundtruthing level is also used for well known dataset widely used~\cite{pascal-voc-2012, yao2007introduction}.}

Hereafter we detail the two levels of annotation (visual and semantic) that form the ground truth and how they are indexed in a file.
The combination of visual and semantic annotation provides the advantage of making this ground truth relevant for document analysis and semantic evaluation which are both part of the comics understanding process.

% each element to annotate and the protocol to follow.
% follows specific guidelines according to the rules below:

\subsubsection{Visual annotation} % (fold)
\label{sub:visual_annotation}
The first annotation consist in defining the spacial region where elements are located in the image.
We describe hare how the visual annotation have been performed for the panels, balloons, texts and comic characters.

% subsection visual_annotation (end)
\paragraph{Panels} % (fold)
\label{par:panels}
The frame or panels are defined as an image area, generally rectangular, representing a single scene in the story. 
There is always at least one panel per page, the entire page region can be used as panel if necessary.
When a panel has a black border, the bounding box is placed as close as possible to its frame.
Sometimes, images have not been scanned perfectly horizontally, it is then impossible to have an horizontal bounding box sticking exactly to the border.
When the panel border is partially absent or suggested by the neighbourhood, the bounding box just defines the content of the panel.
In all cases, the other elements (balloon, text, drawings) extending from the frame are truncated (Figure~\ref{fig:gt:segPanel}).

%%%%%%%%%%%%%%%%%%%%%%%%%%%%%%%%%%%%%%%%%%%%%%%%%%%%%%%%%
\begin{figure}[h!]
\begin{center}
\includegraphics[width=0.8\textwidth]{segPanel.png}
\caption[Panel annotation]{Example of three panel annotations. The bounding box (transparent red) is defined without taking into account of non panel elements in all cases.}
\label{fig:gt:segPanel}
\end{center}
\end{figure}
%%%%%%%%%%%%%%%%%%%%%%%%%%%%%%%%%%%%%%%%%%%%%%%%%%%%%%%%%

%They can be frame-less, in that case the bounding box will be set according to the contained drawings, see Fig. \ref{fig:segmentation}c. %, ignoring text and overlapping features.
%In both cases, text and overlapping features are ignored.
%There is necessarily at least one panel in a page. 
% paragraph panels (end)

% The panel reading order is also annotated in a metadata called \texttt{rank}, it is in integer  between $1$ and $n$ which is increased incrementally from the first to the last panel $n$ in an image.%, le premier de la séquence prenant la valeur $1$ et le dernier la valeur $n$ pour une page contenant $n$ cases.

\paragraph{Balloons} % (fold)
\label{par:balloons}
We define a balloon (phylactery or bubble) as the region of an image including one or several lines of text, graphically defined by an identifiable physical boundary or suggested by the presence of an arrow pointing to the speaker (the tail).
Although rare, empty balloons (not containing lines of text) are also annotated if they are clearly identifiable by their shape or
the tail representation.
Pixel level annotation follows the contour of the balloon (Figure~\ref{fig:gt:segBalloons}c), while bounding box annotation does not consider the contour of phylactery and truncates the tail.
Sometimes it crosses the entire panel and generates an unrepresentative position of the desired balloon (Figure~\ref{fig:gt:segBalloons}a).
When the balloon is not closed (e.g. open contour) the annotated contour has to stick as close as possible to the contained text (Figure~\ref{fig:gt:segBalloons}b).
Note, the first version of the ground truth (2013) have been defined at bounding box level ignoring the tail and the second version (2014) at pixel level following the contour variations and the tail region.

%%%%%%%%%%%%%%%%%%%%%%%%%%%%%%%%%%%%%%%%%%%%%%ù
\begin{figure}[h!]
\begin{center}
\begin{tabular}{ccc}
a) \includegraphics[width=80px]{segBalloon1.png} 
& 
b) \includegraphics[width=80px]{segBalloon2.png}
&
c) \includegraphics[width=80px]{segBalloon3.png}
\end{tabular}
\caption[Speech balloon contour annotation]{Example of balloon clipping: a) using bounding box excluding the tail, b) bounding box of non closed balloon, c) pixel level contour annotation.} 
\label{fig:gt:segBalloons}
\end{center}
\end{figure}
%%%%%%%%%%%%%%%%%%%%%%%%%%%%%%%%%%%%%%%%%%%%%%ù

% paragraph balloons (end)

\paragraph{Text lines} % (fold)
\label{par:text_lines}
The text lines are defined as a sequence of text characters aligned in the same direction (Figure~\ref{fig:gt:segLines}a).
This definition encompasses both speech texts and narrative text, often located inside balloon, onomatopoeia (graphic sound) that are written or drawn directly in the panel without particular container.
%The text of the latter, although they are occasionally parallel to the edges of the box is still clipped by a horizontal bounding box for consistency across the ground truth.
Comics are static graphics, the expression of emotions of a comic character is the joint action of drawing and text, sometimes in the form of a single punctuation symbol.
For instance, an exclamation mark for surprise or a question mark for a misunderstanding.
These isolated symbols convey information and are segmented as text line as well (Figure~\ref{fig:gt:segLines}b).
Similarly, we have chosen to include in this category the illustrative text, such as a road sign or storefront (Figure~\ref{fig:gt:segLines}c).
Although at the boundary between text and graphic, these elements are still invariably read by the reader and their annotation is potentially interesting for multiple purposes, including story and scene analysis.

%%%%%%%%%%%%%%%%%%%%%%%%%%%%%%%%%%%%%%%%%%%%%%ù
\begin{figure}[h!]
\begin{center}
\begin{tabular}{ccc}
a) \includegraphics[width=80px]{segLine1.png} 
& 
b) \includegraphics[width=80px]{segLine2.png}
&
c) \includegraphics[width=80px]{segLine3.png}
\end{tabular}
\caption[Text line location annotation]{Different examples of text line position annotation: a) a classical text line in a speech balloon, b) a unique symbol, c) illustrative text, non horizontal.} 
\label{fig:gt:segLines}
\end{center}
\end{figure}
%%%%%%%%%%%%%%%%%%%%%%%%%%%%%%%%%%%%%%%%%%%%%%ù

% paragraph text_lines (end)

\paragraph{Comic characters} % (fold)
\label{par:comic_characters}
The comic characters position have been included in the second version of the ground truth only (2014).
The concept of ``character'' may have different interpretations when used for comics and must be specified.
Characters in a comic have not necessarily a human-like, or even living beings appearance.
Even so, it would be appropriate to annotate every human being appearing in a box while some are nothing but a part of the scenery.
Therefore, we have chosen to limit the annotation to the comic characters that emits at least one speech balloon in the album (minimal impact in the story).
Their bounding box has been defined to maximize the region occupied by the comic character inside the box region.
Therefore, some parts of the character such as arms or legs, are clipped sometimes (Figure~\ref{fig:gt:segCharacter}).

%%%%%%%%%%%%%%%%%%%%%%%%%%%%%%%%%%%%%%%%%%%%%%
\begin{figure}[h!]
\begin{center}
\includegraphics[width=0.75\textwidth]{segCharacter.png}
\caption[Comic character position annotation]{Example of comic character annotation: sniper's arms are not included in the bounding box in order to maximize the region occupied by the sniper in its bounding box. The two snipers and the two characters in the car are annotated because they emit a speech balloon in a different panel.}
\label{fig:gt:segCharacter}
\end{center}
\end{figure}
%%%%%%%%%%%%%%%%%%%%%%%%%%%%%%%%%%%%%%%%%%%%%%

% paragraph comic_characters (end)


% section ground_truth_construction (end)

\subsubsection{Semantic annotation} % (fold)
\label{sub:gt:semantic_annotation}
This second level of annotation complete each spacial region with additional information about its semantic.
Also, the image itself is annotated with extra information about its origin (e.g. album, collection, author, publisher).
%Once segmented, each object is annotated with a set of predefined metadata as follows. 

\paragraph{Images}
The image, often assimilated as pages, has been annotated with bibliographical information, so that anyone using this ground truth is free to get is own paper copy of the comic books for extra uses.
The first annotation is the page number (\texttt{{pageNumber}}) then the comic book title, from which the page has been picked up, and its release date (\texttt{{albumTitle, releaseDate}}), the series it belongs to (\texttt{{collectionTitle}}), the authors and editor names (\texttt{{writerName, drawerName, editorName}}) and, finally, the website and/or ISBN (\texttt{{website, ISBN}}).
The album title is not mandatory for webcomics.
Structural information about the page content has been added as well, such as resolution (\texttt{{resolution}}), reading direction (\texttt{{readingDirection}}), main language of the text (\texttt{{language}}) and single or double page information (\texttt{{doublePage}}).
% These annotation are stored in a tag named \texttt{{Pages}}.

\paragraph{Panels} 
The panels are annotated with a \texttt{{rank}} metadata which stand for its position in the reading sequence. 
The first panel to be read on a given page has its rank property set to 1, while the last one is set to \textit{n}, where \textit{n} is the number of panels in the page.

\paragraph{Balloons}
Balloons are also annotated with a \texttt{{rank}} property that defines their reading order relatively to the image because balloons are not always included in panels.
%, their rank is set according to the page as a whole. 
For a page containing $m$ balloons, the first balloon's rank will be 1 and the last will be $m$.
%Moreover, two additional metadata are given. 
A second information concerns the \texttt{{shape}} of the balloon.
This feature conveys an information about how the contained text is spoken (tone).
The type of \texttt{{shape}} is given from the following list \{\texttt{smooth, wavy, spiky, suggested, other}\} as pictured in Figure~\ref{fig:gt:balloonShape}.
Finally, the tail tip position (extremity) and its pointing direction have been added into the second version of the ground truth.
There are given through the \texttt{{tailTip}} and \texttt{{tailDirection}} properties.
The possible values of the direction are reduced to the eight cardinal directions plus a ninth additional value for the lack of tail: \{\texttt{N, NE, E, SE, S, SW, W, NW, none}\}.
In the second version of the ground truth (2014), we added the identifier of the comic character which is emitting the balloon \texttt{{idCharacter}}. 

%%%%%%%%%%%%%%%%%%%%%%%%%%%%%%%%%%%%%%%%%%%%%%%%%%%
\begin{figure}[h!]
\begin{center}
\includegraphics[width=0.65\textwidth]{balloonShape.png}
\caption[Balloon contour styles]{Balloon contour styles, from top-left to bottom-right: cloud, spiky, suggested and two smoothed contour type. Note that the two last ones are not labelled as rectangle or oval because here we annotated the type of contour and not shape (Section~\ref{sub:in:balloon_classification})}
\label{fig:gt:balloonShape}
\end{center}
\end{figure}
%%%%%%%%%%%%%%%%%%%%%%%%%%%%%%%%%%%%%%%%%%%%%%%%%%%

\paragraph{Text lines}
The text lines are associated with their transcription and the identifier of the corresponding balloon is added in \texttt{{idBalloon}} if the text line is included in a balloon.
In a second step, the function of each text line was specified through metadata \texttt{textType}. 
We identified six distinct categories of text.
The text that is used to tell the story and that can be either spoken (\texttt{speech}) or thought (\texttt{thought}) or narrated (\texttt{narrative}).
On the other hand, there are textual information that bring timely and contextual information, they are the onomatopoeia (\texttt{onomatopoeia}) and drawn text (\texttt{illustrative}) as part of the drawing such as license plate, storefront, brand.
The sixth category have been defined as notes (\texttt{note}) for embedded text in the page, such as the signature of the author, the page number, the title (every other readable text type).
%Another information have been added in 2014: the \texttt{textType} that specifies if the text is of type \{\texttt{speech, thought, onomatopoeia, narrative, illustrative, note}\}

\paragraph{Comic characters} % (fold)
\label{par:comic_characters}
The comic character are identified by \texttt{{idCharacter}} in order to be easily referred.

% paragraph comic_characters (end)

% \paragraph{Speaking character and speech balloon} % (fold)
% \label{par:speaking_character_and_speech_balloon}
% In the second version of the ground truth, we added a new type of relation which is between regions.
%  balloons and comic characters regions that are related in the story (balloon spoken by a character).
% The information is stored in a structure called \texttt{{LinkSBSC}} that consist a pair of identifier \texttt{{idBalloon}} and \texttt{{idCharacter}}.
% paragraph speaking_character_and_speech_balloon (end)


%%%%%%%%%%%%%%%%%%%%%%%%%%%%%%%%%%%%%%%%%%%
% \begin{figure}
% \begin{center}
% \includegraphics[width=0.45\textwidth]{balloons_shape3.png}
% \caption{Different speech balloon shapes. Top-down, from left to right: cloud, peak, suggested, rectangular and oval.}
% \label{fig:balloons_shape}
% \end{center}
% \end{figure}
%%%%%%%%%%%%%%%%%%%%%%%%%%%%%%%%%%%%%%%%%%%

% subsection semantic_annotation (end)

\subsubsection{File structure} % (fold)
\label{sub:file_structure}

The ground truth file structure have been thought according to existing comics related formalism such as Comics Markup Language (ComicsML)~\cite{McIntosh2011}, Comic Book Markup Language (CBML)~\cite{Walsh2012a}, Periodical Comics\footnote{\url{http://www.w3.org/wiki/WebSchemas/PeriodicalsComics}}, A Comics Ontology~\cite{Rissen2012}, Advanced Comic Book Format (ACBF)\footnote{\url{https://launchpad.net/acbf}} and the Grand Comics Database (GCD)\footnote{\url{http://www.comics.org}}.
See the Ph.D. thesis of Gu{\'e}rin~\cite{phdthesisGuerin14} for an extended review.

As we wanted to keep the ground truth file system simple and easy to share, visual and semantic annotations about a given page are gathered in a single full-text file following the specifications of Scalable Vector Graphics (SVG). 
Besides being an open-standard developed by the World Wide Web Consortium (W3C) since 1999, the SVG format fulfils two essential needs for this database.

First, using a recent Internet browser or your favourite image viewer, it provides a simple, fast and elegant way to display the visual annotation of any desired object over a comic book page using layers.
No need to install software such as Matlab, Adobe Illustrator or equivalent open source to visualize the ground truth information.

It is XML-based vector image format that allows to display an animate the annotated region, stored as polygon object in the SVG file, as desired using the Cascading Style Sheets (CSS) properties (Figure~\ref{fig:gt:svgImage}). 

%%%%%%%%%%%%%%%%%%%%%%%%%%%%%%%%%%%%%%%%%%%%%%%%%

\begin{figure}[h!]
\begin{center}
\begin{tabular}{ccc}
a) \includegraphics[width=100px]{svgPanel.png} 
& 
b) \includegraphics[width=100px]{svgBalloon.png}
&
c) \includegraphics[width=100px]{svgTextlines.png}
\end{tabular}
\caption[Annotation rendering in a browser]{Example of rendering for each class of element. For example, red for panels (a), cyan for balloons (b) and green for text (c). The opacity is set to 50\% to allow seeing the corresponding image by transparency.} 
\label{fig:gt:svgImage}
\end{center}
\end{figure}
%%%%%%%%%%%%%%%%%%%%%%%%%%%%%%%%%%%%%%%%%%%%%%%%%%

Each layer can be displayed or not in order to enhance the clearness of the annotations when browsing the database.
Secondly, SVG being a XLM-based language, it makes the integration of semantic annotation very easy via the use of the predefined \texttt{metadata} element.

One ground truth file contains the complete description of one comics image. 
There is no hierarchical link between pages from a same comic book. 
Following the basic XML encoding information, a SVG file starts with a root \texttt{<svg>} element containing the title of the document, \texttt{<title>}, and five \texttt{<svg>} children with different class attributes.
These contain annotations collected on five types of elements which are the page, panels, balloons, text lines and comic characters.
The type of element in a tag is specified by its \texttt{class} attribute.
The first tag, \texttt{class = ``Page''} contains description on the image and has two daughters.
The first one, \texttt{image} has several attributes which specifies a link to the image file in the dataset \texttt{xlink: href} and two specifying the \texttt{width} and \texttt{height} of the image.
The second, \texttt{metadata}, contains bibliographic information about the album and page properties described in Section\ref{sub:gt:semantic_annotation}.
%A SVG file begins with information about the XML version and encoding system and then a root \texttt{<svg>} element containing the title of the document \texttt{<title>} and four others \texttt{<svg>} elements with different class attributes. % that describe the page, the panels, the text lines and the text areas the entire set of annotations for the attached page.
%The annotations are the same for all the \texttt{<svg>} nodes, each of them describing one kind of element (e.g. panel, balloon) according to its {\tt class} attribute. 
%The first \texttt{<svg>} element, \texttt{<svg class=``Page''>}, has two children.
%The first one is \texttt{<image>} and contains a link to the corresponding image file and the size it has to be displayed.
%The next child is a \texttt{<metadata>} element containing the bibliographical information described in \ref{sub:gt:semantic_annotation}.
The four following \texttt{<svg>} siblings, \texttt{<svg class=``Panel''>}, \texttt{<svg class=``Balloon''>}, \texttt{<svg class=``Line''>} and \texttt{<svg class=``Character''>} respectively contain the annotations about panels, balloons, text lines and comic characters. 
They all contain SVG \texttt{<polygon>} elements with a list of five or more points in a \texttt{point} attribute that define the position of the bounding box corners or the pixel-level contour. %four corners position coordinates of a bounding box. 
Note that the last point is always equal the first one to ``close'' the polygon according to the SVG specifications. 
Those points are used by the viewer to draw polygons with the corresponding CSS style. 
Each \texttt{<polygon>} has a \texttt{<metadata>} child to store information on the corresponding polygon, according to the attributes list described Section~\ref{sub:gt:semantic_annotation}.
An example of ground truth file is given Appendix~\ref{app:groundtruth}.
% présente un exemple du contenu de l'un de ces fichiers.

% \begin{lstlisting}[language=XML, frame=single, caption=Example of SVG file from the eBDtheque ground truth, captionpos=b, label=list:svg]
% <?xml version="1.0" encoding="UTF-8" standalone="no"?>
% <svg>
%   <title>CYB_BUBBLEGOM_T01_005</title>
%   <svg class="Page">
%     <image 
%       x="0" 
%       y="0" 
%       width="750" 
%       height="1060" 
%       href="CYB_BUBBLEGOM_T01_005.jpg"
%     />
%     <metadata 
%       collectionTitle="Bubblegom_Gom"
%       editorName="Studio_Cyborga" 
%       doublePage="false"
%       website="http://bubblegom.over-blog.com"
%       albumTitle="La_Legende_des_Yaouanks"
%       drawerName="Cyborg_07"
%       language="french"
%       resolution="300"
%       ISBN="979-10-90655-01-0"
%       readingDirection="leftToRight" 
%       writerName="Cyborg_07"
%       releaseDate="2009"
%       pageNumber="5"
%     />
%   </svg>
%   <svg class="Panel">
%     <polygon points="53,95 268,95 268,292 53,292 53,95">
%       <metadata 
%         idPanel="P01"
%         rank="1"
%       />
%     </polygon>
%     ...
%   </svg>
%   <svg class="Balloon">
%     <polygon points="61,103 143,103 143,172 61,172 61,103">
%       <metadata
%         idBalloon="B01" 
%         shape="smooth"
%         tailTaip="153,167"
%         tailDirection="SE" 
%         rank="1"
%       />
%     </polygon>
%     ...
%   </svg>
%   <svg class="Line">
%     <polygon points="373,121 432,121 432,132 373,132 373,121">
%       <metadata 
%         idLine="L01"
%         idBalloon="B01"
%       >
%         LIKE YOU.
%       </metadata>
%     </polygon>
%     ...
%   </svg>
%   <svg class="Character">
%     <polygon points="84,153 261,153 261,298 84,298 84,153"/>
%       <metadata idCharacter="C01"/>
%     ...
%   </svg>
%   <svg class="LinkSBSC">
%     <polygon points="34,234 56,235 79,340 79,339 34,234"/>
%       <metadata 
%         idLinkSBSC="LSBSC01"
%         idBalloon="B01"
%         idCharacter="C01"
%       />
%     ...
%   </svg>
% </svg>
% \end{lstlisting}

%{\bf Add SVG code figure here?} Arnaud suggested it too but, we don't have room and it will be ugly...

%
%\begin{lstlisting}
%<svg>
%</svg>
%\end{lstlisting}

%{\color{red}short file as an illustration?}

% \begin{figure}
% \begin{center}
% \begin{tabular}{ccc}
% \includegraphics[width=0.14\textwidth]{fig/bbg_panel2.png} & 
% \includegraphics[width=0.14\textwidth]{fig/bbg_balloon2.png} &
% \includegraphics[width=0.14\textwidth]{fig/bbg_textlines2.png}
% %\includegraphics[width=0.20\textwidth]{fig/bbg_all.png} &
% \end{tabular}
% \caption{Left-to-right: segmentation of a panel, a speech balloon and text lines. Credits: \cite{Bubble09}}
% \label{fig:css}
% \end{center}
% \end{figure}
% subsection file_structure (end)

% \begin{itemize}
%   \item Comic Book Markup Language \url{http://digitalhumanities.org/dhq/vol/6/1/000117/000117.html}
% \end{itemize}

\subsection{Ground truth quality assessment}
\label{sec:gt:ground_truth_quality_assessment}

When several persons are involved in the creation of a graphical ground truth, it is very difficult to obtain a perfectly homogeneous segmentation.
Indeed, it could vary from one person to another because each person has a different sensitivity at reading comics and at integrating instructions.
Therefore, in addition to the package of pages he was in charge of, each participant has been asked to annotate the panels of a extra page. 
This extra page was the same for everybody and was chosen for its graphical components heterogeneity. 
It contained ten panels from which, four were full-framed, five half-framed and one was frame less.
%, see Fig. \ref{fig:test_page}. 
% This heterogeneity is somehow representative of the whole corpus.
We defined an acceptable error for the position of a corner given by several persons.
The images of dataset being of different definitions, using a percentage of the page size makes more sense than using a specific number of pixels.
%A percentage of the page width and height is more relevant.
%Let set an acceptable error of 0.5\% of the page for the position on each corner. 
We set this percentage $p$ at 0.5\% of the page height and width in $x$ and $y$. 
Given the definition of the test image of 750x1060 pixels, this makes a delta of +/- 5 pixels in $y$ axis and +/- 4 pixels in $x$ axis. 

We asked to each one of the twenty involved persons to draw the four points bounding box of the panels ignoring text area.
A mean position from the twenty different values has been calculated for each of them.
Then, the distance of each point to its mean value is computed. 
Figure~\ref{fig:gt:graphiqueStdVT} shows the amount of corners for a distance, centred on zero.


%%%%%%%%%%%%%%%%%%%%%%%%%%%%%%%%%%%%%%%%%%%%%ù
\begin{figure}[h!]
\begin{center}
\includegraphics[width=0.7\textwidth]{stdVT.png}
\caption[Distance to the mean position]{Number of corners for a given standard deviation value. This has been calculated on $y$ axis and $x$ axis and produces similar plot.}
\label{fig:gt:graphiqueStdVT}
\end{center}
\end{figure}
%%%%%%%%%%%%%%%%%%%%%%%%%%%%%%%%%%%%%%%%%%%%%ù

Given the threshold $th=0.5$, 87.5\% of pointed corners can be considered as being homogeneous over the group of labelling people. 
%{\bf what is p?} p is the threshold
The overall mean standard deviation on this page reaches 0,15\%
 (1.13 pixels) for the width, and 0.12\% (1.28 pixels) for the height.
The two bumps, at -40 and 15, are related to the missegmentation of 13 of the 80 panels. 
Indeed, instructions have been misunderstood by some people who included text area outside of the panels or missed some panel's parts.
Figure~\ref{fig:gt:diffVT} shows the difference between areas labelled as a panel by at least one person and areas labelled as a panel by every participant.
However, such mistakes have been manually corrected before publishing the ground truth.

%%%%%%%%%%%%%%%%%%%%%%%%%%%%%%%%%%%%%%%%%%%%%%%%%
\begin{figure}[h!]
\begin{center}
\includegraphics[width=0.7\textwidth]{segDifference.png}
\caption[Image used for ground truth quality assessment]{Image used for error measurement. Red hatched areas are the difference between areas labelled as panels by at least one person and areas labelled by everybody. Image credit:~\cite{Bubble09}.}
\label{fig:gt:diffVT}
\end{center}
\end{figure}
%%%%%%%%%%%%%%%%%%%%%%%%%%%%%%%%%%%%%%%%%%%%%%%%%

Even though the error criterion has only been estimated on panels, it is reasonable to extend it to balloons and text lines as well.
Indeed, the segmentation protocol being quite similar for all features (bounding box as close as possible to the drawing), the observed standard deviation of panel corner positions has no reason to be different from balloons and text lines.
The pixel-level balloon and the comic characters visual annotation have been carried out by a single person, the homogeneity is only subject to the regularity of the person over time and is, therefore, difficult to assess quantitatively.

\subsection{Discussions}
\label{ssec:gt:siscussions}

We presented the eBDtheque dataset, the first comic image dataset and associated ground truth on comic books containing spatial and semantic annotations, publicly available for the scientific community.
% The corpus has been introduced as well as the construction protocol.
The corpus can be extended to enlarge the diversity of the dataset and provide consecutive page or full album in order to allow a wider level of comic books analysis.
 New semantic annotations can be added, such as the view angle and the shot type for panels, text at the level of letters, comic character's role, profile and relationships. 


% section dataset_and_ground_truth_construction (end)

\section{Metrics} % (fold)
\label{sec:ex:metrics}
The contribution of this thesis are from different nature that needs to be evaluated separately using appropriate metrics.
Object localisation developments are evaluated using the commonly used recall and precision metrics and other contribution are evaluated using accuracy.

\subsection{Object localisation metric} % (fold)
\label{sub:ex:object_localisation_metric}


% paragraph metrics (end)
We evaluate the different extractions (panel, balloon and text regions) in terms of object bounding boxes such as the PASCAL VOC challenge~\cite{everingham2010pascal}.
The detections are assigned to ground truth objects and judged to be true or false positives by measuring bounding box overlap.
To be considered a correct detection, the overlap ratio $a_0$ between the predicted bounding box $B_p$ and the ground truth bounding box $B_{gt}$ (Formula~\ref{eq:ex:overlap_ratio}) must exceed 0.5.
The predicted objects are considered as true positive $TP$ if $a_0 > 0.5$ or false positive $FP$ (prediction errors).

\begin{equation}
\label{eq:ex:overlap_ratio}
  a_0 = \frac{area(B_p \cap B_{gt})}{area(B_p \cup B_{gt})}
\end{equation}

Detections returned by a method are assigned to ground truth objects satisfying the overlap criterion ranked by the confidence output (decreasing).
Multiple detections of the same object in an image are considered false detections (e.g. 5 detections of a single object counted as 1 correct detection and 4 false detections).

The number $TP$, $FP$ and false negative (missed elements) $FN$ was used to compute the recall $R$ and the precision $P$ of each the methods using Formula~\ref{eq:recall} and~\ref{eq:precision}.
We also compute the F-measure $F$ for each result.

\begin{equation}
\label{eq:recall}
  R = \frac{TP}{TP + FN}
\end{equation}
% where $FN$ is the number of false negative (missed elements).

\begin{equation}
\label{eq:precision}
  P = \frac{TP}{TP + FP}
\end{equation}
% where $FP$ is the number of false positives (prediction errors).


\subsection{Object segmentation metric} % (fold)
\label{sub:ex:object_segmentation_metric}

Object bounding box based evaluation is appropriate for surfaces comparison but not for detailed region extraction evaluation.
In section~\ref{sub:in:balloon_classification} we extract balloon contour at the level of pixel for analysis and classification purposes.
In order to make the difference between the errors sources we need to use a more precise evaluation framework.
Here we keep using recall and precision metrics as introduced in the previous section but instead of counting the number of object that valid a certain criterion, we simply count the number of pixel that have been correctly ($TP$), incorrectly ($FP$) or missed ($FN$).


\subsection{Text recognition metric} % (fold)
\label{sub:ex:text_recognition_metric}
Even if text recognition would require more investigation to be fully treated, we give a first baseline evaluation in Section~\ref{par:ex:text_recognition_evaluation} using commercial OCR systems on the eBDtheque dataset.
We evaluated the text detection accuracy $A_{textReco}$ at a given string edit distance between the predicted recognition and its corresponding transcription in the ground truth~\cite{Guerin2013}.

% subsection text_recognition (end)

% subsection object_localisation (end)
\subsection{Tail detection metric} % (fold)
\label{sub:ex:tail_detection_metric}
% We evaluated the proposed tail extraction method on the 1092 balloons of the eBDtheque dataset~\cite{Guerin2013} ``version 2014''.
% In the ground truth, some balloons do not have a tail but we did not remove them as the proposed method can also detect when there is no tail and return a correct answer.

Tail tip and tail direction are not surfaces, therefore they can not be evaluated using recall and precision metrics presented Section~\ref{sub:ex:object_localisation_metric}.
%+For this particular case, we redefine the recall and precision by measuring the Euclidean distance between the detected tip and its ground truth (recall) and the direction error (precision).
Thus, we define two accuracy metrics $A_{tailTip}$, the accuracy of the predicted position of the tail tip and $A_{tailDir}$ the accuracy of the tail direction prediction.
The Euclidean distance $d_0$ between the predicted position of the tip and its ground truth is measured relative to balloon size (Formula~\ref{eq:ex:accuracy_tailtip}).
Note that we consider incorrect the predicted positions at a distance $d_0$ superior to the balloon size ($A_{tailTip} < 0$). 

\begin{equation}
\label{eq:ex:accuracy_tailtip}
  A_{tailTip} = 1 - \frac{d_0}{0.5 * (B_{width} + B_{height})}
\end{equation}
where $B_{width}$ and $B_{height}$ correspond to the balloon width and height respectively.
 % as follow $A_{tailTip} = 1 - d_0/ (B_{width} + B_{height})/2 )$.

The direction accuracy $A_{tailDir}$ was measured according to the distance $d_1$ within the eight cardinal coordinate sequences defined in Section~\ref{tab:se:offset_panel_corner} (Formula~\ref{eq:ex:accuracy_taildir}).

\begin{equation}
\label{eq:ex:accuracy_taildir}
  A_{tailDir} = 1- \frac{d_1}{8}
\end{equation}

For instance if the detected direction was $S$ (south) and the ground truth was $SE$ (south-east) then $d_1=1$.
Note that our method can also detect when there is no tail on the balloon contour $C_{tail}=0\%$ (confidence equal to zero percent); in this case $A_{tailTip}=A_{tailDir}=100\%$ if there was effectively no tail to detect or $A_{tailTip}=A_{tailDir}=0\%$.
% For this experiment, the local window size $M$ of the tail direction process was set to 10\% of the mean balloon size (Equation~\ref{eq:se:mean_balloon_size}) in order to be invariant to the image definition \modif{TODO: justify}.

% subsection tail_detection (end)

\subsection{Semantic links metric} % (fold)
\label{sub:ex:semantic_links_metric}

The semantic links between speech text and speech balloon are called $STSB$ and the ones between speech balloon and speaking character $SBSC$; they characterise a dialogue.
They are considered true or false according to their existence or not in the ground truth.
We evaluated the semantic relations $STSB$ and $SBSC$ according to the metadata in the ground truth of the eBDtheque dataset~\cite{Guerin2013} called \texttt{{isLineOf}} and \texttt{{isSaidBy}}, which represent 3427 and 829 relations respectively.
We defined two accuracy metrics $A_{STSB}$ and $A_{SBSC}$ to measure the percentage of correctly predicted semantic links.

% subsection semantic_links_and_text_recognition (end)

% section metrics (end)

% \section{Parameter validation}

% \section{Evaluation} % (fold)
% \label{sec:ex:evaluation}

% We evaluate the contributions of the thesis at different levels.
% We first evaluate the sequential, independent and knowledge-driven approaches separately and then we compare them in the global evaluation section.

% \section{Sequential information extraction evaluation} % (fold) %IJDAR
% \label{sub:ex:sequential_information_extraction_evaluation}

% TODO

\section{Panel extraction evaluation} % (fold)
\label{sub:ex:panel_extraction_evaluation}
% We evaluated the proposed method on the 850 panels of the eBDtheque dataset~\cite{Guerin2013} ``version 2014'' at bounding box level.

In this section we evaluate our three approaches (Method A, B and C) and compare them to two methods from the literature.
We compare our results to Arai~\cite{Arai10} and Ho~\cite{Ho2012} that are two state of the art methods (Section~\ref{sec:sota:layout_panel}).
The first one use connected-component analysis similarly to our proposition and the second is based on growing region.

All the evaluation are performed on the 850 panels of the eBDtheque dataset (Section~\ref{sec:dataset_and_ground_truth_construction}) at object bounding box level, using the recall and precision metrics introduced Section~\ref{sub:ex:object_localisation_metric}.

% \subsection{Experimental settings} % (fold)
% \label{sub:experimental_settings}

% TODO

\subsection{Arai's method} % (fold)
\label{sub:ex:panel_extraction_arai}
We re implemented the comic panel extraction method presented in~\cite{Arai10} except the division line detection by lake of detail in the original paper.

This method consists in a bi-level segmentation with an empiric threshold value of 250 followed by a connected-component extraction, binary image inversion and blob selection.
The final blob selection is based on a minimal size of $Image.Width / 6$ and $Image.Height / 8$.
From the selected blobs, a line detection approach is applied as final decision.
This line detection approach is able to cut overlapped panel on a page width or height basis which is appropriate for pages with a single panel per strip as illustrated Figure~\ref{fig:ex:division_line_detection}.
The author did not share the code and were not able to re implement this part by lake of detail in the original paper.
Anyway, the line division method works only for panels that are as large as the page which is not so common and would not have affected the result significantly (Figure~\ref{fig:ex:division_line_detection}).

%%%%%%%%%%%%%%%%%%%%%%%%%%%%%%%%%%%%%%%%%%%%%
\begin{figure}[h!]
\begin{center}
\includegraphics[width=0.8\textwidth]{division_line_detection_arai10.png}
\caption[Division line detection for panel extraction]{Division line detection from~\cite{Arai10} (figure 5 in the author's paper).}
\label{fig:ex:division_line_detection}
\end{center}
\end{figure}
%%%%%%%%%%%%%%%%%%%%%%%%%%%%%%%%%%%%%%%%%%%%%

The score of this method was in average for recall and precision of 20.00\% and 18.75\% (Figure~\ref{fig:ex:panel_arai_extraction_detail}).

%%%%%%%%%%%%%%%%%%%%%%%%%%%%%%%%%%%%%%%%%%%%%%%%%%%%%%
% \begin{figure}
%  \includegraphics[width=\textwidth,height=4cm]{2014-07-20_svg_Arai2010_no_division_linePanel_object.pdf}
%  \caption{Panel extraction score details for each image of the eBDtheque dataset (Appendix~\ref{app:dataset}).
%  }
%  \label{fig:ex:panel_arai_extraction_detail}
% \end{figure}
%%%%%%%%%%%%%%%%%%%%%%%%%%%%%%%%%%%%%%%%%%%%%%%%%%%%%%


% subsection arai_cite_arai (end)

\subsection{Ho's method} % (fold)
We re implemented the Ho's method~\cite{Ho2012} with the original parameters and set the minimal and maximal lower brightness difference of the growing region method to 20 (not mentioned in the original paper).
The image border is filled with the average five-pixel page border colour and four seeds are initialized on the four corners of the image.
When the region growing algorithm stops, the background is removed which separated panel blocks.
Mathematical morphology (dilatation) is then applied until block become smaller than $1/6$ of the page size and then the same number of erosion is applied in order to give back to the objects their initial size.
This manipulation separates the connected panels but has the disadvantage of creating unwanted holes as well (Figure~\ref{fig:ex:panel_ho_extraction_process}).

%%%%%%%%%%%%%%%%%%%%%%%%%%%%%%%%%%%%%%%%%%%%%%%%%%%
\begin{figure}[!ht] %trim=l b r t  width=0.5\textwidth, 
  \centering
  %\includegraphics[height=60mm]{figure/BUBBLEGOM_T01_P007_crop.jpg}
  %\includegraphics[trim= 0mm 0mm 0mm 0mm]{figure/BUBBLEGOM_T01_P007.jpg}
  \subfloat[Original image]{\label{fig:ex:se:panel_ho_img}\includegraphics[trim= 0mm 0mm 0mm 0mm, clip, width=0.2\textwidth]{ho01.png}}
  \hspace{1em}
  \subfloat[Binary segmentation]{\label{fig:se:panel_ho_binary}\includegraphics[trim= 0mm 0mm 0mm 0mm, clip, width=0.2\textwidth]{ho02.png}}
  \hspace{1em}
  \subfloat[After dilatation step]{\label{fig:ex:panel_ho_dilate}\includegraphics[trim= 0mm 0mm 0mm 0mm, clip, width=0.2\textwidth]{ho03.png}}
  \subfloat[After erosion step]{\label{fig:ex:panel_ho_erode}\includegraphics[trim= 0mm 0mm 0mm 0mm, clip, width=0.2\textwidth]{ho04.png}}  
    \caption[Panel extraction process of Ho]{Panel extraction and separation process of Ho~\cite{Ho2012}.}
    \label{fig:ex:panel_ho_extraction_process}
\end{figure}
%%%%%%%%%%%%%%%%%%%%%%%%%%%%%%%%%%%%%%%%%%%%%%%%%%%

The score of this method was in average for recall and precision of 49.76\% and 68.74\% (Figure~\ref{fig:ex:panel_ho_extraction_detail}).

%%%%%%%%%%%%%%%%%%%%%%%%%%%%%%%%%%%%%%%%%%%%%%%%%%%%%%
% \begin{figure}
%  \includegraphics[width=\textwidth,height=4cm]{2014-07-20_svg_Ho2012Panel_object.pdf}
%  \caption{Panel extraction score details for each image of the eBDtheque dataset (Appendix~\ref{app:dataset}).
%  }
%  \label{fig:ex:panel_ho_extraction_detail}
% \end{figure}
%%%%%%%%%%%%%%%%%%%%%%%%%%%%%%%%%%%%%%%%%%%%%%%%%%%%%%

\subsection{Method A} % (fold)
\label{sub:ex:panel_extraction_rigaud_method_A}

The method presented Section~\ref{sec:se:panel_and_text} is parameter free except for the number of cluster for k-means clustering algorithm that we fixed to $k=3$ for the reason explained in the presentation of the method (Section~\ref{sec:se:panel_and_text}).
Note that the method also extract text region at the same time which does not interfere with panel (Section~\ref{sub:ex:text_extraction_recognition_evaluation}).
The score of this method  was in average for recall and precision of 64.24\% and 83.81\% (Figure~\ref{fig:ex:panel_methodA_extraction_detail}).

%%%%%%%%%%%%%%%%%%%%%%%%%%%%%%%%%%%%%%%%%%%%%%%%%%%%%%
% \begin{figure}[h]
%  \includegraphics[width=\textwidth,height=4cm]{Panel_object_proposed.pdf}
%  \caption{Panel extraction score details for each image of the eBDtheque dataset \modif{TODO: update}.
%  }
%  \label{fig:ex:panel_methodA_extraction_detail}
% \end{figure}
%%%%%%%%%%%%%%%%%%%%%%%%%%%%%%%%%%%%%%%%%%%%%%%%%%%%%%

\subsection{Method B} % (fold)
% \label{sub:method_b}
The method presented Section~\ref{sec:in:panel_extraction} requires only a minimal area factor.
Assuming that a panel is a big region, we ignored the panel detection with a area lower than 4\% ($minAreaFactor$) of the page area according to a validation on the eBDtheque dataset.
This parameter avoids considering small and isolated elements (e.g text, logo, page number) as panel.
The score of this method was in average for recall and precision of 62.94\% and 87.30\% (Figure~\ref{fig:ex:panel_methodB_extraction_detail}).

%%%%%%%%%%%%%%%%%%%%%%%%%%%%%%%%%%%%%%%%%%%%%%%%%%%%%%
% \begin{figure}[h]
%  \includegraphics[width=\textwidth,height=4cm]{Panel_object_proposed.pdf}
%  \caption{Panel extraction score details for each image of the eBDtheque dataset \modif{TODO: update}.
%  }
%  \label{fig:ex:panel_methodB_extraction_detail}
% \end{figure}
%%%%%%%%%%%%%%%%%%%%%%%%%%%%%%%%%%%%%%%%%%%%%%%%%%%%%%

% subsection method_b (end)

\subsection{Method C} % (fold)
% \label{sub:method_c}
The knowledge-driven method can be used as a post processing of the panel extraction.
It validates or rejects panel candidates using the proposed model (Section~\ref{sub:kn:validation}).
In the model, the only rule about the panel is that they should be contained in a image.
The score of this method was in average for recall and precision of \modif{??}\% and \modif{??}\% (Figure~\ref{fig:ex:panel_methodC_extraction_detail}).

% %%%%%%%%%%%%%%%%%%%%%%%%%%%%%%%%%%%%%%%%%%%%%%%%%%%%%%
% \begin{figure}[h]
%  \includegraphics[width=\textwidth,height=4cm]{Panel_object_proposed.pdf}
%  \caption{Panel extraction score details for each image of the eBDtheque dataset \modif{TODO: update}.
%  }
%  \label{fig:ex:panel_methodC_extraction_detail}
% \end{figure}
%%%%%%%%%%%%%%%%%%%%%%%%%%%%%%%%%%%%%%%%%%%%%%%%%%%%%%
% subsection method_c (end)


%%%%%%%%%%%%%%%%%%%%%%%%%%%%%%%%%%%%%%%%%%%%%%%%%%%
\begin{figure}[!ht] %trim=l b r t  width=0.5\textwidth, 
  \centering
  %\includegraphics[height=60mm]{figure/BUBBLEGOM_T01_P007_crop.jpg}
  %\includegraphics[trim= 0mm 0mm 0mm 0mm]{figure/BUBBLEGOM_T01_P007.jpg}
  \subfloat[Arai's method]{\label{fig:ex:panel_arai_extraction_detail}\includegraphics[trim= 0mm 108mm 0mm 0mm, clip, width=\textwidth]{2014-07-20_svg_Arai2010_no_division_linePanel_object.pdf}}
  %\hspace{1em}
  \\
  \subfloat[Ho's method]{\label{fig:ex:panel_ho_extraction_detail}\includegraphics[trim= 0mm 108mm 0mm 0mm, clip, width=\textwidth]{2014-07-20_svg_Ho2012Panel_object.pdf}}
  %\hspace{1em}
  \\
  \subfloat[Method A]{\label{fig:ex:panel_methodA_extraction_detail}\includegraphics[trim= 0mm 108mm 0mm 0mm, clip, width=\textwidth]{2014-07-24_svg_Rigaud2012LNCS_Method_APanel_object.pdf}}
  %\hspace{1em}
  \\
  % \subfloat[Method A]{\label{fig:in:panel_outermost_contours}\includegraphics[trim= 0mm 0mm 0mm 0mm, clip, width=\textwidth,height=4cm\textwidth]{2014-07-20_svg_Arai2010_no_division_linePanel_object.pdf}}
  % \\
  \subfloat[Method B]{\label{fig:ex:panel_methodB_extraction_detail}\includegraphics[trim= 0mm 108mm 0mm 0mm, clip, width=\textwidth]{2014-08-19_svg_Rigaud2014Outermost_Method_BPanel_object.pdf}}% \hspace{1em}
  \\
  % \subfloat[Method A]{\label{fig:in:panel_outermost_contours}\includegraphics[trim= 0mm 0mm 0mm 0mm, clip, width=\textwidth,height=4cm\textwidth]{2014-07-20_svg_Arai2010_no_division_linePanel_object.pdf}}
  % \\
  \subfloat[Method C \modif{TODO: update}]{\label{fig:ex:panel_methodC_extraction_detail}\includegraphics[trim= 0mm 0mm 0mm 0mm, clip, width=\textwidth]{2014-08-19_svg_Rigaud2014Outermost_Method_BPanel_object.pdf}}% \hspace{1em}
  % \\
  % \subfloat[Method C]{\label{fig:in:panel_big_contour_hull}\includegraphics[trim= 0mm 0mm 0mm 0mm, clip, width=\textwidth,height=4cm]{panel_big_contour_hull.png}}
    \caption[Panel extraction score details for each image of the eBDtheque dataset]{Panel extraction score details for each image of the eBDtheque dataset (Appendix~\ref{app:dataset}).}
    \label{fig:ex:panel_detection_result_details}
\end{figure}
%%%%%%%%%%%%%%%%%%%%%%%%%%%%%%%%%%%%%%%%%%%%%%%%%%%


%-----------------------------------------------------------
\subsection{Comparison and analysis} % (fold)
\label{sub:result_analysis}

Table~\ref{tab:panel_extraction_comparision_results} compare the average results we obtained for the three proposed methods and two methods from the literature.
Note that some of the dataset images were digitized with a dark background surrounding the cover of the book.
We automatically remove this by cropping the image where a panel with an area $>$ 90\% of the page area was detected.


    %%%%%%%%%%%%%%%%%%%%%%%%%%%%%%%%%%%%%%%%%%%%%%%%%%%
  \begin{table}[ht]
    \normalsize
%\renewcommand{\arraystretch}{1.2}

    \centering
    \caption[Panel extraction evaluation results]{Panel extraction evaluation results.}
    \begin{tabular}{|c|c|c|c|}
          % \hline
          %   & \multicolumn{2}{|c|}{Character 1}   & \multicolumn{2}{|c|}{Character 2}   \\
          \hline
          &  $R$ (\%)  & $P$ (\%)  & $F$ (\%)     \\
          \hline
          Arai~\cite{Arai11}   & 20.0       & 18.75    & 19.35    \\
          \hline
          Ho~\cite{Ho2012}   & 49.76       & 68.74    & 57.48    \\
          \hline
          Method A          & 64.24       & 83.81   & \bf{72.72}     \\
          \hline
          Method B          & 62.94       & 87.30   & \modif{?}      \\
          \hline
          Method C          & \modif{?}   & \modif{?} & \modif{?}      \\
          \hline
        \end{tabular}
    \label{tab:panel_extraction_comparision_results}
  \end{table}%
    %%%%%%%%%%%%%%%%%%%%%%%%%%%%%%%%%%%%%%%%%%%%%%%%%%%

%TODO: re evaluate Arai's result from his method (still waiting for is email 2014-07-01)

The main drawback of the Arai's method is the binary segmentation with a fixed value that works best for comic book with a white background but it is not appropriate for digitization defaults and paper yellowing.
This method is only appropriate for webcomics with a perfectly white background clear separation between panels (Figure~\ref{fig:ex:panel_arai_extraction_detail}).

Ho's method is adaptive to background variation but considers as panel all the elements that are directly included (first topological level) in the page background.
This produce an over-segmentation by including some text which are inside the frame-less panels or inside implicit balloons.
When those out of panel elements are small, they prevent the connected panel separation process to work properly because the stopping criterion is based on element size.

The first method we proposed (Method A) extracts panel and text simultaneously and then other elements in a sequential manner.
It assumes that there are the distinguishable cluster of connected-component height in the image which is not always the case in this dataset.
Anyway, it still more powerful that the two method from the literature, mainly because of the more generic pre-processing steps.

The second proposed panel extraction method (Method B), based on connected-component clustering, is simple to implement, fast and efficient for comics with disconnected panels (separated by a white gutter).
% Figure~\ref{ap:panel_extraction} shows the details for each image tested, which were mainly comics with gutters.
% The proposed method is not appropriate for gutterless comics such as ``INOUE'' or strip without panel border such as ``MONTAIGNE'' or a extra frame around several panels (``SAINTOGAN\_PROSPER'').
This method is not appropriate for gutterless comics (e.b. some mangas) or strip without panel borders such as those with an extra frame around several panels.
Another weakness is when panels are connected by other elements, they may not be split as desired but will remain clustered as panel anyway.

Method C benefits form the domain knowledge to filter out irrelevant panel candidates from the independent extraction approach (Method B).
Nevertheless, the validation by the expert system was not significant here because the low level processing had already reached the limits of the model (Section~\ref{sec:kn:knowledge_representation}).

% This experiment was performed in \modif{??} seconds for the whole dataset using one CPU at 2.5GHz (\modif{??}s per panel on average).

% subsection result_analysis (end)


% subsection sequential_panel_extraction_evaluation (end)


\section{Text extraction evaluation} % (fold)
\label{sub:ex:text_extraction_recognition_evaluation}

In this section we evaluated our three propositions for text extraction and compare them to \modif{??} methods from the literature.
We selected \modif{TODO: method names, comparison reasons}.
All the evaluation of text localisation was performed on the 4691 text lines of the eBDtheque dataset~\cite{Guerin2013} at object bounding box level (Section~\ref{sub:ex:object_localisation_metric}).
Text recognition evaluation have been performed on the same dataset but using the accuracy metric presented Section~\ref{sub:ex:text_recognition_metric}.

% \subsection{Experimental settings} % (fold)
% % \label{sub:experimental_settings}

% TODO

\subsection[Arai's method]{Arai's method} % (fold)
We implemented Arai's method~\cite{Arai11} which is a sequential approach that requires panel and balloon extraction as input,
presented in Section~\ref{sub:ex:panel_extraction_arai} and~\ref{sub:ex:balloon_extraction_arai} respectively.
This method was developed for grey-scale Japanese mangas and is divided in five steps: pre-processing, blob extraction, feature extraction, text blob selection and text blob extraction.
Pre-processing consists in an adaptive bi-level segmentation at 30\% (chosen empirically) above the average pixel value and mathematical morphology to group potential text letters into block (similar to the Run Length Smoothing Algorithm).
Note that the threshold selection method can exceed the maximal pixel value and then produce an empty segmentation.
The size of the kernel was not specified in the original paper, we used $3*5$ pixels.
Blob extraction is performed by imposing a minimal size of the blob relatively to the size of the balloon in which they are included ($Balloon.width / 20$ and $Balloon.hight / 40$).
The extracted feature are the \emph{average text blob width} and the horizontal coordinate of its centroid.
These features are used to classify text / non-text blobs according to two rules based on blob inter-distances and alignments.
Text blob are then extracted from the original image and directly sent to an OCR system (Figure~\ref{fig:ex:balloon_extraction_arai}).

%%%%%%%%%%%%%%%%%%%%%%%%%%%%%%%%%%%%%%%%%%%%%%%%%%%%%%
\begin{figure}[h]
 \centering
 \includegraphics[width=0.9\textwidth]{balloon_extraction_arai.png}
 \caption[Sample of text extraction process of Arai's method]{Sample of text extraction process. (a) Original balloon text image; (b) Threshold; (c) Morphology-Erosion filter; (d) Morphology-Opening filter; (e) Extracted blob from original image. Image from the original paper~\cite{Arai11}.
 }
 \label{fig:ex:balloon_extraction_arai}
\end{figure}
%%%%%%%%%%%%%%%%%%%%%%%%%%%%%%%%%%%%%%%%%%%%%%%%%%%%%%

This approach was only developed for vertical text block extraction but we adapt it to also handle horizontal text.
Its adaptation to horizontal text consist in switching width and height related parameters in the pre-processing and feature extraction steps.
Also, the kernel used for mathematical morphology is rotated of 90$^{\circ}$.

The score of this method was in average for recall and precision of 2.81\% and 1.63\% (Figure~\ref{fig:ex:panel_arai_extraction_detail}).

\subsection{Method A} % (fold)
% \label{sub:method_1}
The method proposed Section~\ref{sec:se:panel_and_text} does not require any particular parameter, only the number of cluster has to be fixed $k=3$ for the reason explained in the presentation of the method (Section~\ref{sec:se:panel_and_text}).
Note that the method is designed to simultaneously extract panel regions which does not interfere with text.
The score of this method  was in average for recall and precision of 54.68\% and 56.92\% (Figure~\ref{fig:ex:text_methodA_extraction_detail}).

% %%%%%%%%%%%%%%%%%%%%%%%%%%%%%%%%%%%%%%%%%%%%%%%%%%%%%%
% \begin{figure}[h]
%  \includegraphics[width=\textwidth,height=4cm]{Panel_object_proposed.pdf}
%  \caption{Text extraction score details using Method A for each image of the eBDtheque dataset \modif{TODO: update}.
%  }
%  \label{fig:ex:text_methodA_extraction_detail}
% \end{figure}
% %%%%%%%%%%%%%%%%%%%%%%%%%%%%%%%%%%%%%%%%%%%%%%%%%%%%%%


% subsection method_1 (end)
\subsection{Method B} % (fold)
\label{sub:ex:text_extraction_evaluation_method_b}

The method presented Section~\ref{sec:in:text_localisation_and_recognition} consist in a bi-level segmentation, text/graphic separation, text line generation and finally text line recognition.
A comparison between two threshold selection methods and two colour-to-grey image conversion are shown Table~\ref{fig:ex:text_extraction_methodB_segmentation}.

%%%%%%%%%%%%%%%%%%%%%%%%%%%%%%%%%%%%%%%%%%%%%%%%%%%

  \begin{table}[ht]
    % \normalsize
    \centering
    \caption[F-measure results for fixed and adaptive threshold selection method corresponding to combined and separated colour to grey conversion methods]{F-measure results for fixed ($th=250$) and adaptive threshold selection method corresponding to combined (RGB to grey) and separated (Luminance layer) colour to grey conversion methods.}
    \begin{tabular}{|c|c|c|}
      \hline
       & Fixed & Otsu \\ 
      \hline
      RGB to grey & 11.74 & \textbf{51.35}  \\ 
      \hline
      Luminance layer & 11.66 & 51.31 \\ 
      \hline
    \end{tabular}
    \label{fig:ex:text_extraction_methodB_segmentation}
  \end{table}%

  % \begin{figure}[!ht]%trim=l b r t  width=0.5\textwidth,
  % \begin{center}
  %   \begin{tabular}{|c|c|c|}
  %     \hline
  %      & Fixed & Otsu \\ 
  %     \hline
  %     RGB to grey & \modif{??} & \modif{??}  \\ 
  %     \hline
  %     Luminance layer & \modif{??} & \modif{??} \\ 
  %     \hline
  %   \end{tabular}
  % \caption[Results for fixed and adaptive threshold selection method corresponding to combined and separated colour to grey conversion methods]{Results for fixed ($th=250$) and adaptive threshold selection method corresponding to combined (RGB to grey) and separated (Luminance layer) colour to grey conversion methods.}
  % \label{fig:ex:text_extraction_methodB_segmentation}
  % \end{center}
  % \end{figure}  
  %%%%%%%%%%%%%%%%%%%%%%%%%%%%%%%%%%%%%%%%%%%%%%%%%%%

Text and graphics separation is performed using the Mahalanobis' distance which is related to the number of standard deviations.
A validation on the eBDtheque dataset shown that the best Mahalanobis' distance is $d_m=2$.
% The similarity offset $a$ and the overlapping percentage $b$ have been also validated over the eBDtheque dataset and their best values are $a=$\modif{???} and $b=$\modif{???} respectively.
Text line recognition is used as last filtering operation to validate the presence of text inside the candidate regions, its average benefit is shown Table~\ref{tab:text_results}.
The score of this method was in average for recall and precision of 64.14\% and 70.28\% (Figure~\ref{fig:ex:text_methodB_extraction_detail}).

%The detailed results using the OCR system are given Figure~\ref{fig:ex:text_methodB_extraction_detail}.


%%%%%%%%%%%%%%%%%%%%%%%%%%%%%%%%%%%%%%%%%%%%%%%%%%%%%%
% \begin{figure}[h]
%  \includegraphics[width=\textwidth,height=4cm]{Panel_object_proposed.pdf}
%  \caption{Text extraction score details using Method B and the best bi-level combination method (Otsu + Luminance) for each image of the eBDtheque dataset \modif{TODO: update}.
%  }
%  \label{fig:ex:text_methodB_extraction_detail}
% \end{figure}
%%%%%%%%%%%%%%%%%%%%%%%%%%%%%%%%%%%%%%%%%%%%%%%%%%%%%%


% subsection method_b (end)

\subsection{Method C} % (fold)
% \label{sub:method_c}

The knowledge-driven method can be used as a post processing of the text extraction.
It validates or rejects text candidates using the proposed model (Section~\ref{sub:kn:validation}).
In the model, the only rule concerning text location is that they should be contained in a panel.
Considering the best method for panel extraction method from Table~\ref{tab:panel_extraction_comparision_results}, the score of this method was in average for recall and precision of \modif{??}\% and \modif{??}\% (Figure~\ref{fig:ex:text_methodC_extraction_detail}).

%%%%%%%%%%%%%%%%%%%%%%%%%%%%%%%%%%%%%%%%%%%%%%%%%%%%%%
% \begin{figure}[h]
%  \includegraphics[width=\textwidth,height=4cm]{Panel_object_proposed.pdf}
%  \caption{Text extraction score details using Method C for each image of the eBDtheque dataset \modif{TODO: update}.
%  }
%  \label{fig:ex:text_methodC_extraction_detail}
% \end{figure}
%%%%%%%%%%%%%%%%%%%%%%%%%%%%%%%%%%%%%%%%%%%%%%%%%%%%%%

% subsection method_c (end)

% subsection experimental_settings (end)




\subsection{Comparison and analysis} % (fold)
\label{sub:results_analysis}

Table~\ref{tab:text_results} summarise the recall, precision and f-measure of the experimented methods.

    %%%%%%%%%%%%%%%%%%%%%%%%%%%%%%%%%%%%%%%%%%%%%%%%%%%
  \begin{table}[ht]
    \normalsize
    \centering
    \caption{Text localisation results.}
    \begin{tabular}{|c|c|c|c|}
          % \hline
          %   & \multicolumn{2}{|c|}{Character 1}   & \multicolumn{2}{|c|}{Character 2}   \\
          \hline
          &  $R$ (\%)  & $P$ (\%)  & $F$ (\%)  \\
          % \hline
          % Before validation   & ?   & ?           \\
          \hline
          Tanaka?\\ Chinese ICDAR? Stroke-based? Lluis?  & \modif{??}   & \modif{??}    & \modif{??}       \\
          \hline
          Arai~\cite{Arai11}  & 2.81    & 1.63      & 2.07       \\
          \hline
          Method A            & 54.91   & 57.15     & 56.01       \\
          \hline
          Method B (without OCR)  & \textbf{67.21}   & 41.54    & 51.35       \\
          \hline
          Method B (with OCR)  & 64.14   & \textbf{70.28}    & \textbf{67.07}       \\
          \hline
          Method C  & \modif{?}   & \modif{?}    & \modif{?}       \\
          \hline
          % Proposed (\cite{Rigaud2013VISAPP}+OCR)  & 60.13   & 42.43   & 49.75        \\
          % \hline
          % Proposed + validation   & 44.54     & 65.05     & 52.88      \\
          % \hline
          %Proposed (\cite{Rigaud2013VISAPP}+OCR+$ST$ only)  & ?   & ?   & ?        \\
          % \hline
          % Proposed (OCR transcription)  & ?   & ?           \\
          % \hline
        \end{tabular}
    \label{tab:text_results}
  \end{table}%
    %%%%%%%%%%%%%%%%%%%%%%%%%%%%%%%%%%%%%%%%%%%%%%%%%%%

% In our previous work~\cite{Rigaud2013VISAPP} text extraction was evaluated on a subset of 20 pages of the eBDtheque dataset~\cite{Guerin2013}.
% Here we applied it to the whole dataset.
% We used our previous method as a baseline to show an improvement in the precision of 20\% when using an OCR-based filter, without a significant loss in recall.
% The validation by the expert system improved the precision as expected but also resulted in a drop in recall.
% The drop in recall can be explained by the fact that the text extractor is also able to detect texts which are not in the speech balloons but the model considers them as noise.
% in recall that could probably be filled for specific writing style by training the OCR on a specific font. %, when using an OCR system to filter out non text regions.

    %%%%%%%%%%%%%%%%%%%%%%%%%%%%%%%%%%%%%%%%%%%%%%%%%%%
  % \begin{table}[ht]
  %   \normalsize
  %   \centering
  %   \caption{Text localisation results.}
  %   \begin{tabular}{|c|c|c|c|}
  %         % \hline
  %         %   & \multicolumn{2}{|c|}{Character 1}   & \multicolumn{2}{|c|}{Character 2}   \\
  %         \hline
  %         &  $R$ (\%)  & $P$ (\%)  & $F$ (\%)  \\
  %         % \hline
  %         % Before validation   & ?   & ?           \\
  %         \hline
  %         Rigaud~\cite{Rigaud2013VISAPP}  & 61.00   & 19.66    & 29.75       \\
  %         \hline
  %         Proposed (\cite{Rigaud2013VISAPP}+OCR)  & 60.13   & 42.43   & 49.75        \\
  %         \hline
  %         Proposed + validation   & 44.54     & 65.05     & 52.88      \\
  %         % \hline
  %         %Proposed (\cite{Rigaud2013VISAPP}+OCR+$ST$ only)  & ?   & ?   & ?        \\
  %         % \hline
  %         % Proposed (OCR transcription)  & ?   & ?           \\
  %         \hline
  %       \end{tabular}
  %   \label{tab:text_results}
  % \end{table}%
    %%%%%%%%%%%%%%%%%%%%%%%%%%%%%%%%%%%%%%%%%%%%%%%%%%%

Arai's method is a sequential approach that requires panel and balloon regions as input.
As shown in Table~\ref{tab:ex:balloon_localisation_comparison_results}, balloon extraction is very low using this approach thus, it narrows down the text extraction in turn (Table~\ref{tab:text_results}).
Moreover, this text extraction method uses several thresholds which reduces its fields of application.


% In our previous work~\cite{Rigaud2013VISAPP} text extraction was evaluated on a subset of 20 pages of the eBDtheque dataset~\cite{Guerin2013}.
% Here we applied it to the whole dataset.
Method A improves by far the recall thanks to its genericity, moreover, even if the panel extraction is performed at the same time, it does not bias the text extraction because text areas are extracted from the whole page and not from panel regions.
Method B is slightly better using RGB to grey image conversion than the luminance layer only (Table~\ref{fig:ex:text_extraction_methodB_segmentation}).
It outperforms Method A in recall only when not using the last OCR filtering step and both recall and precision when using the OCR.
The difference for Method B when using or not the OCR system is about an increase of 26.74\% of precision and a loss 3.07\% of recall.
The validation by the expert system in Method C improved the precision as expected but also resulted in a drop in recall.
The drop in recall of Method C can be explained by the fact that the text extractor is also able to detect texts which are not in the speech balloons but the model considers them as noise.
All the methods still encounter difficulties with certain types of text that can be found in the comics (e.g. illustrative text, graphic sounds) due to strong deformation or colour variations.


%%%%%%%%%%%%%%%%%%%%%%%%%%%%%%%%%%%%%%%%%%%%%%%%%%%
\begin{figure}[!ht] %trim=l b r t  width=0.5\textwidth, 
  \centering
  \subfloat[Arai's method]{\label{fig:ex:panel_arai_extraction_detail}\includegraphics[trim= 0mm 108mm 0mm 0mm, clip, width=\textwidth]{2014-07-26_svg_Arai2011IJIPLine_object.pdf}}
  \\
  % \subfloat[Ho's method]{\label{fig:ex:panel_ho_extraction_detail}\includegraphics[trim= 0mm 108mm 0mm 0mm, clip, width=\textwidth]{2014-07-20_svg_Ho2012Panel_object.pdf}}
  % \\
  \subfloat[Method A]{\label{fig:ex:text_methodA_extraction_detail}\includegraphics[trim= 0mm 108mm 0mm 0mm, clip, width=\textwidth]{2014-07-24_svg_Rigaud2012LNCS_letter2lineRigaud2014IJDARLine_object.pdf}}
  \\
  \subfloat[Method B using the best bi-level combination method (rgb2grey + Otsu) and OCR]{\label{fig:ex:text_methodB_extraction_detail}\includegraphics[trim= 0mm 108mm 0mm 0mm, clip, width=\textwidth]{2014-08-19_svg_RigaudVISAPPThesis2014_rgb2gray_otsu_ocr_Line.pdf}}
  \\
  \subfloat[Method C \modif{TODO: update}]{\label{fig:ex:text_methodC_extraction_detail}\includegraphics[trim= 0mm 0mm 0mm 0mm, clip, width=\textwidth]{2014-08-19_svg_Rigaud2014Outermost_Method_BPanel_object.pdf}}

  \caption[Text line extraction score details for each image of the eBDtheque dataset]{Text line extraction score details for each image of the eBDtheque dataset (Appendix~\ref{app:dataset}).}
    \label{fig:ex:text_detection_result_details}
\end{figure}
%%%%%%%%%%%%%%%%%%%%%%%%%%%%%%%%%%%%%%%%%%%%%%%%%%%

\section{Text recognition evaluation} % (fold)
\label{par:ex:text_recognition_evaluation}
% \paragraph{Text recognition evaluation} % (fold)
% \label{par:ex:text_recognition_evaluation}

We evaluated text transcription using string edit distance between a predicted text transcription given by the OCR and its corresponding transcription in the ground truth (Section~\ref{sub:ex:text_recognition_metric}).
The eBDtheque dataset is composed of English, Japanese and French texts.
We evaluated the transcription given by the OCR engine Tesseract version 3.02.02 with the provided training data for each language\footnote{\url{https://code.google.com/p/tesseract-ocr/downloads/list}}.
% corresponding to the language defined in the ground truth of each image of the dataset.
This was performed at the text line level, taking as correct the text lines that were transcribed exactly as the ground truth transcription, considering all the letters as lower case and ignoring accents (for predicted and ground truth regions).
% with a edit distance $<$ 10\% of their length using equation~\ref{eq:recall} and~\ref{eq:precision}.
%For instance a recognised text line of ten elements will be considered as correct if less than three are split, deleted, transposed, replaced or inserted. 
Table~\ref{tab:ex:text_recogniton_results} shows text recognition results given the text line position from the ground truth (best reachable score) and from the best automatic text localisation method (\modif{Method B} in Figure~\ref{tab:text_results}).

    %%%%%%%%%%%%%%%%%%%%%%%%%%%%%%%%%%%%%%%%%%%%%%%%%%%
  \begin{table}[ht]
    % \normalsize
    \centering
    \caption{Text recognition accuracy $A_{textReco}$ results from automatic text line localisation at an edit distance ($ED$) of 0, 1 and 2.}
    \begin{tabular}{|c|c|c|c|}
          \hline
          Text extraction method &  $ED=0$  & $ED=1$  & $ED=2$  \\
          \hline
          % Ground truth  & \modif{??}   & \modif{??} & \modif{??}       \\
          % \hline
          Method B  &  11.11  & 16.53     & 20.67       \\
          \hline
        \end{tabular}
    \label{tab:ex:text_recogniton_results}
  \end{table}%
    %%%%%%%%%%%%%%%%%%%%%%%%%%%%%%%%%%%%%%%%%%%%%%%%%%%

We obtained a score of $A_{textReco}=11.11\%$ for perfect transcription from the automatic text extraction and OCR method which constitute a baseline for future work on text recognition on the eBDtheque dataset~\cite{Guerin2013}.
Table~\ref{tab:ex:text_recogniton_results} shows the results for a more relaxed evaluation where we also considered as correct the text lines at a text edit distance equal to one and two, the accuracy rise up to 20.67\%.
%The distribution of the text line lengths is given in Figure~\ref{fig:textline_lenth_distribution}.
Note that the average text line length is quite short in comic books compared to other documents, the distribution is given Figure~\ref{fig:ex:textline_lenth_distribution}.

% subsection results_analysis (end)



\section{Balloon extraction evaluation} % (fold)
\label{sub:ex:balloon_extraction_evaluation}
In this section we evaluate our three balloon extraction approaches (Method A, B and C) and compare them to two methods from the literature.
We compare our results to Arai~\cite{Arai10} and Ho~\cite{Ho2012} that are two state of the art methods (Section~\ref{sec:sota:layout_panel}).
The first one use connected-component analysis similarly to our proposition and the second is based on growing region.
The performed the evaluation on the 1091 balloons of the eBDtheque dataset~\cite{Guerin2013} at object and pixel levels in order to provide a performance evaluation for localisation and contour analysis purposes (sections~\ref{sub:ex:object_localisation_metric} and~\ref{sub:ex:object_segmentation_metric}).
%also measure the impact on the balloon classification based on balloon's contour analysis.
% We evaluate the three approaches on the 1092 balloons of the eBDtheque dataset~\cite{Guerin2013} at object bounding box and pixel levels, and compare to previous work from the literature.
% We also evaluate the tail detection for each balloon extraction approaches

In the eBDtheque ground truth, 84.5\% of the balloons are closed and 15.5\% are not (implicit).
Thus we do not expect to reach 100\% recall and precision for the regular balloon extraction methods because they are not designed for implicit balloons.
One the other hand, implicit the balloon extraction method is able to extract closed balloon as well.

% \subsection{Experimental settings} % (fold)
% % \label{sub:experimental_settings}

% TODO

\subsection{Arai's method} % (fold)
\label{sub:ex:balloon_extraction_arai}

Arai's method in a sequential method that extracts balloon inside panel regions in order to extract text inside~\cite{Arai11}.
It it similar to its panel extraction work-flow also presented in the same paper (Section~\ref{sub:ex:panel_extraction_arai}), without image inversion which basically detect white blobs (assuming speech balloon have a white background).
Not all the white blobs correspond to balloon in the comic book images, especially for the monochrome ones.
They select blob candidate according to four rules based on blob size, white pixel occurrence, vertical straight line detection and width to length ratio.
We re-implemented all the rules and heuristics of the original paper except for the straight line detection that we rotated of 90 degrees for images using horizontal text.

The score of this method  was in average for recall and precision of 13.40\% and 11.76\% (Figure~\ref{fig:ex:balloon_arai_extraction_detail}).

% %%%%%%%%%%%%%%%%%%%%%%%%%%%%%%%%%%%%%%%%%%%%%%%%%%%%%%
% \begin{figure}[h]
%  \includegraphics[width=\textwidth,height=4cm]{Panel_object_proposed.pdf}
%  \caption{Balloon extraction score details of Arai's method for each image of the eBDtheque dataset \modif{TODO: update}.
%  }
%  \label{fig:ex:balloon_arai_extraction_detail}
% \end{figure}
% %%%%%%%%%%%%%%%%%%%%%%%%%%%%%%%%%%%%%%%%%%%%%%%%%%%%%%

% subsection arai_s_method_cite (end)

\subsection{Ho's method~\cite{Ho2012}} % (fold)
% \label{sub:arai_s_method_cite}

Ho's method is a sequential approach, similarly to Arai's approach, that extracts balloon inside panel regions in order to extract text inside~\cite{Ho2012}.
First it converts the image from RBG to HSV colour representation and select candidate regions with a high value (V) and low saturation (S) level.
To reduce the number of candidate regions, a second selection is applied according to size and shape.
Small regions are removed and only region with a ratio between the number of pixel in the region over the number of pixel in its bounding higher than are kept.
In the original paper, only the ratio between the region area and its bounding box was given (60\%).
We removed the value (V) from the saturation (S) in order to reduce the number of parameters and fixed it at 200 (region with a higher brightness are kept).
We considered as small regions the region with a width or height inferior to three pixels.


The score of this method  was in average for recall and precision of 13.96\% and 24.73\% (Figure~\ref{fig:ex:balloon_ho_extraction_detail}).

%%%%%%%%%%%%%%%%%%%%%%%%%%%%%%%%%%%%%%%%%%%%%%%%%%%%%%
% \begin{figure}[h]
%  \includegraphics[width=\textwidth,height=4cm]{Panel_object_proposed.pdf}
%  \caption{Balloon extraction score details of Ho's method for each image of the eBDtheque dataset \modif{TODO: update}.
%  }
%  \label{fig:ex:balloon_ho_extraction_detail}
% \end{figure}
%%%%%%%%%%%%%%%%%%%%%%%%%%%%%%%%%%%%%%%%%%%%%%%%%%%%%%

% TODO: describe the method here? Ho's method is a sequential approach that search for panel first and then balloon inside panels and text inside balloons.
% The parameter used in the experiment are not given, I propose mine...

% subsection ho_s_method_cite (end)

\subsection{Method A} % (fold)
% \label{sub:method_a}

In sections~\ref{sub:se:regular_balloon_extraction} and~\ref{sub:se:implicit_balloon_extraction} we presented two sequential approaches for regular and implicit balloon extraction from text line positions.
Here, we evaluated them separately and detail the benefits of energies used for implicit balloon extraction by active contour model.
For both regular and implicit balloon extraction, the input text is given from the sequential text extraction method (Section~\ref{sec:se:panel_and_text}) as it is the previous step of this approach.

\paragraph{Regular balloons} % (fold)
\label{par:regular_balloons}

Regular balloon extraction method is based on a white blob selection according to the central position of text lines in the balloon.
This central position is converted as confidence value $C_{balloon}$ for each balloon candidate.
The evolution of the performance of recall and precision according to this confidence rejection criterion is plotted Figure~\ref{fig:ex:validation_confidence_balloon_methodA_regular}.
The best score of $R=37.25\%$ and $P=45.19\%$ was obtained for \modif{$C_{balloon} > 10\%$} (Figure~\ref{fig:ex:balloon_methodA_extraction_detail}).


%%%%%%%%%%%%%%%%%%%%%%%%%%%%%%%%%%%%%%%%%%%%%%%%%%%%%%
\begin{figure}[h]
  \centering
 \includegraphics[width=0.7\textwidth]{figure_here.png}
 \caption{Validation of the regular balloon confidence value for Method A.
 }
 \label{fig:ex:validation_confidence_balloon_methodA_regular}
\end{figure}
%%%%%%%%%%%%%%%%%%%%%%%%%%%%%%%%%%%%%%%%%%%%%%%%%%%%%%


%%%%%%%%%%%%%%%%%%%%%%%%%%%%%%%%%%%%%%%%%%%%%%%%%%%%%%
% \begin{figure}[h]
%  \includegraphics[width=\textwidth,height=4cm]{Panel_object_proposed.pdf}
%  \caption{Balloon extraction score details of Method A for each image of the eBDtheque dataset \modif{TODO: update}.
%  }
%  \label{fig:ex:balloon_methodA_extraction_detail}
% \end{figure}
%%%%%%%%%%%%%%%%%%%%%%%%%%%%%%%%%%%%%%%%%%%%%%%%%%%%%%

%We consider as correctly detected the balloon with a confidence value higher than \modif{10\%} (Table~\ref{tab:ex:balloon_localisation_comparison_results}, Method A (1)).

% paragraph regular_balloons (end)


\paragraph{Implicit balloons} % (fold)
\label{par:ex:implicit_balloons_reminder}

The implicit balloon extraction method was evaluated using three different scenarios in order to highlight the benefits of both active contour theory and domain specific knowledge.
% that considers as balloon any white connected components that overlaps at more than $10\%$ with text regions.
First, the performance using active contour model with only the internal energy, then with the distance transform based external energy described section~\ref{sec:se:external_energie} and third, we added the domain knowledge energy $E_{text}$ from Section~\ref{sec:se:text_energie} (Table~\ref{tab:ex:implicit_balloon_performance_object_pixel_comparison}).
% Fourth, after a two stage contour fitting (4).
% Results are presented in table~\ref{tab:ex:balloon_localisation_comparison_results}.


%%%%%%%%%%%%%%%%%%%%%%%%%%%%%%%%%%%%%%%%%%%%%%%%%%
\begin{table}[h]
  \normalsize
% \renewcommand{\arraystretch}{1.3}
% \extrarowheight as needed to properly center the text within the cells
  \centering
  \caption{Implicit balloon performance evaluation at object and pixel levels using the original form and the proposed (with $E_{text}$) energy functions.}
  \begin{tabular}{|c|c|c|c|c|c|c|}
  \hline
    & \multicolumn{3}{|c|}{Object level}  & \multicolumn{3}{|c|}{Pixel level}   \\
  \hline
  Energy function  &  $ R$ (\%)  & $P$ (\%)& $F$ (\%)   &  $R$ (\%)  & $P$ (\%)   & $F$ (\%)\\
  \hline
%   Ho~\cite{Ho2012}  & ???     & ???   & & ???     & ???&    \\

  %$E = E_{int}$     & \modif{???}       & \modif{???}     & \modif{???}       & \modif{???}  & \modif{???} & \modif{???}   \\
  %\hline
   $E = E_{int} + E_{ext}$    & 56.01       & 40.40     & 46.94       & 69.40  & \textbf{83.98} & 76.00   \\
  \hline
  $E = E_{int} + E_{ext} + E_{text}$   & \textbf{57.76} & \textbf{41.62} & \textbf{48.38} & \textbf{74.80} & 82.67 & \textbf{78.55}    \\
  \hline
  \end{tabular}
      \label{tab:ex:implicit_balloon_performance_object_pixel_comparison}
\end{table}%
    %%%%%%%%%%%%%%%%%%%%%%%%%%%%%%%%%%%%%%%%%%%%%%%%%%%

% paragraph implicit_balloons (end)

%, the recall $R=??\%$, precision $P=?\%$ and f-measure $F=?\%$.

% subsection method_a (end)
\subsection{Method B} % (fold)
% \label{sub:method_b}

We introduced Method B in Section~\ref{sub:in:balloon_segmentation}, it is a independent approach that does not require any previous other element extraction.
This method requires one parameter which is the minimum number of children $minNbChildren$ of a balloon (number of included connected-components).
We set $minNbChildren=8$ according to the parameter validation, just before the first peak in the distribution of the number of letters per speech balloon (Figure~\ref{fig:ex:min_number_children_validation}).
Note that in Figure~\ref{fig:ex:min_number_children_validation}, there are about 3.5\% of the balloons bellow the selected threshold that contain one or two letters, usually punctuation marks.
%such as ``?'' or ``!''. 
We voluntary omitted them here to avoid detecting a lot of non balloon regions.

    %%%%%%%%%%%%%%%%%%%%%%%%%%%%%%%%%%%%%%%%%%%%%%%%%%%%%%%%
    \begin{figure}[h]%trim=l b r t  width=0.5\textwidth,  
      \centering
      \includegraphics[trim= 10px 0px 60px 0px, clip, width=0.75\textwidth]{number_of_letter_per_balloon.pdf}
      \caption[Distribution of the number of letter per speech balloon]{Distribution of the number of letter per speech balloon in the eBDtheque dataset~\cite{Guerin2013}.
      }
      \label{fig:ex:min_number_children_validation}
    \end{figure}  
    %%%%%%%%%%%%%%%%%%%%%%%%%%%%%%%%%%%%%%%%%%%%%%%%%%%%%%%%

% Balloons with a confidence value $C_{balloon}$ lower than 10\% were rejected according to the validation experiments on the eBDtheque dataset~\cite{Guerin2013}.

This regular balloon extraction method attributes a confidence value $C_{balloon}$ to each balloon candidate according to the alignment of the connected-components it contains (children), similarly to regular balloon extraction of Method A.
The evolution of the performance of recall and precision according to this confidence rejection criterion is plotted Figure~\ref{fig:ex:validation_confidence_balloon_methodB_regular}.
The best score of $R=52.68\%$ and $P=44.17\%$ was obtained for \modif{$C_{balloon} > 10\%$}, details are given Figure~\ref{fig:ex:balloon_methodB_extraction_detail}.


%%%%%%%%%%%%%%%%%%%%%%%%%%%%%%%%%%%%%%%%%%%%%%%%%%%%%%
\begin{figure}[h]
  \centering
 \includegraphics[width=0.7\textwidth]{figure_here.png}
 \caption{Validation of the regular balloon confidence value for Method B.
 }
 \label{fig:ex:validation_confidence_balloon_methodB_regular}
\end{figure}
%%%%%%%%%%%%%%%%%%%%%%%%%%%%%%%%%%%%%%%%%%%%%%%%%%%%%%


%%%%%%%%%%%%%%%%%%%%%%%%%%%%%%%%%%%%%%%%%%%%%%%%%%%%%%
% \begin{figure}[h]
%  \includegraphics[width=\textwidth,height=4cm]{Panel_object_proposed.pdf}
%  \caption{Balloon extraction score details of Method B for each image of the eBDtheque dataset \modif{TODO: update}.
%  }
%  \label{fig:ex:balloon_methodB_extraction_detail}
% \end{figure}
%%%%%%%%%%%%%%%%%%%%%%%%%%%%%%%%%%%%%%%%%%%%%%%%%%%%%%

Figure~\ref{fig:ex:balloon_methodB_extraction_detail} confirms that our method works best when the balloons are closed, well segmented and with non cursive text inside.

% subsection method_b (end)
\subsection{Method C} % (fold)
% \label{sub:method_c}

The knowledge-driven method can be used as a post processing of the balloon extraction.
It validates or rejects balloon candidates according to contextual information from the proposed model (Section~\ref{sub:kn:validation}).
In the model, the rules concerning balloon are that they should be contained in a panel and not contain other balloon, panel or comic character.
Considering the best method for balloon extraction (\modif{???}, the score of this method was in average for recall and precision of \modif{??}\% and \modif{??}\% (Figure~\ref{fig:ex:balloon_methodC_extraction_detail}).

%%%%%%%%%%%%%%%%%%%%%%%%%%%%%%%%%%%%%%%%%%%%%%%%%%%%%%
% \begin{figure}[h]
%  \includegraphics[width=\textwidth,height=4cm]{Panel_object_proposed.pdf}
%  \caption{Balloon extraction score details using Method C for each image of the eBDtheque dataset \modif{TODO: update}.
%  }
%  \label{fig:ex:balloon_methodC_extraction_detail}
% \end{figure}
%%%%%%%%%%%%%%%%%%%%%%%%%%%%%%%%%%%%%%%%%%%%%%%%%%%%%%

% subsection method_c (end)
% subsection experimental_settings (end)

\subsection{Balloon classification} % (fold)
\label{sub:balloon_classification}
We evaluated balloon classification using the pixel-level balloon segmentation from the eBDtheque ground truth and from the best automatic segmentation of regular balloon from Method B.
The ground truth was composed of only know balloon types, ignoring the class ``other'' which produce a subset of 940 balloons from the 1091 balloons present in the dataset.
The repartition of the balloon types is 87.96\%, 8.75\% and 3.29\% of type ``smooth'', ``wavy'' and ``spiky'' respectively.
This unbalanced repartition reflects the usual speech balloon types repartition in comics.

For the shape/contour separation part (Section~\ref{sec:contour_detection}) we first cast the original signal $o$ into 360 values and then smooth it to approximate the shape signal $s$.
The shape signal is also composed by 360 values where each of them corresponds to the local mean in a window of size $M$ from the original signal (Equation~\ref{eq:ex:shape_approximation}).
Note that in order to avoid side effects we copy $M/2$ values from the opposite side when needed (wrapping).
A validation of parameter $M$ have been performed and its best value was $M=7$.

\begin{equation}
  \label{eq:ex:shape_approximation}
  s(i) = 1/M \sum_{j=i-M/2}^{i+M/2} o(i)
\end{equation}

 %smoothing process divides the 360 values into $M$ segments and compute the mean of size $360/M$ according to 
 % A validation of parameter $M$ have been performed and its best value was $M=7$.
 % (Table~\ref{tab:validation_parameter_m}).
%In other words, we split the balloon contour into 20 sectors of 18 degrees each.
%which correspond to 5.5\% of 360 values 
% This parameter can be adjusted according the contour descriptor type.

  %%%%%%%%%%%%%%%%%%%%%%%%%%%%%%%%%%%%%%%%%%%%%%%%%%%
%   \begin{table}[ht]
%     \normalsize
% \renewcommand{\arraystretch}{1.2}

%     \centering
%     \caption{Validation of parameter $M$ for $D=(f_1,f_2)$ scenario \modif{TODO: update with new training set?}.}
%     \begin{tabular}{c|c|c|c|c|c|}
%   \cline{2-6}
%     & \multicolumn{5}{|c|}{Accuracy per class (\%)} \\
%   \cline{2-6}
%     & Smooth & Wavy & spiky & Avg & Global \\
%   \hline
%   \multicolumn{1}{ |c| }{$M=10$}& 80.63   & \textbf{70.58} & 58.80 & 70.00 & 75.8 \\
%   \hline
%   \multicolumn{1}{ |c| }{$M=20$}& \textbf{89.30}  & 69.1  & 86.20 & \textbf{81.53} & \textbf{85.2} \\
%   \hline
%   \multicolumn{1}{ |c| }{$M=30$}& 88.53 & 67.64 & \textbf{86.25}  & 80.81 & 84.4 \\
%   %\hline
%   \hline
%   \multicolumn{1}{ |c| }{$M=40$}& 87.35 & 61.76 & 82.35 & 77.15 & 81.99 \\
%   %\hline
%   \hline
%   \multicolumn{1}{ |c| }{$M=50$}& 79.05 & 63.23 & 80.39 & 74.22 & 76.34 \\
%   %\hline
%   \hline
% %       Proposed multi scale & ???  &???  & ???   & ???       \\
% %   \hline
%         \end{tabular}
%     \label{tab:validation_parameter_m}
%   \end{table}%
  %%%%%%%%%%%%%%%%%%%%%%%%%%%%%%%%%%%%%%%%%%%%%%%%%%%


We evaluated our method based on a one-versus-all scenario, classification results are presented in Table~\ref{tab:ex:desc_test_balloon_classification} for different descriptors.
First we show the effect of the two features $f_1$ and $f_2$ separately with without a priori information $P(l_1)=P(l_2)=P(l_3)$ and then using the prior probabilities about the classes repartition from the eBDtheque dataset $P(l_1)=0.88, P(l_2)= 0.09$ and $P(l_3)=0.03$.
We detailed our best results for balloon contour extraction (for $D=(f_1, f_2)$) in the confusion matrix Table~\ref{tab:ex:confusion_matrix_balloon_classification}.
Note that we also show the results when we automatically remove the tail region from its automatic detection (Section~\ref{sec:se:from_balloon_to_tail}).
Tail removal is performed by ignoring the points of the contour that are between the two points defined as tail origin (Section~\ref{sec:se:from_balloon_to_tail}), see Figure~\ref{fig:ex:tail_removal}.
Note that this process relies on the tail detection performance (Section~\ref{sec:tail_detection_evaluation}).

%   %%%%%%%%%%%%%%%%%%%%%%%%%%%%%%%%%%%%%%%%%%%%%%%%%%%
   \begin{figure}[!ht]  %trim=l b r t  width=0.5\textwidth,
     \centering
    \includegraphics[trim= 0px 0px 0px 0px, clip, width=0.75\textwidth]{tail_removal.png}
    \caption[Tail removal for balloon classification]{Two examples of tail removal for balloon classification. From left to right, original contour detection in red, corresponding mask and partial contour removal between the two green points that represent the tail origin. The final contour is represented in purple in the right figures.}
    \label{fig:ex:tail_removal}
   \end{figure}
%   %%%%%%%%%%%%%%%%%%%%%%%%%%%%%%%%%%%%%%%%%%%%%%%%%%%


    %%%%%%%%%%%%%%%%%%%%%%%%%%%%%%%%%%%%%%%%%%%%%%%%%%%
  \begin{table}[ht]
    \normalsize
    \renewcommand{\arraystretch}{1.2}

    \centering
    \caption[Classification result accuracy for different descriptor configuration]{Classification result accuracy for different descriptor configuration. The global accuracy represents the number of good classification divided by the number of element to classify independently from the class.
}
    \begin{tabular}{c|c|c|c|c|c|}
  \cline{2-6}
    & \multicolumn{5}{|c|}{Accuracy per class (\%)} \\
  \cline{2-6}
    & Smooth & Wavy & spiky & Average & Global \\
  \hline
  \multicolumn{1}{ |c| }{ $D=(f_1)$ } & 22.51 & 68.75 & 33.33  & 41.53 & 26.91  \\
  \hline
  \multicolumn{1}{ |c| }{ $D=(f_2)$ } & 79.73 & 36.25 & 70.00 & 61.99 & 75.60 \\
  \hline
  \multicolumn{1}{ |c| }{$D=(f_1, f_2)$ } & 76.62 & \textbf{60.00} & \textbf{73.34} & \textbf{64.33} & 75.05\\
  %\hline
  \hline
  \multicolumn{1}{ |c| }{$D=(f_1, f_2)$ + $priors$}   & \textbf{98.26} & 11.25 & 63.34 & 57.62 & \textbf{89.50}  \\
  %\hline
  \hline
%       Proposed multi scale & ???  &???  & ???   & ???       \\
%   \hline
        \end{tabular}
    \label{tab:ex:desc_test_balloon_classification}
  \end{table}%
    %%%%%%%%%%%%%%%%%%%%%%%%%%%%%%%%%%%%%%%%%%%%%%%%%%%


    %%%%%%%%%%%%%%%%%%%%%%%%%%%%%%%%%%%%%%%%%%%%%%%%%%%
  \begin{table}[ht]
    \normalsize
    \renewcommand{\arraystretch}{1.2}

    \centering
    \caption{Confusion matrix for smooth (S), wavy (W) and spiky (Z) balloon contour classification results with $D=(f_1, f_2)$.}
    \begin{tabular}{cc|c|c|c|c|c|c|}
  \cline{3-8}
   & & \multicolumn{3}{|c|}{Ground truth}  & \multicolumn{3}{|c|}{Ground truth no tail}  \\
  \cline{3-8}
   & & S & W  & Z & S & W  & Z  \\
  \hline
  \multicolumn{1}{ |c|  }{\multirow{3}{*}{Prediction} } & S & \textbf{616} & 180 & 8 & 114 & 349 & 341   \\
  \cline{2-8}
  \multicolumn{1}{ |c  }{} & \multicolumn{1}{ |c| }{ W } & 30  & 48 & 2 & 7 & \textbf{53} & 20  \\
  \cline{2-8}
  \multicolumn{1}{ |c  }{} & \multicolumn{1}{ |c| }{ Z } & 4 & 4 & \textbf{22} & 4 & 13 & 13  \\
  % \hline
  % \multicolumn{2}{ |c|  }{TOTAL}  & 804 & 80 & 30  & 804 & 80 & 30  \\
  %\hline
  \hline
%       Proposed multi scale & ???  &???  & ???   & ???       \\
%   \hline
        \end{tabular}
    \label{tab:ex:confusion_matrix_balloon_classification}
  \end{table}%
    %%%%%%%%%%%%%%%%%%%%%%%%%%%%%%%%%%%%%%%%%%%%%%%%%%%


% An example of the Gaussian distribution parameters of the training set presented Figure~\ref{fig:ex:balloon_training_set} is given Table~\ref{tab:ex:coefficient_balloon_classification}.
% Those parameters can be used for other dataset containing similar images without needing to train the classifier again.


%   %%%%%%%%%%%%%%%%%%%%%%%%%%%%%%%%%%%%%%%%%%%%%%%%%%%
%    \begin{figure}[!ht]  %trim=l b r t  width=0.5\textwidth,
%      \centering
%     \includegraphics[trim= 0px 0px 0px 0px, clip, width=0.7\textwidth]{figure_here.png}
%     \caption[Training set for balloon contour classification]{Training set for balloon contour classification.}
%     \label{fig:ex:balloon_training_set}
%    \end{figure}
%   %%%%%%%%%%%%%%%%%%%%%%%%%%%%%%%%%%%%%%%%%%%%%%%%%%%



  %%%%%%%%%%%%%%%%%%%%%%%%%%%%%%%%%%%%%%%%%%%%%%%%%%%
%   \begin{table}[ht]
%     % \normalsize
% % \renewcommand{\arraystretch}{1.2}

%     \centering
%     \caption{Example of Gaussian distribution parameters.}
%     \begin{tabular}{|c|c|c|c|c|c|c|}
%   \hline
%     & \multicolumn{2}{|c|}{Smooth}& \multicolumn{2}{|c|}{Wavy}& \multicolumn{2}{|c|}{spiky}  \\
%   \hline
%     & $\mu$ & $\sigma$ & $\mu$ & $\sigma$ & $\mu$ & $\sigma$    \\
%   \hline
%     Feature $f_1$ & 0.019   & 0.012 & 0.042 & 0.020 & 0.148 & 0.118   \\
%   \hline
%     Feature $f_2$ & -0.004  & 0.006 & 0.007 & 0.010 & -0.017 & 0.023   \\
%   \hline
%    %\multicolumn{1}{ |c| }{ Feature 2 } &  8  & 5 & 15 & 6 & 8 & 4 \\
%   %\cline{2-5}
%   %\multicolumn{1}{ |c  }{} & \multicolumn{1}{ |c| }{\textbf{ spiky }}   & ? & ? & 12 \\
%   %\hline
%   %\hline
% %       Proposed multi scale & ???  &???  & ???   & ???       \\
% %   \hline
%         \end{tabular}
%     \label{tab:ex:coefficient_balloon_classification}
%   \end{table}%
  %%%%%%%%%%%%%%%%%%%%%%%%%%%%%%%%%%%%%%%%%%%%%%%%%%%


% \modif{TODO: Ces valeurs sont pour quels types de dessins (cf. Figure4.11). 
% ESt ce que cette méthode peut prendre en compte cette diversité de représentation.... Cette remarque est un peu provocatrice, mais on pourrait te poser la question.. 
% Puisque tu commences la section en montrant les différents styles et la complexité et tu enchaines sur une méthode... on peut donc s'attendre à une conclusion sur ce que ta méthode est capable d'apporter par rapport à toutes ces représentations.}


The different accuracy results Table~\ref{tab:ex:desc_test_balloon_classification} show that the second feature $f_2$ is more discriminant than the first one $f_1$ and their combination does not exceed the second itself.
Adding the prior knowledge about the testing classes repartition improves the ``smooth'' class only and decrease the two others.
The confusion matrix Table~\ref{tab:ex:confusion_matrix_balloon_classification} shows good classification results for the ``spiky'' class even for cases that are quite different in terms of drawing styles (Figure~\ref{fig:good_detection_balloon_classification}) and sources (real and synthetic).
Concerning the ``smooth'' and ``wavy'' classes, some speech balloons are very hard to classify out of context causing confusion (Figure~\ref{fig:bad_detection_balloon_classification}).
Also, the tail of the balloon creates some noise and the radius average ($\bar{o}$) we use to normalize the descriptor features may not be appropriate for elongated balloons.


  %%%%%%%%%%%%%%%%%%%%%%%%%%%%%%%%%%%%%%%%%%%%%%%%%%
  \begin{figure}[h]  %trim=l b r t  width=0.5\textwidth,
    \centering
    \includegraphics[trim = 0mm 0mm 0mm 0mm, clip, width=300px]{good_detection.png}
    \caption{Correct classification examples for ``spiky'' (top) and ``smooth'' classes (bottom).}
    \label{fig:good_detection_balloon_classification}
  \end{figure}
  %%%%%%%%%%%%%%%%%%%%%%%%%%%%%%%%%%%%%%%%%%%%%%%%%%


  %%%%%%%%%%%%%%%%%%%%%%%%%%%%%%%%%%%%%%%%%%%%%%%%%%
  \begin{figure}[h]  %trim=l b r t  width=0.5\textwidth,
    \centering
    \includegraphics[trim = 0mm 0mm 0mm 0mm, clip, width=300px]{bad_detection.png}
    \includegraphics[trim = 0mm 0mm 0mm 0mm, clip, width=300px]{wavy_instead_smooth.png}
    \caption{Wrong classification examples which have been classified as ``smooth'' instead of  ``wavy'' in the first row and ``wavy'' instead of  ``smooth'' in the second row.}
    \label{fig:bad_detection_balloon_classification}
  \end{figure}
  %%%%%%%%%%%%%%%%%%%%%%%%%%%%%%%%%%%%%%%%%%%%%%%%%%


% subsection balloon_classification (end)




\subsection{Comparison and analysis} % (fold)
% \label{sub:result_analysis}


In this section, we discuss balloon localisation at bounding box and pixel levels, and also balloon classification.


% subsection result_analysis (end)

% \paragraph{Regular balloons} % (fold)
% \label{par:regular_balloons}

% paragraph regular_balloon (end)
% The second method for regular balloon extraction presented Section~\ref{sub:in:balloon_segmentation}
% The recall and precision scores are given Figure~\ref{fig:ex:regular_balloon_extraction} according to the confidence value $C_{balloon}$.


  %%%%%%%%%%%%%%%%%%%%%%%%%%%%%%%%%%%%%%%%%%%%%%%%%%%
  % \begin{figure}[h!]  %trim=l b r t  width=0.5\textwidth,
  %   \centering
  %   \includegraphics[trim= 0px 0px 0px 0px, clip, width=0.75\textwidth]{figure_here.png}
  %   \caption[Regular balloon extraction results]{Regular balloon extraction results.}
  %   \label{fig:ex:regular_balloon_extraction}
  % \end{figure}
  %%%%%%%%%%%%%%%%%%%%%%%%%%%%%%%%%%%%%%%%%%%%%%%%%%%



% \paragraph{Implicit balloons} % (fold)
% \label{par:implicit_balloons}

% We evaluated our different contributions separately.
% First we measured balloon localization performance at bounding
% box level to highlight the benefits of both active contour theory
% and domain specific knowledge.
% Second, we performed pixel level evaluation on a smaller subset to show the ability of our method to fit balloon contour details.

% In order to provide a comparative analysis we attempted comparison to the methods of Arai~\cite{Arai11} and Ho~\cite{Ho2012}, which are based on connected component detection and filtering. Unfortunately, direct comparison to these methods is not possible as Arai's approach~\cite{Arai11} is based on two rules specifically designed for Japanese manga comics with vertical text and Ho~\cite{Ho2012} is based on growing region segmentation which is not appropriate for open balloon detection neither.
% We also compared to the original active contour implementation proposed by Kass et al.~\cite{Kass1988} but because the initialization is not close enough to the edges, the internal energies make the snake retracts on itself.

% We compare our results using four different scenarios.
% First with the regular balloon extraction method presented just above (1).
% % that considers as balloon any white connected components that overlaps at more than $10\%$ with text regions.
% Second, the active contour with the distance transform based external energy described section~\ref{sec:external_energie} (2) and third, we add the domain knowledge energy $E_{text}$ (3) from Section~\ref{sec:text_energie}.
% Fourth, after a two stage contour fitting (4).
% Results are presented in table~\ref{tab:ex:balloon_localisation_comparison_results}.
% For each method, we present two variants, one making use of ground truth localization for the text areas as seeds for balloon localization, and another making use the results of the automatic text localization method presented Section~\ref{sec:se:panel_and_text}.

\paragraph{Balloon localisation} % (fold)
\label{par:balloon_localisation}
Balloon localisation aims to evaluate how the presented methods are good at predicting where the balloons are in the image.
Object localisation are performed at the level of object bounding box (Section~\ref{sub:ex:object_localisation_metric}).
Table~\ref{tab:ex:balloon_localisation_comparison_results} compare the average results we obtained for the best performance of the three proposed methods and two methods from the literature.

% paragraph balloon_localisation (end)
% Table~\ref{tab:text_results} summarises the recall, precision and f-measure of the experimented methods.

    %%%%%%%%%%%%%%%%%%%%%%%%%%%%%%%%%%%%%%%%%%%%%%%%%%%
  % \begin{table}[ht]
  %   % \normalsize
  %   \centering
  %   \caption{Balloon localisation result summary.}
  %   \begin{tabular}{|c|c|c|c|}
  %         % \hline
  %         %   & \multicolumn{2}{|c|}{Character 1}   & \multicolumn{2}{|c|}{Character 2}   \\
  %         \hline
  %         &  $R$ (\%)  & $P$ (\%)  & $F$ (\%)  \\
  %         \hline
  %         Arai~\cite{Arai11}& 13.40     & 11.76       & 12.53       \\
  %         \hline
  %         Ho~\cite{Ho2012}  & 13.96     & 24.76       & 17.84       \\
  %         \hline
  %         Method A (regular)& 37.25     & \textbf{45.19}       & 40.83       \\
  %         \hline
  %         Method A (implicit)& ?        & ?           & ?       \\
  %         \hline
  %         Method B          & \textbf{52.68}     & 44.17       & \textbf{48.05}       \\
  %         \hline
  %         Method C          & \modif{?}   & \modif{?}    & \modif{?}       \\
  %         \hline
  %         % Proposed (\cite{Rigaud2013VISAPP}+OCR)  & 60.13   & 42.43   & 49.75        \\
  %         % \hline
  %         % Proposed + validation   & 44.54     & 65.05     & 52.88      \\
  %         % \hline
  %         %Proposed (\cite{Rigaud2013VISAPP}+OCR+$ST$ only)  & ?   & ?   & ?        \\
  %         % \hline
  %         % Proposed (OCR transcription)  & ?   & ?           \\
  %         % \hline
  %       \end{tabular}
  %   \label{tab:ex:balloon_localisation_comparison_results}
  % \end{table}
    %%%%%%%%%%%%%%%%%%%%%%%%%%%%%%%%%%%%%%%%%%%%%%%%%%%


%%%%%%%%%%%%%%%%%%%%%%%%%%%%%%%%%%%%%%%%%%%%%%%%%%
\begin{table}[ht]
  % \normalsize
% \renewcommand{\arraystretch}{1.3}
% \extrarowheight as needed to properly center the text within the cells
  \centering
  \caption{Balloon localisation result summary.}
  \begin{tabular}{|c|c|c|c|c|c|c|}
  \hline
    & \multicolumn{3}{|c|}{Object level}  & \multicolumn{3}{|c|}{Pixel level}   \\
  \hline
  Method  &  $ R$ (\%)  & $P$ (\%)& $F$ (\%)   &  $R$ (\%)  & $P$ (\%)   & $F$ (\%)\\
  \hline
  Arai~\cite{Arai11}& 13.40     & 11.76       & 12.53 & 18.70 & 23.14 & 20.69       \\
  \hline
  Ho~\cite{Ho2012}  & 13.96     & 24.76       & 17.84 & 14.78 & 32.37 & 20.30     \\
  \hline
  Method A (regular)& 37.25     & \textbf{45.19}       & 40.83 & 47.75 & 44.58 & 46.11     \\
  \hline
  Method A (implicit)& \textbf{57.76} & 41.62 & \textbf{48.38} & \textbf{74.80} & \textbf{82.67} & \textbf{78.55}      \\
  \hline
  Method B (regular)          & 52.68     & 44.17       & 48.05 & 69.81 & 32.83 & 44.66      \\
  \hline
  Method C (regular)          & \modif{?}   & \modif{?}    & \modif{?} & & &      \\
  \hline
  \end{tabular}
      \label{tab:ex:balloon_localisation_comparison_results}
\end{table}%
    %%%%%%%%%%%%%%%%%%%%%%%%%%%%%%%%%%%%%%%%%%%%%%%%%%%

%%%%%%%%%%%%%%%%%%%%%%%%%%%%%%%%%%%%%%%%%%%%%%%%%%
% \begin{table}[ht]
%   \normalsize
% % \renewcommand{\arraystretch}{1.3}
% % \extrarowheight as needed to properly center the text within the cells
%   \centering
%   \caption[Balloon localization recall and precision results]{Balloon localization recall and precision results.}
%   \begin{tabular}{|c|c|c|c|}
%   % \hline

%     % & \multicolumn{3}{|c|}{Text from GT}  & \multicolumn{3}{|c|}{Text from method 2}   \\
%   \hline
%   Method  &  $ R$ (\%)  & $P$ (\%)& $F$ (\%)  \\
%   \hline
  
%   Arai~\cite{Arai11}  & 13.21       & 11.61           & 12.36          \\
  
%   \hline
%   Ho~\cite{Ho2012}    & 13.95       & 24.73           & 17.84 \\  
%   % \hline
%   % Method A (1)    & 37.24     & 45.19   & 40.83    \\

%   % \hline
%   % Method A (2)    & \modif{82.1}  & \modif{53.7}    & \modif{64.9}  \\
%   \hline
%   Method A        & \modif{83.4}    & \modif{55.5}      & \modif{66.6}  \\
%   % \hline
%   % Method A (4)   & \modif{?}      & \modif{?}      & \modif{?}  \\
%   \hline
%   Method B        & 52.66           & 44.17             & 48.05    \\
%   \hline
%   Method C        & \modif{??}      & \modif{??}        & \modif{??}  \\
%   \hline
%   \end{tabular}
%       \label{tab:ex:balloon_localisation_comparison_results}
% \end{table}%
    %%%%%%%%%%%%%%%%%%%%%%%%%%%%%%%%%%%%%%%%%%%%%%%%%%%

Our methods outperforms~\cite{Arai11} thanks to its genericity, since it can process all the image styles of the eBDtheque dataset.
This was expected as~\cite{Arai11} was specifically developed for manga comics that have certain stylistic particularities.

Method A (regular) detects less than half of the balloons of this dataset (about $37.25\%$ recall).
This approach is not originally designed for detecting implicit balloons but knowing the position of the panels allows closing open panels by drawing their bounding box.
The bounding box creates an artificial frame that sometimes recover implicit balloons extraction (Figure~\ref{fig:ex:implicit_balloon_recovery_by_panel}).

  %%%%%%%%%%%%%%%%%%%%%%%%%%%%%%%%%%%%%%%%%%%%%%%%%%%
  \begin{figure}[h]  %trim=l b r t  width=0.5\textwidth,
    \centering
    \includegraphics[trim= 0px 0px 0px 0px, clip, width=0.75\textwidth]{implicit_balloon_recovery.png}
    \caption[Implicit balloon recovering by closing open panels]{Implicit balloon recovering by closing open panels. From left to right: original panel, panel bounding box detection, panel with artificial frame from its bounding box that closes two balloons. The implicit balloons are now regular (closed).}
    \label{fig:ex:implicit_balloon_recovery_by_panel}
  \end{figure}
  %%%%%%%%%%%%%%%%%%%%%%%%%%%%%%%%%%%%%%%%%%%%%%%%%%%

However, it has two advantages in comparison with the proposed active contour based method.
First, as the connected-component is considered as balloon, when correct, the precision at the pixel level is maximal.
Second, it is faster to compute.

The presented implicit balloon extraction version of method A highly depends on the active contour initialization success.
In this study, we relied on text detection as we assume it is the most common feature that balloons include, while past experiments have shown that its accurate detection is feasible and stable.
A side-effect of this choice is that balloon sometimes can not be detected because their contain non text information (e.g. drawings, punctuation).
% the text line detector is not always used was not able to detect balloons that contain other contents (e.g. drawings, punctuation).
%and its bounding box level may also include non-textual information (e.g. portion of the balloon contour) that alter our method. % see fig.~\ref{fig:case_of_failure}. Moreover, : text line bounding boxes removal may also remove some part of the drawing before the external energy edge detection
We believe there is room for improvement of the different energy terms we used.
For example, one could use the Gradient Vector Flow proposed by Xu~\cite{Xu1998} for the external energy, especially in the case of missing data balloon boundaries.
%We could also initialize the snake with a more balloon-like shape as the minimal ellipse including the text area for example. This would also reduce computation time.
On the other hand, the ground truth of implicit balloons is at best questionable as the exact localization of the balloon is quite subjective.
An way to circumvent this problem could be to either define the boundary in a flexible way.
%, or match the ground truth at the pixel level.
% All the materials for reproducing and comparing the results presented in this paper are publicly available through \url{http://ebdtheque.univ-lr.fr/references}.
Our results using active contour with distance transform shows a significant improvement compared to the regular balloon extraction, thanks to the active contour theory that detect much more balloons whether open or closed than connected component based methods (Table~\ref{tab:ex:balloon_localisation_comparison_results}).


%Making the edges attract the active contour from further is an essential step as we can see line (3). We gain up to {\bf X RECALL AND Y PRECISION}.
%Mono-resolution and multi-resolution shows similar recall result because the high resolution affects only the shape details, not significantly changing the location of the detection. Next section explains how to evaluate the benefits of the high-resolution detection. %However there is a slight improvement for the precision measure due to higher precision in the shape detection. {\bf ADD RESULT FIGURE?}
% Doing just the low resolution localization step, or continuing to include the high resolution fitting step does not cause any difference for Method A (3) under this evaluation scheme, as the evaluation is performed at the level of bounding boxes object overlapping.


% \modif{TODO: OTHER VERSION?}

% Our methods outperforms~\cite{Arai11} thanks to its genericity, since it can process all the image styles of the eBDtheque dataset.
% This was expected as~\cite{Arai11} was specifically developed for manga comics that have certain stylistic particularities.
% We also surpassed our previous method~\cite{rigaud2013active} because it needs text lines as input which were given in our proposed text extraction method (Section~\ref{sub:text_extraction}).
Sequential approaches such as Arai, Ho and Method A suffer from limitations of dependency between the different processing.
For instance, in Method A the performance of our text extractor was 56.01\% (Table~\ref{tab:text_results}) which was used as input for balloon extraction so the balloon extraction was inevitably affected.
Also, such approaches search for balloon inside the panel region, as a consequence, these methods can not extract balloons that overlap several panels.
This is easily handled by independent method such as Method B because it does not consider panel positions and search for balloons everywhere in the image.
The post validation by the expert system improved the precision but decreased the recall of the extraction while improving the overall f-measure by almost 3\% (Method C).
The drop in recall was due to the balloons that were correctly extracted but which contained undetected text regions.
%Nevertheless, the proposed method does not beat our previous method~\cite{rigaud2013active} that is also able to extract non closed balloons from text regions (from the ground truth here).
% This experiment was performed in 22 minutes for the whole dataset using one CPU at 2.5GHz (2.2s per balloon on average).



% \paragraph{Balloon segmentation (pixel-level)}
% \label{par:}
% Balloon localisation is enough for object detection purposes at bounding box level.


% \modif{TODO}

% Balloon segmentation is performed at the level of pixel and can reflects such level of details.
% To evaluate the segmentation level, we repeated the evaluation using a pixel level evaluation scheme.
% The results are shown in table~\ref{tab:high_res}.
% Note that for this experiment we selected three pages where ground truth and automatic text detection give the same results.

%\subsection{Balloon shape detection}
%The bounding box level ground truth we used for balloon localization evaluation is not precise enough to know if all the detail of the balloon shape have been correctly detected or not. Here we evaluate the balloon detection from a pixel level ground truth (including the tail) on few pages\footnote{Page 1 refer to ``CYB\_BUBBLEGOM\_T01\_008'', page 2 to ``6673465787\_668ec4eff4\_o'' and page 3 to ``LAMISSEB\_ETPISTAF\_013''.{\bf should I add complete ref. in ref. section instead?}} from the same dataset as section~\ref{sec:dataset} and the evaluation measures of section~\ref{sec:eval}, in order to compare the performance between mono and multi-scale detection with more accuracy (see table~\ref{tab:high_res}). Note, for this experiment we selected only pages where ground truth and automatic text detection give same results.


% Table~\ref{tab:ex:high_res} shows...



% higher score for the two-stage variant for the page 1 and 3.
% These two pages contain mainly closed balloons, we see here that the second stage improves the accuracy of the detection of closed balloons.
% In the case of implied balloon boundaries, as in page 2, the second stage is not resulting in any improvement as there is no extra local information that can assist in the fitting.

% In this case the results mainly depends  of the low resolution detection. Note, processing time was about 10 seconds for a 300DPI A4 image on a regular machine.


  %%%%%%%%%%%%%%%%%%%%%%%%%%%%%%%%%%%%%%%%%%%%%%%%%%%
%   \begin{figure}[!ht] %trim=l b r t  width=0.5\textwidth,
%     \centering
%     \includegraphics[width=170px]{fig/mono_multi_detection.png}
%     \caption{Examples of closed balloon detection. The pixel level ground truth in red, the mono-resolution detection in grey and the multi-resolution in green. {\bf ADD OPEN BALLOON EXAMPLE}}
%     \label{fig:mono_multi_detection}
%   \end{figure}
  %%%%%%%%%%%%%%%%%%%%%%%%%%%%%%%%%%%%%%%%%%%%%%%%%%%

% \subsection{Result comparison} % (fold)
% \label{sub:result_comparison}
% Here we compare the best result we obtain for each of the contribution to methods from the literature at localisation and segmentation level.
% Balloon localisation is enough for object detection purposes at bounding box level.
% As mention Section~\ref{??}, the balloon contour contains part of information that is also important to retrieve in the context of understanding comic book documents.
% Balloon segmentation is performed at the level of pixel and can reflects such level of details.


%%%%%%%%%%%%%%%%%%%%%%%%%%%%%%%%%%%%%%%%%%%%%%%%%%%
\begin{figure}[!ht] %trim=l b r t  width=0.5\textwidth, 
  \centering
  \subfloat[Arai's method]{\label{fig:ex:balloon_arai_extraction_detail}\includegraphics[trim= 0mm 108mm 0mm 0mm, clip, width=\textwidth]{2014-07-21_svg_Arai2011_Balloon.pdf}}
  \\
  \subfloat[Ho's method]{\label{fig:ex:balloon_ho_extraction_detail}\includegraphics[trim= 0mm 108mm 0mm 0mm, clip, width=\textwidth]{2014-07-22_svg_Ho2012_Balloon.pdf}}
  \\
  \subfloat[Method A (regular)]{\label{fig:ex:balloon_methodA_extraction_detail}\includegraphics[trim= 0mm 108mm 0mm 0mm, clip, width=\textwidth]{2014-07-26_svg_closed_thesis_method_A_Balloon.pdf}}
  \\
  \subfloat[Method A (implicit)]{\label{fig:ex:balloon_methodA_extraction_detail}\includegraphics[trim= 0mm 108mm 0mm 0mm, clip, width=\textwidth]{2014-07-22_svg_implicit_ICDAR_Balloon_object_level.pdf}}
  \\
  \subfloat[Method B]{\label{fig:ex:balloon_methodB_extraction_detail}\includegraphics[trim= 0mm 108mm 0mm 0mm, clip, width=\textwidth]{2014-07-21_svg_closed_IJDAR_method_B_Balloon.pdf}}
  \\
  \subfloat[Method C \modif{TODO: update}]{\label{fig:ex:balloon_methodC_extraction_detail}\includegraphics[trim= 0mm 0mm 0mm 0mm, clip, width=\textwidth]{2014-08-19_svg_Rigaud2014Outermost_Method_BPanel_object.pdf}}

  \caption[Balloon extraction score details at object level for each image of the eBDtheque dataset]{Balloon extraction score details at object level for each image of the eBDtheque dataset (Appendix~\ref{app:dataset}).}
    \label{fig:ex:balloon_detection_result_details}
\end{figure}
%%%%%%%%%%%%%%%%%%%%%%%%%%%%%%%%%%%%%%%%%%%%%%%%%%%

\paragraph{Balloon classification analysis} % (fold)
\label{par:balloon_classification_analysis}

As mentioned Section~\ref{sub:in:balloon_classification}, the balloon contour contains part of information that is also important to retrieve in the context of understanding comic book documents (e.g. tail, tone information).
In this context, the location of the balloon is not enough sufficient.
 % in our case because later we will also consider tail detection based on balloon contour analysis.
Such analysis requires a finer extraction of the balloon contour at the level of pixel (Table~\ref{tab:ex:balloon_localisation_comparison_results}).

The proposed method covers the comics balloon classification in general accept for open balloons.
However, in this particular case, we can probably use extra features (e.g. tail type recognition, language analysis) to get more information about the text tone.
Also, other distortion measures have to be investigated, especially for ``wavy'' contours that are the most difficult contour to classify using the proposed method.
%In this study, the proposed descriptor we proposed is not sensitive enough to properly classify the ``wavy'' nor the definition of what is a ``wavy'' contour as to be revised.
Concerning the tail region, its removal definitely improves the accuracy of the classification process (Table~\ref{tab:ex:confusion_matrix_balloon_classification}).

% paragraph balloon_classification_analysis (end)

% subsection result_comparison (end)



\section{Tail detection evaluation} % (fold)
\label{sec:tail_detection_evaluation}

% \subsection{Experimental settings} % (fold)
% \label{sub:experimental_settings}

In this section we evaluate the tail extraction method on the 1091 balloons of the eBDtheque dataset~\cite{Guerin2013} from three balloon extraction methods, one manual and two automatic extractions.
The automatic extraction methods are the two best methods for regular (Method B) and implicit (Method A implicit) from the summary Table~\ref{tab:ex:balloon_localisation_comparison_results}.
The manual method is the balloon contours from the eBDtheque dataset.
The comparison of the three balloon extraction method will highlight the part of error which comes from the balloon extraction and the proper tail extraction method's error when there is not possible error from the balloon extraction.
As introduced in Section~\ref{sub:ex:tail_detection_metric}, we use two accuracy metrics, one for the predicted position of the tail tip $A_{tailTip}$ and the other one for the tail direction $A_{tailDir}$
 (Table~\ref{tab:ex:tail_evaluation}).

%and from the regular and implicit balloon extraction methods presented Section~\ref{sec:se:from_text_to_balloon} as their are the previous steps in the sequential approach (Table~\ref{tab:ex:tail_evaluation}).


% \subsection{Result analysis} % (fold)
% \label{sub:result_analysis}

    %%%%%%%%%%%%%%%%%%%%%%%%%%%%%%%%%%%%%%%%%%%%%%%%%%%
  \begin{table}[h]
    % \normalsize
%\renewcommand{\arraystretch}{1.2}

    \centering
    \caption{Tail tip position ($A_{tailTip}$) and tail direction ($A_{tailDir}$) extraction results from automatic and manual balloon contour extractions.}
    \begin{tabular}{|c|c|c|}
          % \hline
          %   & \multicolumn{2}{|c|}{Character 1}   & \multicolumn{2}{|c|}{Character 2}   \\
          \hline
          Balloon extraction method&  $A_{tailTip}$ (\%)  & $A_{tailDir}$ (\%)    \\
          \hline
          Manual        & \textbf{96.61}    & \textbf{80.40}     \\
          \hline
          Automatic (regular balloons)    & 55.77    & 27.59     \\
          \hline
          Automatic (regular + implicit balloons)    & 64.34    & 20.21     \\
          \hline
          % Proposed + validation   & \modif{??}       & \modif{??}    & \modif{??}    \\
          % \hline
           % TOTAL   & 93.4    & 92.8          \\
          % \hline
        %       Proposed multi scale & ???  &???  & ???   & ???       \\
        %   \hline
        \end{tabular}
    \label{tab:ex:tail_evaluation}
  \end{table}%
    %%%%%%%%%%%%%%%%%%%%%%%%%%%%%%%%%%%%%%%%%%%%%%%%%%%


% \begin{itemize}
%   \item balloon segmentation must be done on the exterior edge of the balloon contour. It has an impact on the tail direction computation. 
%   \item issue when the tail does not correspond to any vertex of the convex hull (assumption in 3.4)
% \end{itemize}
% subsection result_analysis (end)

The proposed method was very good at locating the tail tip when it existed and also at giving a confidence value equal to zero when the balloon had no tail.
The few mistakes concerning the tail tip position detection was due to very small tails or when there are variation along the contour but no tail to detect which confuse the vertex selection form the balloon contour convex hull.
The tail direction was sensitive to the quality of the pixel-level balloon contour extraction in the area of the tail tip and the difference between the white and black tips (Figure~\ref{fig:ex:tail_extraction_mistake_errors}).
The tail tip position and tail direction highly rely on the performance of the balloon contour extraction.
They are satisfactory when balloon contours are perfectly extracted (manual in Table~\ref{tab:ex:tail_evaluation}) but decreases quickly according to the automatic balloon extractor method performance.
The importance of a precise pixel-level segmentation in the region of the tail tip is illustrated by the difference of 7.38\% for $A_{tailDir}$ between the regular and implicit automatic methods due to the lack of precision of the active contour extraction method.


  %%%%%%%%%%%%%%%%%%%%%%%%%%%%%%%%%%%%%%%%%%%%%%%%%%%
  \begin{figure}[h]  %trim=l b r t  width=0.5\textwidth,
    \centering
    \includegraphics[trim= 0px 0px 0px 0px, clip, width=0.75\textwidth]{tail_extraction_errors.png}
    \caption[Examples of difficult balloons for tail tip position (left) and tail direction (right) extractions]{Examples of difficult balloons for tail tip position (left) and tail direction (right) extractions. Left balloons have not been detected without tail because of the contour variations, a tail have been mistakenly predicted. In the right balloons, the tail tip position have been correctly predicted but not the tail direction because of the change of direction of the last segment.}
    \label{fig:ex:tail_extraction_mistake_errors}
  \end{figure}
  %%%%%%%%%%%%%%%%%%%%%%%%%%%%%%%%%%%%%%%%%%%%%%%%%%%

% paragraph tail_detection (end)



% subsection sequential_information_extraction_evaluation (end)


\section{Comic character extraction evaluation} % (fold)
\label{sec:comic_character_extraction_evaluation}
We evaluated the detection of comic characters on the 1597 characters of the eBDtheque dataset.
In the eBDtheque ground truth, only 829 (52\%) of the character instances are speaking (and 48\% are not).
Therefore, we do not expect to reach 100\% recall and precision with the Method A and B because they are only able to extract speaking characters.
The independent Method B is potentially able to retrieve all the character instance given one example of each of them.

% \subsubsection{Sequential approach} % (fold)
% \label{sub:sequential_approach}

% subsection sequential_approach (end)
Similarly to the other object extraction methods, the overlapping ratio between a predicted character region and a region from the ground truth should be above 50\% ($a_0>0.5$) for the predicted region to be considered as valid (Section~\ref{sub:ex:object_localisation_metric}).
This ratio can be relaxed according to the application e.g. face only, full body, cropped body, an example of the impact of this ratio on validation and rejection of predicted regions is shown on Figure~\ref{fig:ex:characters_overlap_ratio}.
For instance, the predictions that are presented on the left part of Figure~\ref{fig:ex:characters_overlap_ratio} could be accepted if the target was to roughly detect comic character locations.
In this case, all the predicted region would have been accepted as correct with a validation criterion set to $a_0>0.1$.


%, at level of object using equation
%  with $th_0=0.2$.
%Here we set $th_0=0.2$ (not 0.5) in order to accept prediction of the comic character location even if they are only partial, see figure~\ref{fig:ex:characters_overlap_ratio}.
  %to equation~\ref{eq:recall}.
%In the eBDtheque dataset~\cite{Guerin2013}, ??\% of the characters are in an image where there are no  


    %%%%%%%%%%%%%%%%%%%%%%%%%%%%%%%%%%%%%%%%%%%%%%%%%%%%%%%
    \begin{figure}[ht]%trim=l b r t  width=0.5\textwidth,  
      \centering
      \includegraphics[trim= 0mm 0mm 190mm 0mm, clip, width=0.41\textwidth]{characters_overlap_ratio.png}
      % \hspace{1em}
      \includegraphics[trim= 0mm 0mm 0mm 0mm, clip, width=0.49\textwidth]{overlap_ratio_evolution.pdf}
      \caption[Example of comic character region prediction]{Example of predicted region considered as true (green) or false (red) positives according to the validation criteria $a_0>0.5$ (dashed lines in the right hand side figure). Thin grey rectangles are the ground truth regions and the red numbers in the top left corner of each ground truth region is the value of the overlapping ratio $a_0$.}
      \label{fig:ex:characters_overlap_ratio}
    \end{figure}  
    %%%%%%%%%%%%%%%%%%%%%%%%%%%%%%%%%%%%%%%%%%%%%%%%%%%%%%%

% \subsection{Experimental settings} % (fold)
% \label{sub:experimental_settings}

\subsection{Method A} % (fold)
% \label{sub:method_a}

Method A is a sequential approach that computes a region of interest for comic characters from the tail tip position and tail direction inside a panel region (Section~\ref{sec:se:tail_to_character}).
Its position is shifted in the region of the tail tip according to the tail direction which is quantized in eight cones of $\pi/4$ radius.
We evaluated the performance of this method given the input information of panel, balloon and tail from the eBDtheqie ground truth and from the best automatic extraction methods from Table~\ref{tab:panel_extraction_comparision_results},~\ref{tab:ex:balloon_localisation_comparison_results} and~\ref{tab:ex:tail_evaluation} respectively (Figure~\ref{fig:ex:character_method_A_GT_extraction_detail} and~\ref{fig:ex:character_method_A_auto_extraction_detail}).
Also, two set of images have been tested, one including all the comic character instance of the eBDtheque dataset and an other one with only the speaking character as this method is only able to predict speaking character location from the tail direction (Table~\ref{tab:ex:character_localisation_result_summary}).
% The score of this method was in average for recall and precision of 15.59\% and 83.81% (Figure 6.12c).

 
% Table~\ref{tab:ex:character_localisation_result_summary} shows the results we obtained and the detail are given for each image of the dataset in Figure~\ref{fig:ex:character_methodA_extraction_detail}.


%%%%%%%%%%%%%%%%%%%%%%%%%%%%%%%%%%%%%%%%%%%%%%%%%%
% \begin{table}[ht]
%   % \normalsize
% % \renewcommand{\arraystretch}{1.3}
% % \extrarowheight as needed to properly center the text within the cells
%   \centering
%   \caption{Comic character localisation result for method A from manual and automatic panel and balloon element extractions.}
%   \begin{tabular}{|c|c|c|c|}
%   % \hline
%   % Previous extractions  & \multicolumn{3}{|c|}{Performance}   \\
%   \hline
%   Previous extractions  &  $ R$ (\%)  & $P$ (\%)& $F$ (\%)   \\
%   \hline
%   Ground truth &  15.59    &  23.18   &  18.64   \\
%   \hline
%   Automatic &  6.84    &   12.13     &   8.75  \\
%   \hline
%   \end{tabular}
%       \label{tab:ex:character_localisation_result_summary}
% \end{table}%
    %%%%%%%%%%%%%%%%%%%%%%%%%%%%%%%%%%%%%%%%%%%%%%%%%%%



%%%%%%%%%%%%%%%%%%%%%%%%%%%%%%%%%%%%%%%%%%%%%%%%%%%
\begin{figure}[!ht] %trim=l b r t  width=0.5\textwidth, 
  \centering
  \subfloat[Method A (ground truth)]{\label{fig:ex:character_method_A_GT_extraction_detail}\includegraphics[trim= 0mm 108mm 0mm 0mm, clip, width=\textwidth]{2014-08-28_svg_method_A_from_GT_panel_balloon_Object.pdf}}
  \\
  \subfloat[Method A (automatic)]{\label{fig:ex:character_method_A_auto_extraction_detail}\includegraphics[trim= 0mm 0mm 0mm 0mm, clip, width=\textwidth]{2014-08-28_svg_method_A_from_method_B_panel_and_balloon_Object.pdf}}
  \\
  % \subfloat[Method A (regular)]{\label{fig:ex:balloon_methodA_extraction_detail}\includegraphics[trim= 0mm 108mm 0mm 0mm, clip, width=\textwidth]{2014-07-26_svg_closed_thesis_method_A_Balloon.pdf}}
  % \\
  % \subfloat[Method A (implicit)]{\label{fig:ex:balloon_methodA_extraction_detail}\includegraphics[trim= 0mm 108mm 0mm 0mm, clip, width=\textwidth]{2014-07-22_svg_implicit_ICDAR_Balloon_object_level.pdf}}
  % \\
  % \subfloat[Method B]{\label{fig:ex:balloon_methodB_extraction_detail}\includegraphics[trim= 0mm 108mm 0mm 0mm, clip, width=\textwidth]{2014-07-21_svg_closed_IJDAR_method_B_Balloon.pdf}}
  % \\
  % \subfloat[Method C \modif{TODO: update}]{\label{fig:ex:balloon_methodC_extraction_detail}\includegraphics[trim= 0mm 0mm 0mm 0mm, clip, width=\textwidth]{2014-08-28_svg_method_A_from_method_B_panel_and_balloon_Object.pdf}}

  \caption[Character extraction score details at object level for each image of the eBDtheque dataset]{Character extraction score details at object level for each image of the eBDtheque dataset for speaking and non speaking characters (Appendix~\ref{app:dataset}).}
    \label{fig:ex:character_methodA_extraction_detail}
\end{figure}
%%%%%%%%%%%%%%%%%%%%%%%%%%%%%%%%%%%%%%%%%%%%%%%%%%%

%%%%%%%%%%%%%%%%%%%%%%%%%%%%%%%%%%%%%%%%%%%%%%%%%%%%%%
% \begin{figure}[!h]
%  \includegraphics[width=\textwidth,height=4cm]{Panel_object_proposed.pdf}
%  \caption{Character extraction score details for each image of the eBDtheque dataset \modif{TODO: update}.
%  }
%  \label{fig:ex:character_methodA_extraction_detail}
% \end{figure}
%%%%%%%%%%%%%%%%%%%%%%%%%%%%%%%%%%%%%%%%%%%%%%%%%%%%%%

% subsection method_a (end)

\subsection{Method B} % (fold)
\label{sub:ex:character_extraction_method_b}

Method B consists in an independent comic character instance spotting given an example.
To perform this experiments, we asked the users to manually highlight the object position as described in Section~\ref{sec:in:input_query}.
The selection usually includes hair, head and top body colours because lower body parts are often hidden by the frame border or the posture. 
From the user pixel selection, we computed the bounding box to show that the system does not need a precise object colour selection. 
% We reduced the number of colour to a \modif{standard web palette $P_c$ of 256 colours} because comics are colourful drawings similar to web pages and icons that contain a limited number of colour. 
We fixed the descriptor size to $N=3$ according to the evaluation Figure~\ref{fig:ex:descriptor_size_evaluation}.
In the retrieval process, we set $T=N$ which means that the confidence of the candidate region equal 100\% (all the query colours must be present in the candidate region to be considered as a correct detection).
We used three squared window sizes according to the query height: half of the query height, query height and twice the query height.


%%%%%%%%%%%%%%%%%%%%%%%%%%%%%%%%%%%%%%%%%%%%%%%%%%%
 \begin{figure}[!ht]  %trim=l b r t  width=0.5\textwidth,
   \centering
  \includegraphics[trim= 5mm 0mm 30mm 0mm, clip, width=250px]{descriptor_size_evaluation.png}
  \caption[Character spotting descriptor size validation]{Descriptor size validation. Measure of recall (blue square) and precision (red diamond) detection for different descriptor sizes from a testing set of 17 comics pages. The maximum recall is for $N=3$.}
  \label{fig:ex:descriptor_size_evaluation}
 \end{figure}
%%%%%%%%%%%%%%%%%%%%%%%%%%%%%%%%%%%%%%%%%%%%%%%%%%%


% \paragraph{Performance evaluation}
The following results were obtained with the common evaluation measures of recall, precision at object level which means that we considered as correctly detected the objects that are overlapped by a detected region without considering the percentage of overlapping.
Recall ($R$) is the number of correctly detected object divided by the number of objects to detect.
Precision ($P$) is the number of correctly detected objects divided by the number of detected objects.
%Each detected object $D$ was compared to the corresponding ground truth one $G$ with the nearest centroid, considering only one matching per ground truth balloon.
We consider as candidate regions, regions that have $C=100\%$ confidence (contains at least all the descriptor colours).% Lower confidence detection need additional processing to be considerate as correct results.

We evaluated our method on a subset of 34 comic book images from 10 different albums which represent 22 comic characters that appear 352 times in total.
This subset have been chosen from relevant images from the eBDtheque dataset that are coloured and with a sufficient amount of character instances. 
We obtained 90.3\% recall and 46.7\% precision in average.
Results are detailed in Table~\ref{tab:ex:character_spotting_detail_result} where album 1 to 10 correspond to image identifier in the eBDtheque dataset: MIDAM\_GAMEOVER\_T05, TARZAN\_COCHON, CYB\_COSMOZONE, CYB\_MAGICIENLOOSE, CYB\_MOUETTEMAN, LAMISSEB\_LESNOEILS1, LUBBIN\_LESBULLESDULABO, MIDAM\_KIDPADDLE7, PIKE\_BOYLOVEGIRLS\_T41 and TRONDHEIM\_LES\_TROIS\_CHEMINS\_005 (~\ref{app:dataset}) \modif{TODO: TARZAN\_COCHON is not from eBDtheque dataset.}.
%Another album of two pages has also been evaluated, we had 83\% recall and 66\% precision score.


%%%%%%%%%%%%%%%%%%%%%%%%%%%%%%%%%%%%%%%%%%%%%%%%%%%%%%%%%%%%%%%%%%%%%%
\begin{table}[h]%[htbp]
\caption{Character detection performance}
\begin{tabular}{|c|l|l|r|r|r|}
\hline
\multicolumn{1}{|c|}{\textbf{Album}} & \multicolumn{1}{c|}{\textbf{Nb pages}} & \multicolumn{1}{c|}{\textbf{Nb character}} & \multicolumn{1}{c|}{\textbf{Character ID}}  & \multicolumn{1}{c|}{\textbf{Recall}} & \multicolumn{1}{c|}{\textbf{Precision}} \\ \hline
\multicolumn{1}{|r|}{1}  & \multicolumn{1}{r|}{10} & \multicolumn{1}{r|}{2} & 0  & 92.3\% & 16.8\% \\ \hline
 &  &   & 1 & 91.8\% & 65.0\% \\ \hline
\multicolumn{1}{|r|}{2}  & \multicolumn{1}{r|}{1} & \multicolumn{1}{r|}{1} & 2  & 75.0\% & 26.5\% \\ \hline
\multicolumn{1}{|r|}{3}  & \multicolumn{1}{r|}{5} & \multicolumn{1}{r|}{4} & 3  & 100.0\% & 50.0\% \\ \hline
 &  &   & 4  & 93.3\% & 60.9\% \\ \hline
 &  &   & 5  & 100.0\% & 33.3\% \\ \hline
 &  &   & 6  & 100.0\% & 100.0\% \\ \hline
\multicolumn{1}{|r|}{4}  & \multicolumn{1}{r|}{1} & \multicolumn{1}{r|}{1} & 7  & 85.7\% & 85.7\% \\ \hline
\multicolumn{1}{|r|}{5}  & \multicolumn{1}{r|}{4} & \multicolumn{1}{r|}{2} & 8 & 100.0\% & 75.0\% \\ \hline
 &   &  & 9 & 50.0\% & 27.8\% \\ \hline
\multicolumn{1}{|r|}{6}  & \multicolumn{1}{r|}{5} & \multicolumn{1}{r|}{2} & 10 & 71.8\% & 10.1\% \\ \hline
 &   &  & 11  & 100.0\% & 29.9\% \\ \hline
\multicolumn{1}{|r|}{7}  & \multicolumn{1}{r|}{2} & \multicolumn{1}{r|}{2} & 12 & 100.0\% & 13.8\% \\ \hline
 &   &  & 13 & 100.0\%  & 54.5\% \\ \hline
 &   &  & 14 & 100.0\%  & 55.6\% \\ \hline
\multicolumn{1}{|r|}{8}  & \multicolumn{1}{r|}{2} & \multicolumn{1}{r|}{2} & 15 & 81.8\% & 75.0\% \\ \hline
 &   &  & 16 & 100.0\%  & 12.7\% \\ \hline
\multicolumn{1}{|r|}{9}  & \multicolumn{1}{r|}{2} & \multicolumn{1}{r|}{1} & 17 & 83.3\% & 58.8\% \\ \hline
\multicolumn{1}{|r|}{10} & \multicolumn{1}{r|}{2} & \multicolumn{1}{r|}{4} & 18 & 81.3\% & 35.1\% \\ \hline
 &   &  & 19 & 100.0\% & 48.4\% \\ \hline
 &   &  & 20 & 92.9\% & 17.1\% \\ \hline
 &   &  & 21 & 90.9\% & 24.4\% \\ \hline
 %&  &  &  & \multicolumn{1}{l|}{} & \multicolumn{1}{l|}{} & \multicolumn{1}{l|}{} & \multicolumn{1}{l|}{} & \multicolumn{1}{l|}{} & \multicolumn{1}{l|}{} & \multicolumn{1}{l|}{} \\ \hline
 &  &  &  & &  \\ \hline %\multicolumn{1}{l|}{} & \multicolumn{1}{l|}{} & \multicolumn{1}{l|}{} & \multicolumn{1}{l|}{} & \multicolumn{1}{l|}{} & \multicolumn{1}{l|}{} & \multicolumn{1}{l|}{} \\ \hline
\multicolumn{1}{|r|}{\textbf{10}} & \multicolumn{1}{r|}{\textbf{34}} & \multicolumn{1}{r|}{\textbf{22}} & \multicolumn{1}{r|}{\textbf{22}} & \textbf{90.3\%} & \textbf{46.7\%} \\ \hline
\end{tabular}
\label{tab:ex:character_spotting_detail_result}
\end{table}
%%%%%%%%%%%%%%%%%%%%%%%%%%%%%%%%%%%%%%%%%%%%%%%%%%%%%%%%%%%%%%%%%%%%%%


% subsection method_b (end)

\subsection{Method C} % (fold)
% \label{sub:method_c}

The knowledge-driven method can be used as a post processing of the comic character extraction.
It validates or rejects character candidates according to contextual information from the proposed model (Section~\ref{sub:kn:validation}).
In the model, the rules concerning comic book characters are that they should not contain any other extracted region because we assume that a character can not contain a panel, a balloon, text and another character. If they do, they are automatically deleted.
We post processed the output of Method A and we obtained and average score for recall and precision of \modif{??}\% and \modif{??}\% (Figure~\ref{fig:ex:character_methodC_extraction_detail}).

%%%%%%%%%%%%%%%%%%%%%%%%%%%%%%%%%%%%%%%%%%%%%%%%%%%%%%
\begin{figure}[h]
 \includegraphics[width=\textwidth,height=4cm]{Panel_object_proposed.pdf}
 \caption{Balloon extraction score details using Method C for each image of the eBDtheque dataset \modif{TODO: update}.
 }
 \label{fig:ex:character_methodC_extraction_detail}
\end{figure}
%%%%%%%%%%%%%%%%%%%%%%%%%%%%%%%%%%%%%%%%%%%%%%%%%%%%%%

% subsection method_c (end)

% subsection experimental_settings (end)

\subsection{Comparison and analysis} % (fold)
\label{sub:result_analysis}

% \paragraph{Character localisation} % (fold)
% \label{par:character_localisation}

% paragraph character_localiatioin (end)
The comic characters extraction are presented in Table~\ref{tab:ex:character_localisation_result_summary} for Method A and C.
Method B is evaluated separately on a smaller subset of the dataset.
Note that the scores in Table~\ref{tab:ex:character_localisation_result_summary} are given for the whole dataset including the characters that are not speaking.


 % which shows the performance of two steps of the process:
% \begin{itemize}
%   \item [a)] Hypothesis of ROI from Section~\ref{sub:hypo_roi} (A)
%   \item [b)] Character extraction from Section~\ref{sub:character_extraction} (B)
%   \item [c)] Character extraction validation from Section~\ref{sub:validation} (C)
%   % \item [c)] Character spotting (see section~\ref{sub:character_extraction})
% \end{itemize}
%for the hypothesis of ROI of characters ($a$) corresponding to section~\ref{sub:hypo_roi}, the query pre processing ($b$) corresponding to section~\ref{pa:query_pre_processing} and the character spotting ($c$) of section~\ref{sub:character_extraction}.% and figure~\ref{fig:overlap_ratio_evolution}.


%%%%%%%%%%%%%%%%%%%%%%%%%%%%%%%%%%%%%%%%%%%%%%%%%%
\begin{table}[ht]
  % \normalsize
% \renewcommand{\arraystretch}{1.3}
% \extrarowheight as needed to properly center the text within the cells
  \centering
  \caption{Comic character localisation result for method A from manual and automatic panel and balloon element extractions.}
  \begin{tabular}{|c|c|c|c|c|c|c|}
  \hline
    & \multicolumn{3}{|c|}{Speaking only}  & \multicolumn{3}{|c|}{All}   \\
  \hline
  Methods  &  $ R$ (\%)  & $P$ (\%)& $F$ (\%)   &  $R$ (\%)  & $P$ (\%)   & $F$ (\%)\\
  \hline
  Method A (ground truth)  & \textbf{30.04} & \textbf{28.03} & \textbf{29.00} &  15.59    &  23.18   &  18.64   \\
  \hline
  Method A (automatic)     & 11.77  & 11.82 & 11.79 & 6.84    &   12.13     &   8.75  \\
  \hline
  Method C (ground truth)  &   & & &     &     &    \\
  \hline
  Method C (automatic)     &   & & &    &      &   \\
  \hline
  \end{tabular}
      \label{tab:ex:character_localisation_result_summary}
\end{table}%
    %%%%%%%%%%%%%%%%%%%%%%%%%%%%%%%%%%%%%%%%%%%%%%%%%%%


    %%%%%%%%%%%%%%%%%%%%%%%%%%%%%%%%%%%%%%%%%%%%%%%%%%%
%   \begin{table}[ht]
%     \normalsize
% %\renewcommand{\arraystretch}{1.2}

%     \centering
%     \caption{Character localisation results.}
%     \begin{tabular}{|c|c|c|c|}
%           % \hline
%           %   & \multicolumn{2}{|c|}{Character 1}   & \multicolumn{2}{|c|}{Character 2}   \\
%           \hline
%           &  $R$ (\%)  & $P$ (\%)   & $F$ (\%)  \\
%           % \hline
%           % Before validation   & ?   & ?           \\
%           \hline
%           %Proposed (ROI)   & 70   & 49           \\
%           Method A   & \modif{??}   & \modif{??}    & \modif{??}      \\
%           \hline
%           %Proposed (ROI + spotting)  & \modifc{46}   & \modifc{43}           \\
%           % Method B (all query?) & \modifc{?} & \modifc{?} & \modifc{?} \\
%           % \hline
%           % %Proposed (ROI + spotting)  & \modifc{46}   & \modifc{43}           \\
%           % $a + b + c$  & \modifc{?} & \modifc{?} & ? \\
%           % \hline
%           Method C   & \modif{??}   & \modif{??}    & \modif{??}      \\
%           \hline
%           % TOTAL   & 93.4    & 92.8          \\
%           % \hline
%         %       Proposed multi scale & ???  &???  & ???   & ???       \\
%         %   \hline
%         \end{tabular}
%     \label{tab:ex:character_localisation_result_summary}
%   \end{table}%
    %%%%%%%%%%%%%%%%%%%%%%%%%%%%%%%%%%%%%%%%%%%%%%%%%%%


    % %%%%%%%%%%%%%%%%%%%%%%%%%%%%%%%%%%%%%%%%%%%%%%%%%%%%%%%%
    % \begin{figure}[ht]%trim=l b r t  width=0.5\textwidth,  
    %   \centering
    %   \includegraphics[width=200px]{fig/overlap_ratio_evolution.pdf}
    % \caption{Score evolution according to overlap ratio. Black lines represent the score for $a_0=0.5$ as given in table~\ref{tab:ex:character_localisation_result_summary}.}
    % \label{fig:overlap_ratio_evolution}
    % \end{figure}
    %%%%%%%%%%%%%%%%%%%%%%%%%%%%%%%%%%%%%%%%%%%%%%%%%%%%%%%%

% As shown on Figure~\ref{fig:ex:characters_overlap_ratio}, the decision criterion $a_0>0.5$ may seem restrictive and have to be adjusted according to the application.
This low level of performance comes from various reasons.
First, the limitation of the extractor to process speaking characters
When we focused on the subset of 829 speaking characters, the recall and precision increased by 14.45\% and 4.85\% respectively.
%First, the limitation of the extractor to process speaking characters; 
Second, the variability of character styles in the eBDtheque dataset; third, the error propagation of previous processes (panel, text, balloon, and tail extractions) that are required to guess comic character locations.


% subsection result_analysis (end)

% \paragraph{Character spotting} % (fold)
% \label{par:character_spotting}

% \paragraph{Detection results}
% \label{sec:eval}

As mentioned Section~\ref{sub:ex:character_extraction_method_b}, Method B have been evaluated on a subset of the eBDtheque dataset composed by coloured images and with a sufficient amount of character instances.
The experiments shown interesting detection results based on a user-defined colours query descriptor.
% The processing time was about 6 seconds per image for a A4 300DPI image on a regular machine.
% Most of the time consumption is for the mask creation and it is proportional to the number of mask to create.
Detection result examples are illustrated Figure~\ref{fig:ex:character_spotting_results}.

The detected region aims to localize the smallest regions containing all the colours of the query descriptor.
This region is usually smaller than the ground truth region which is defined at character bounding box level.
A post-processing is needed to compute the character bounding box from the detected region by extending it to the colour region boundaries.

%%%%%%%%%%%%%%%%%%%%%%%%%%%%%%%%%%%%%%%%%%%%%%%%%%%
 \begin{figure}[!h]  %trim=l b r t  width=0.5\textwidth,
   \centering
  \includegraphics[width=0.5\textwidth]{result_exemples.png}
  \caption[Character spotting result sample]{
  Each line shows a character query region bounding box in white (left column), a correct detection (middle column) and a wrong detection (right column). Green rectangles represents the ground truth region and the red rectangle corresponds to the region detections. The corresponding character IDs in Table~\ref{tab:ex:character_spotting_detail_result} are 12, 8, 0, 10, 16 and 2 from top to bottom. %and second columns show correct and wrong detection result for album 1. Third and fourth columns are correct and wrong detection results examples from album 2. Top first and top third columns are the user defined queries. %{\bf Update true and false positive examples and draw detection window (according to $S$ definition)
  }
  \label{fig:ex:character_spotting_results}
 \end{figure}
%%%%%%%%%%%%%%%%%%%%%%%%%%%%%%%%%%%%%%%%%%%%%%%%%%%

Correct detections show the variety of comics character position and deformation that we are able to detect with the presented framework. There are few missing detection and over detection are essentially due to other comics character detection and image pre processing (colour reduction).
%character part missing, small size objects or low resolution images.

    %%%%%%%%%%%%%%%%%%%%%%%%%%%%%%%%%%%%%%%%%%%%%%%%%%%
%   \begin{table}[ht]
%     \normalsize
% \renewcommand{\arraystretch}{1.2}
% 
%     \centering
%     \caption{Comics character detection results.}
%     \begin{tabular}{|c|c|c|c|c|}
%           \hline
%             & \multicolumn{2}{|c|}{Character 1}   & \multicolumn{2}{|c|}{Character 2}   \\
%           \hline
%           &  Detected   & Missed    &  Detected  & Missed   \\
%           \hline
%           Album 1, p. 1   & 7       & 2     & 95.9      & 90.1    \\
%           \hline
%           Album 1, p. 2   & 8       & 2     & 78.4      & 94.8    \\
%           \hline
%           Album 1, p. 3   & 5   & 1     & 97.7    & 88.7      \\
%           \hline
%           TOTAL   & 93.4    & 92.8    & 97.7    & 88.7      \\
%           \hline
%         %       Proposed multi scale & ???  &???  & ???   & ???       \\
%         %   \hline
%         \end{tabular}
%     \label{tab:high_res}
%   \end{table}%
    %%%%%%%%%%%%%%%%%%%%%%%%%%%%%%%%%%%%%%%%%%%%%%%%%%%



%balloon localization performance at bounding box level to highlight the benefits of both active contour theory and domain specific knowledge. Second, we performed pixel level evaluation on a smaller subset to show the ability of our method to fit balloon contour details.


% \paragraph{Discussion}
\modif{TODO: check this discussion}
The proposed framework gives promising results for contemporary comics, we also tried on historical such as FRED\_PHILEMON12 and MCCALL\_ROBINHOOD from the eBDtheque dataset but the printing process generates a lot of noise (thickness of ink). 
A pre-processing smoothing is required in this case~\cite{l0smoothing2011,Kopf2012DigitalReconstruction}.
The colour palette we used has 256 colours, we believe that we can improve the presented method by computing the palette from the user's query in order to ignore unwanted colours and speed up the process.
The sliding windows approach does not allow to segment comics character at a pixel level, a distance measure between the colour mask regions could solve this.
Also, once we retrieved all colour similar objects, we could learn a shape model and try to find more objects occurrences based on shape information in a second stage, this will be a future work.


% paragraph character_spotting (end)

% subsection sequential_comic_character_extraction_evaluation (end)

% \section{Independent information extraction evaluation} % (fold) %IJDAR
% \label{sub:ex:independent_information_extraction_evaluation}

% TODO

% \subsection{Panel extraction evaluation} % (fold)
% \label{sub:ex:independent_panel_extraction_evaluation}
% We evaluated our method on the 850 panels of the eBDtheque dataset~\cite{Guerin2013} ``version 2014'' at bounding box level.
% Assuming that a panel is a big region, we ignored the panel detection with a area lower than 4\% ($minAreaFactor$) of the page area according to a validation on the eBDtheque dataset.
% table~\ref{tab:minAreaFactor_validation}.

%Those heuristics have been chosen very large to filter out aberrant regions such as very small region and page border.

% Table~\ref{tab:panel_extraction_comparision_results} presents the average results we obtained compared to our previous method~\cite{Rigaud2012LNCS} and a method from the literature~\cite{Arai11}.

%     %%%%%%%%%%%%%%%%%%%%%%%%%%%%%%%%%%%%%%%%%%%%%%%%%%%
%   \begin{table}[ht]
%     \normalsize
% %\renewcommand{\arraystretch}{1.2}

%     \centering
%     \caption{Panel extraction results.}
%     \begin{tabular}{|c|c|c|c|}
%           % \hline
%           %   & \multicolumn{2}{|c|}{Character 1}   & \multicolumn{2}{|c|}{Character 2}   \\
%           \hline
%           &  $R$ (\%)  & $P$ (\%)  & $F$ (\%)     \\
%           \hline
%           Arai~\cite{Arai11}   & 58.03       & 75.30    & 65.55    \\
%           \hline
%           Rigaud~\cite{Rigaud2012LNCS}   & 78.02       & 73.17   & 75.52     \\
%           \hline
%           Proposed   & 81.24     & 86.55     & 83.81      \\
%           \hline
%           Proposed + validation   & 80.69     & 87.03     & 83.74      \\
%           \hline
%            % TOTAL   & 93.4    & 92.8          \\
%           % \hline
%         %       Proposed multi scale & ???  &???  & ???   & ???       \\
%         %   \hline
%         \end{tabular}
%     \label{tab:panel_extraction_comparision_results}
%   \end{table}%
%     %%%%%%%%%%%%%%%%%%%%%%%%%%%%%%%%%%%%%%%%%%%%%%%%%%%

% %TODO: re evaluate Arai's result from his method (still waiting for is email 2014-07-01)

% The proposed panel extraction based on connected component analysis is simple to implement, and is a fast and efficient method for comics with disconnected panels (separated by a white gutter).
% The validation by the expert system was not significant here because the low level processing had already reached the limits of the model.
% Figure~\ref{ap:panel_extraction} shows the details for each image tested, which were mainly comics with gutters.
% % The proposed method is not appropriate for gutterless comics such as ``INOUE'' or strip without panel border such as ``MONTAIGNE'' or a extra frame around several panels (``SAINTOGAN\_PROSPER'').

% \begin{figure}
%  \includegraphics[width=\textwidth,height=4cm]{Panel_object_proposed.pdf}
%  \caption[Independent panel extraction score details]{Panel extraction score details for each image of the eBDtheque dataset.
%  }
%  \label{ap:panel_extraction}
% \end{figure}

% Our method is not appropriate for gutterless comics (e.b. some mangas) or strip without panel borders such as those with an extra frame around several panels.

% Another weakness is when panels are connected by other elements.
% This experiment was performed in 28 seconds for the whole dataset using one CPU at 2.5GHz (0.05s per panel on average).
% Note that some of the dataset images were digitized with a dark background surrounding the cover of the book.
% We automatically remove this by cropping the image where a panel with an area $>$ 90\% of the page area was detected.

% subsection independent_panel_extraction_evaluation (end)


% \subsection{Balloon extraction evaluation} % (fold)
% \label{sub:ex:independent_balloon_extraction_evaluation}

% We evaluated our method on the 1092 balloons of the eBDtheque dataset~\cite{Guerin2013} ``version 2014'' at object bounding box level, which includes the tail.
% Note that our method does not require any previous processing, in contrast to~\cite{rigaud2013active} and it is able to detect closed balloons only.
% In the eBDtheque ground truth, only 84.5\% of the balloons are closed and 15.5\% are not.
% Thus we did not expect to reach 100\% recall and precision.


% The minimum number of children $minNbChildren$ of a balloon was set to 8, just before the first peak in the distribution of the number of letters per speech balloon in the eBDtheque dataset~\cite{Guerin2013} (Figure~\ref{fig:ex:min_number_children_validation}).


%     %%%%%%%%%%%%%%%%%%%%%%%%%%%%%%%%%%%%%%%%%%%%%%%%%%%%%%%%
%     \begin{figure}[ht]%trim=l b r t  width=0.5\textwidth,  
%       \centering
%       \includegraphics[trim= 10px 0px 60px 0px, clip, width=0.75\textwidth]{number_of_letter_per_balloon.pdf}
%       \caption[Distribution of the number of letter per speech balloon]{Distribution of the number of letter per speech balloon.
%       }
%       \label{fig:ex:min_number_children_validation}
%     \end{figure}  
%     %%%%%%%%%%%%%%%%%%%%%%%%%%%%%%%%%%%%%%%%%%%%%%%%%%%%%%%%


% Note that in Figure~\ref{fig:ex:min_number_children_validation}, there are about 3.5\% of the balloons bellow the selected threshold that contain one or two letters, usually punctuation marks.
% %such as ``?'' or ``!''. 
% We voluntary omitted them here to avoid detecting a lot of non balloon regions.

% Balloons with a confidence value $C_{balloon}$ lower than 10\% were rejected according to the validation experiments on the eBDtheque dataset~\cite{Guerin2013}.
%  % table~\ref{tab:balloon_confidence_validation}.

% Table~\ref{tab:balloon} shows the average results for the one hundred images of the dataset.
% We also compare to a state of the art method from the literature~\cite{Arai11} and our previous work~\cite{rigaud2013active} based on the best results we had obtained for text localisation (Table~\ref{tab:text_results}).


    %%%%%%%%%%%%%%%%%%%%%%%%%%%%%%%%%%%%%%%%%%%%%%%%%%%
%   \begin{table}[ht]
%     \normalsize
% %\renewcommand{\arraystretch}{1.2}

%     \centering
%     \caption{Balloon extraction results.}
%     \begin{tabular}{|c|c|c|c|}
%           % \hline
%           %   & \multicolumn{2}{|c|}{Character 1}   & \multicolumn{2}{|c|}{Character 2}   \\
%           \hline
%           &  $R$ (\%)  & $P$ (\%)  & $F$ (\%)   \\
%           \hline
%           Arai~\cite{Arai11}   & 6.66       & 10.98   & 8.29     \\
%           \hline
%           %Rigaud~\cite{rigaud2013active} based on GT & 70       & 50     & ?   \\
%           Rigaud~\cite{rigaud2013active} & 46.13       & 17.44     & 25.31   \\
%           \hline
%           Proposed   & 57.90       & 73.84      & 64.91     \\
%           \hline
%           Proposed + validation  & 54.79       & 88.76      & 67.75     \\
%           \hline

%           % TOTAL   & 93.4    & 92.8          \\
%           % \hline
%         %       Proposed multi scale & ???  &???  & ???   & ???       \\
%         %   \hline
%         \end{tabular}
%     \label{tab:balloon}
%   \end{table}%
    %%%%%%%%%%%%%%%%%%%%%%%%%%%%%%%%%%%%%%%%%%%%%%%%%%%

%TODO: update Arai's results with is results
%TODO: I can't find the data for Rigaud~\cite{rigaud2013active}, re evaluation is needed

% Our method outperforms~\cite{Arai11} thanks to its genericity, since it can process all the image styles of the eBDtheque dataset.
% This was expected as~\cite{Arai11} was specifically developed for manga comics that have certain stylistic particularities.
% We also surpassed our previous method~\cite{rigaud2013active} because it needs text lines as input which were given in our proposed text extraction method (Section~\ref{sub:text_extraction}).
% Here we clearly see the limitations of dependency between the processing.
% The performance of our text extractor was 49.75\% (Table~\ref{tab:text_results}) which was used as input for balloon extraction so the balloon extraction~\cite{rigaud2013active} was inevitably affected.
% The validation of the expert system once again improve the precision but decreased the recall of the extraction while improving the overall f-measure by almost 3\%.
% The drop in recall was due to the balloons that were correctly extracted but which contained no detectable.
% %Nevertheless, the proposed method does not beat our previous method~\cite{rigaud2013active} that is also able to extract non closed balloons from text regions (from the ground truth here).
% Figure~\ref{fig:text_extraction_detail} confirms that our method works best when the balloons are closed, well segmented and with non cursive text inside.
% This experiment was performed in 22 minutes for the whole dataset using one CPU at 2.5GHz (2.2s per balloon on average).



% \subsection{Text extraction evaluation} % (fold)
% \label{sub:ex:independent_text_extraction_evaluation}
% We evaluated our method for text extraction on the 4667 text lines of the eBDtheque dataset~\cite{Guerin2013} ``version 2014'' at object bounding box level.
% according with $th_0=0.5$. % to equation~\ref{eq:recall}.

%We also evaluated the transcription considering as correct the text line with an edit distance $<$ 3 compared to the ground truth.

%     %%%%%%%%%%%%%%%%%%%%%%%%%%%%%%%%%%%%%%%%%%%%%%%%%%%
%   \begin{table}[ht]
%     \normalsize
%     \centering
%     \caption{Text localisation results.}
%     \begin{tabular}{|c|c|c|c|}
%           % \hline
%           %   & \multicolumn{2}{|c|}{Character 1}   & \multicolumn{2}{|c|}{Character 2}   \\
%           \hline
%           &  $R$ (\%)  & $P$ (\%)  & $F$ (\%)  \\
%           % \hline
%           % Before validation   & ?   & ?           \\
%           \hline
%           Rigaud~\cite{Rigaud2013VISAPP}  & 61.00   & 19.66    & 29.75       \\
%           \hline
%           Proposed (\cite{Rigaud2013VISAPP}+OCR)  & 60.13   & 42.43   & 49.75        \\
%           \hline
%           Proposed + validation   & 44.54     & 65.05     & 52.88      \\
%           % \hline
%           %Proposed (\cite{Rigaud2013VISAPP}+OCR+$ST$ only)  & ?   & ?   & ?        \\
%           % \hline
%           % Proposed (OCR transcription)  & ?   & ?           \\
%           \hline
%         \end{tabular}
%     \label{tab:text_results}
%   \end{table}%
%     %%%%%%%%%%%%%%%%%%%%%%%%%%%%%%%%%%%%%%%%%%%%%%%%%%%

% In our previous work~\cite{Rigaud2013VISAPP} text extraction was evaluated on a subset of 20 pages of the eBDtheque dataset~\cite{Guerin2013}.
% Here we applied it to the whole dataset.
% We used our previous method as a baseline to show an improvement in the precision of 20\% when using an OCR-based filter, without a significant loss in recall.
% The validation by the expert system improved the precision as expected but also resulted in a drop in recall.
% The drop in recall can be explained by the fact that the text extractor is also able to detect texts which are not in the speech balloons but the model considers them as noise.
% % in recall that could probably be filled for specific writing style by training the OCR on a specific font. %, when using an OCR system to filter out non text regions.
% As in~\cite{Rigaud2013VISAPP}, this method has some difficulty coping with certain types of text that can be found in the comics e.g. graphic sounds.

% \paragraph{Text recognition evaluation} % (fold)
% \label{par:ex:text_recognition_evaluation}

% We also evaluated text transcription using string edit distance~\cite{wagner1974string} between a predicted text transcription given by the OCR and its corresponding transcription in the ground truth. % $T_{gr}$.
% The eBDtheque dataset is composed of English, Japanese and French texts.
% We evaluated the subset of English and French pages using the OCR with the corresponding training data\footnote{\url{https://code.google.com/p/tesseract-ocr/downloads/list}}. % corresponding to the language defined in the ground truth of each image of the dataset.
% This was performed at the text line level taking as correct the text lines that were transcribed exactly as the ground truth transcription, considering all the letters as lower case and ignoring accents (for predicted and ground truth regions).
% % with a edit distance $<$ 10\% of their length using equation~\ref{eq:recall} and~\ref{eq:precision}.
% %For instance a recognised text line of ten elements will be considered as correct if less than three are split, deleted, transposed, replaced or inserted. 
% We obtained a score of $A_{textReco}=$7.18\% which constitute a baseline for future work on text recognition on the eBDtheque dataset~\cite{Guerin2013}.
% We performed a more relaxed evaluation where we also considered as correct the text lines at a text edit distance equal to one, the accuracy rise to 10.46\%.
% %The distribution of the text line lengths is given in Figure~\ref{fig:textline_lenth_distribution}.
% Note that the average text line length is quite short in comic books compared to other documents, the distribution is given Figure~\ref{fig:ex:textline_lenth_distribution} for the eBDtheque dataset.

    %%%%%%%%%%%%%%%%%%%%%%%%%%%%%%%%%%%%%%%%%%%%%%%%%%%%%%%%
    % \begin{figure}[ht]%trim=l b r t  width=0.5\textwidth,  
    %   \centering
    %   \includegraphics[trim= 0px 5px 65px 5px, clip, width=0.85\textwidth]{number_of_letter_per_textline.png}
    %   \caption[Distribution of the number of letter per text lines]{Distribution of the number of letter per text lines.
    %   }
    %   \label{fig:textline_lenth_distribution}
    % \end{figure}  
    %%%%%%%%%%%%%%%%%%%%%%%%%%%%%%%%%%%%%%%%%%%%%%%%%%%%%%%%

% Note that in Figure~\ref{fig:ex:textline_lenth_distribution}, there are more than a hundred text lines of only one letter corresponding to punctuation or single letter words such as ``I'' or ``A''; this is a particularity of comics.

% paragraph text_recognition_evaluation (end)


% \subsection{Comic character extraction evaluation} % (fold)
% \label{ssub:independent_comic_character_extraction_evaluation}

% We evaluated our method for character spotting on \modif{??} of 1597 comic character instances of the eBDtheque dataset~\cite{Guerin2013} ``version 2014'' at object bounding box level.

% To perform this experiments, we asked the users to manually highlight the object position as described in section~\ref{sec:input_query}.
% The selection usually includes hair, head and top body colours because lower body parts are often hidden by the frame border or the posture. 
% From the user pixel selection, we computed the bounding box to show that the system does not need a precise object colour selection. 
% We reduced the number of colour to a \modif{standard web palette $P_c$ of 256 colours} because comics are colourful drawings similar to web pages and icons that contain a limited number of colour. 
% We fixed the descriptor size to $N=3$ according to the evaluation figure~\ref{fig:ex:descriptor_size_evaluation}.
% In the retrieval process, we set $T=N$ which means that the confidence of the candidate region equal 100\% (all the query colours must be present in the candidate region to be considered as a correct detection).
% We used three squared window sizes according to the query height: half of the query height, query height and twice the query height.


% %%%%%%%%%%%%%%%%%%%%%%%%%%%%%%%%%%%%%%%%%%%%%%%%%%%
%  \begin{figure}[!ht]  %trim=l b r t  width=0.5\textwidth,
%    \centering
%   \includegraphics[trim= 5mm 0mm 30mm 0mm, clip, width=250px]{descriptor_size_evaluation.png}
%   \caption[Character spotting descriptor size validation]{Descriptor size validation. Measure of recall (blue square) and precision (red diamond) detection for different descriptor sizes from a testing set of 17 comics pages. The maximum recall is for $N=3$.}
%   \label{fig:ex:descriptor_size_evaluation}
%  \end{figure}
% %%%%%%%%%%%%%%%%%%%%%%%%%%%%%%%%%%%%%%%%%%%%%%%%%%%


% \paragraph{Performance evaluation}
% The following results were obtained with the common evaluation measures of recall, precision at object level which means that we considered as correctly detected the objects that are overlapped by a detected region without considering the percentage of overlapping.
% Recall ($R$) is the number of correctly detected object divided by the number of objects to detect.
% Precision ($P$) is the number of correctly detected objects divided by the number of detected objects.
% %Each detected object $D$ was compared to the corresponding ground truth one $G$ with the nearest centroid, considering only one matching per ground truth balloon.
% We consider as candidate regions, regions that have $C=100\%$ confidence (contains at least all the descriptor colours).% Lower confidence detection need additional processing to be considerate as correct results.

% We evaluated our method on 34 comics pages from 10 different albums which represent 22 comic characters that appear 352 times in total. We obtained 90.3\% recall and 46.7\% precision in average. Results are detailed in table~\ref{tab:ex:character_spotting_detail_result} where album 1 to 10 correspond to MIDAM GAMEOVER T05, TARZAN COCHON, CYB COSMOZONE, CYB MAGICIENLOOSE, CYB MOUETTEMAN, LAMISSEB LESNOEILS1, LUBBIN LESBULLESDULABO, MIDAM KIDPADDLE7, PIKE BOYLOVEGIRLS T41 and TRONDHEIM LES TROIS CHEMINS 005 respectively from the eBDtheque dataset~\cite{Guerin2013}. %Another album of two pages has also been evaluated, we had 83\% recall and 66\% precision score.


% \begin{table}[htbp]
% \caption{Character detection performance}
% \begin{tabular}{|c|l|l|r|r|r|}
% \hline
% \multicolumn{1}{|c|}{\textbf{Album}} & \multicolumn{1}{c|}{\textbf{Nb pages}} & \multicolumn{1}{c|}{\textbf{Nb character}} & \multicolumn{1}{c|}{\textbf{Character ID}}  & \multicolumn{1}{c|}{\textbf{Recall}} & \multicolumn{1}{c|}{\textbf{Precision}} \\ \hline
% \multicolumn{1}{|r|}{1}  & \multicolumn{1}{r|}{10} & \multicolumn{1}{r|}{2} & 0  & 92.3\% & 16.8\% \\ \hline
%  &  &   & 1 & 91.8\% & 65.0\% \\ \hline
% \multicolumn{1}{|r|}{2}  & \multicolumn{1}{r|}{1} & \multicolumn{1}{r|}{1} & 2  & 75.0\% & 26.5\% \\ \hline
% \multicolumn{1}{|r|}{3}  & \multicolumn{1}{r|}{5} & \multicolumn{1}{r|}{4} & 3  & 100.0\% & 50.0\% \\ \hline
%  &  &   & 4  & 93.3\% & 60.9\% \\ \hline
%  &  &   & 5  & 100.0\% & 33.3\% \\ \hline
%  &  &   & 6  & 100.0\% & 100.0\% \\ \hline
% \multicolumn{1}{|r|}{4}  & \multicolumn{1}{r|}{1} & \multicolumn{1}{r|}{1} & 7  & 85.7\% & 85.7\% \\ \hline
% \multicolumn{1}{|r|}{5}  & \multicolumn{1}{r|}{4} & \multicolumn{1}{r|}{2} & 8 & 100.0\% & 75.0\% \\ \hline
%  &   &  & 9 & 50.0\% & 27.8\% \\ \hline
% \multicolumn{1}{|r|}{6}  & \multicolumn{1}{r|}{5} & \multicolumn{1}{r|}{2} & 10 & 71.8\% & 10.1\% \\ \hline
%  &   &  & 11  & 100.0\% & 29.9\% \\ \hline
% \multicolumn{1}{|r|}{7}  & \multicolumn{1}{r|}{2} & \multicolumn{1}{r|}{2} & 12 & 100.0\% & 13.8\% \\ \hline
%  &   &  & 13 & 100.0\%  & 54.5\% \\ \hline
%  &   &  & 14 & 100.0\%  & 55.6\% \\ \hline
% \multicolumn{1}{|r|}{8}  & \multicolumn{1}{r|}{2} & \multicolumn{1}{r|}{2} & 15 & 81.8\% & 75.0\% \\ \hline
%  &   &  & 16 & 100.0\%  & 12.7\% \\ \hline
% \multicolumn{1}{|r|}{9}  & \multicolumn{1}{r|}{2} & \multicolumn{1}{r|}{1} & 17 & 83.3\% & 58.8\% \\ \hline
% \multicolumn{1}{|r|}{10} & \multicolumn{1}{r|}{2} & \multicolumn{1}{r|}{4} & 18 & 81.3\% & 35.1\% \\ \hline
%  &   &  & 19 & 100.0\% & 48.4\% \\ \hline
%  &   &  & 20 & 92.9\% & 17.1\% \\ \hline
%  &   &  & 21 & 90.9\% & 24.4\% \\ \hline
%  %&  &  &  & \multicolumn{1}{l|}{} & \multicolumn{1}{l|}{} & \multicolumn{1}{l|}{} & \multicolumn{1}{l|}{} & \multicolumn{1}{l|}{} & \multicolumn{1}{l|}{} & \multicolumn{1}{l|}{} \\ \hline
%  &  &  &  & &  \\ \hline %\multicolumn{1}{l|}{} & \multicolumn{1}{l|}{} & \multicolumn{1}{l|}{} & \multicolumn{1}{l|}{} & \multicolumn{1}{l|}{} & \multicolumn{1}{l|}{} & \multicolumn{1}{l|}{} \\ \hline
% \multicolumn{1}{|r|}{\textbf{10}} & \multicolumn{1}{r|}{\textbf{34}} & \multicolumn{1}{r|}{\textbf{22}} & \multicolumn{1}{r|}{\textbf{22}} & \textbf{90.3\%} & \textbf{46.7\%} \\ \hline
% \end{tabular}
% \label{tab:ex:character_spotting_detail_result}
% \end{table}


% \paragraph{Detection results}
% % \label{sec:eval}

% We evaluated present interesting detection results based on a three colours user defined query descriptor (see section~\ref{sec:proposed_method}). 
% The processing time was about 6 seconds per image for a A4 300DPI image on a regular machine. Most of the time consumption is for the mask creation and it is proportional to the number of mask to create.  Detection result examples are presented figure~\ref{fig:ex:character_spotting_results}.

% The detected region aims to localize the smallest regions containing all the colours of the query descriptor. This region is usually smaller than the ground truth region which is defined at character bounding box level. A post processing is needed to compute the character bounding box from the detected region by extending it to the colour region boundaries.

% %%%%%%%%%%%%%%%%%%%%%%%%%%%%%%%%%%%%%%%%%%%%%%%%%%%
%  \begin{figure}[!ht]  %trim=l b r t  width=0.5\textwidth,
%    \centering
%   \includegraphics[width=0.5\textwidth]{result_exemples.png}
%   \caption[Character spotting result sample]{
%   Each line shows a character query region bounding box in white (left column), a correct detection (middle column) and a wrong detection (right column). Green rectangles represents the ground truth region and the red rectangle corresponds to the region detections. The corresponding character IDs in table~\ref{tab:ex:character_spotting_detail_result} are 12, 8, 0, 10, 16 and 2 from top to bottom. %and second columns show correct and wrong detection result for album 1. Third and fourth columns are correct and wrong detection results examples from album 2. Top first and top third columns are the user defined queries. %{\bf Update true and false positive examples and draw detection window (according to $S$ definition)
%   }
%   \label{fig:ex:character_spotting_results}
%  \end{figure}
% %%%%%%%%%%%%%%%%%%%%%%%%%%%%%%%%%%%%%%%%%%%%%%%%%%%

% Correct detections show the variety of comics character position and deformation that we are able to detect with the presented framework. There are few missing detection and over detection are essentially due to other comics character detection and image pre processing (colour reduction).%character part missing, small size objects or low resolution images.

%     %%%%%%%%%%%%%%%%%%%%%%%%%%%%%%%%%%%%%%%%%%%%%%%%%%%
% %   \begin{table}[ht]
% %     \normalsize
% % \renewcommand{\arraystretch}{1.2}
% % 
% %     \centering
% %     \caption{Comics character detection results.}
% %     \begin{tabular}{|c|c|c|c|c|}
% %           \hline
% %             & \multicolumn{2}{|c|}{Character 1}   & \multicolumn{2}{|c|}{Character 2}   \\
% %           \hline
% %           &  Detected   & Missed    &  Detected  & Missed   \\
% %           \hline
% %           Album 1, p. 1   & 7       & 2     & 95.9      & 90.1    \\
% %           \hline
% %           Album 1, p. 2   & 8       & 2     & 78.4      & 94.8    \\
% %           \hline
% %           Album 1, p. 3   & 5   & 1     & 97.7    & 88.7      \\
% %           \hline
% %           TOTAL   & 93.4    & 92.8    & 97.7    & 88.7      \\
% %           \hline
% %         %       Proposed multi scale & ???  &???  & ???   & ???       \\
% %         %   \hline
% %         \end{tabular}
% %     \label{tab:high_res}
% %   \end{table}%
%     %%%%%%%%%%%%%%%%%%%%%%%%%%%%%%%%%%%%%%%%%%%%%%%%%%%



% %balloon localization performance at bounding box level to highlight the benefits of both active contour theory and domain specific knowledge. Second, we performed pixel level evaluation on a smaller subset to show the ability of our method to fit balloon contour details.


% \paragraph{Discussion}
% The proposed framework gives promising results for contemporary comics, we also tried on historical such as FRED PHILEMON12 and MCCALL ROBINHOOD from the eBDtheque dataset but the printing process generates a lot of noise (thickness of ink). 
% A pre-processing smoothing~\cite{l0smoothing2011} is required in this case. The colour palette we used has 256 colours, we believe that we can improve the presented method by computing the palette from the user's query in order to ignore unwanted colours and speed up the process. 
% The sliding windows approach does not allow to segment comics character at a pixel level, a distance measure between the colour mask regions could solve this.
%  Also, once we retrieved all colour similar objects, we could learn a shape model and try to find more objects occurrences based on shape information in a second stage, this will be a future work.


% ssubsection independent_comic_character_extraction_evaluation (end)

\section{Semantic links evaluation} % (fold)
\label{sec:semantic_links_evaluation}

% \subsection{Experimental settings} % (fold)
% \label{sub:experimental_settings}

\subsection{Method C} % (fold)
% \label{sub:method_c}
TODO? Only for the inference system~\ref{sub:inference_from_low_level}?

Method C is the only method presented in this thesis able to infer semantic links from the image and comics models (Section~\ref{sec:kn:model}).
The semantic links between speech text and speech balloon are called $STSB$ and the ones between speech balloon and speaking character $SBSC$; they characterise a dialogue.
They are considered true or false according to their existence or not in the ground truth (Section~\ref{sub:ex:semantic_links_metric}).
%We evaluated the semantic relations $STSB$ and $SBSC$ according to the metadata in the ground truth of the eBDtheque dataset~\cite{Guerin2013} called \emph{isLineOf} and \emph{isSaidBy}, which represent 3427 and 829 relations respectively.
Given the panel, balloon and character position from the ground truth, the accuracy of the expert system in predicting the semantic relations was about $A_{STSB}=96.9\%$ and $A_{SBSC}=70.66\%$.

The 3.1\% of missed \emph{isLineOf} relations came from balloons that were not compliant with our model.
In the same way, of the 829 \emph{isSaidBy} relations, that link speech balloons to speaking characters, 9.5\% were undetectable because they were generated from balloons outside the panel.


% section semantic_links_evaluation (end)

% subsection independent_information_extraction_evaluation (end)
\section{Knowledge-driven analysis evaluation} % (fold) %IJDAR
\label{sub:ex:knowledge_driven_analysis_evaluation}


%-------------------------------------------------------------------------
% \subsection{Framework evaluation}
We evaluated our framework during the two iterations of the process loop introduced in Chapter~\ref{chap:knowledge} and particularly at the beginning (Step 1: hypothesis) and at the end (Step 3: inference) of each iteration.
Note that using this framework some processes ($C$, $STSB$, $SBCS$) were related to previous processes which propagated errors and reduced their performance compared to their separated evaluation in Section~\ref{sub:ex:independent_information_extraction_evaluation}.

We evaluate our framework using the f-measure of panel $P$, balloon $B$, text $T$ and character $C$ extractions.
Also, the accuracy of the two semantic links $STSB$ and $SBSC$ is measured.
%  are all automatic here with their respective error rates. panel $P$, balloon $B$ and text $T$ extraction are similar to the 
The change in the amount of information discovered throughout the process is presented in Figure~\ref{fig:ex:score_evolution}.% were we show the % are evaluated using .

\modif{TODO ADD TABLE WITH NUMBERS CORREPONDING TO RADAR CHARTS}


% \begin{figure}[ht!]
% \centering
% \begin{subfigure}[b]{115px}
% \includegraphics[width=115px]{fig/radar_a.png}
% \caption{Hypothesis}
% \label{fig:ex:score_evolution_a}
% \end{subfigure}
% \begin{subfigure}[b]{115px}
% \includegraphics[width=115px]{fig/radar_b.png}
% \caption{Validation \& inference}
% \label{fig:ex:score_evolution_b}
% \end{subfigure}
% \begin{subfigure}[b]{115px}
% \includegraphics[width=115px]{fig/radar_c.png}
% \caption{Hypothesis}
% \label{fig:ex:score_evolution_c}
% \end{subfigure}
% \begin{subfigure}[b]{115px}
% \includegraphics[width=115px]{fig/radar_d.png}
% \caption{Validation \& inference}
% \label{fig:ex:score_evolution_d}
% \end{subfigure}
% \caption{Performance evolution for panels $P$, balloons $B$, text lines $T$, comic characters $C$ and the semantic links $STSB$ and $SBSC$ at the hypothesis and evaluation steps during the two loops of the proposed process.
%   The dashed blue line represents the best score using the proposed model on the data extracted from the ground truth data (optimal condition).
%   The solid red line is the performance of the framework using all the automatic extractions and semantic link inferences presented in this paper.}

%   \label{fig:ex:score_evolution}
% \end{figure}


%%%%%%%%%%%%%%%%%%%%%%%%%%%%%%%%%%%%%%%%%%%%%%%%%%%
\begin{figure}[ht]  %trim=l b r t  width=0.5\textwidth,
  \centering
 \includegraphics[width=230px]{radar_chart.png}
 \caption[Knowledge-driven extraction performance evolution]{Change in performance for panels $P$, balloons $B$, text lines $T$, comic characters $C$ and the semantic links $STSB$ and $SBSC$ at the hypothesis and evaluation steps during the two loops of the process.
 The dashed blue line represents the best score using our model on the data extracted from the ground truth data (optimal condition).
 The solid red line is the performance of the framework using all the automatic extractions and semantic link inferences presented in this paper.}
 \label{fig:ex:score_evolution}
\end{figure}
%%%%%%%%%%%%%%%%%%%%%%%%%%%%%%%%%%%%%%%%%%%%%%%%%%%

%TODO: add tail results figure + legend + analysis
\paragraph{Process details} % (fold)
\label{par:process_details}

% paragraph process_details (end)
Figure~\ref{fig:ex:score_evolution} shows the change of the performance of our framework after the first and second iteration of analysis over the eBDtheque dataset~\cite{Guerin2013}.
The performance of the first iteration was measured after the initial extraction of simple elements which were considered as hypotheses by the expert system (Figure~\ref{fig:ex:score_evolution}a.) and after validation by the expert system (Figure~\ref{fig:ex:score_evolution}b.).
Between the first initialization and validation (first row of Figure~\ref{fig:ex:score_evolution}), the f-measure remained stable for $P$ and increased by 3\% for the balloon $B$ and text $T$.
% which reflects the first contribution of the expert system over the low level processing.
%In fact, the text extraction has a 49\% precision (see table~\ref{tab:text_results}) which is raised up to 72\% by using the rule-based contextual filtering but this has also a cost for the recall which is dropped by 23\%. 
% into more specific elements such as speech balloon $SB$ and speech text $ST$ in order to provide additional information to the second iteration.
At this point, no comic book characters were discovered because their rules in the ontology are related to elements that were not yet discovered.
Nevertheless, the semantic links between speech text and balloon $STSB$ were inferred (47.6\%).
The expert system applied the inference rules to automatically label the balloon with a tail that included text such as speech balloon $SB$ and speech text $ST$ respectively and created a semantic link $STSB$ between each of them.

The newly inferred links were processed by the expert system along with previously validated regions during a second iteration in order to get more information by trying to apply more rules from the knowledge base.
This time, the expert system could make use of rules related to characters because the speech balloons were now part of the knowledge base.
Panels and speech balloons were used to create hypotheses for the location of characters and then the low level processing located them more precisely within these regions (Figure~\ref{fig:ex:score_evolution}c.).
Finally, the expert system validated the newly discovered regions and inferred the semantic links between speech balloons and speaking characters $SBSC$ (Figure~\ref{fig:ex:score_evolution}d.).

% \modifc{ADD Clement: results analysis to explain (and illustrate with examples?) why the results on data from the ground truth are not perfect.}

% In figure~\ref{fig:ex:score_evolution}, the dashed blue line represents how fit the model to the dataset according to section~\ref{sub:knowledge_base} and therefore sets the maximum reachable score for each of the elements using the presented framework on this data.

\paragraph{Low level processing} % (fold)
\label{par:low_level_processing}

% paragraph low_level_processing (end)
The low level processing scores ($P$, $B$, $T$) always increased between the hypothesis and the validation steps, which confirms the benefits of combining different levels of analysis.
The best extraction performance was obtained for the panels that were usually the easiest elements to extract from a page.
% because they overlap most of the content.
The lowest extraction performance was for the comic characters $C$.
% First, the non-extraction of comic characters that do not speak, second the difficulty of locating precisely comics characters in all the different comics styles that compose the eBDtheque dataset 
There are various reasons for this.
First, the limitation of the extractor to process speaking characters; second, the variability of character styles in the eBDtheque dataset; third, the error propagation of previous processes (panel, text, balloon, tail and semantic link extractions) that are required to guess comic character locations.

%, they represent more than half of them.

\paragraph{Semantic links} % (fold)
\label{sec:semantic_links_evaluation}

% We evaluated the ability of our system to produce the semantic relations introduced in \ref{sub:knowledge_representation}, namely \textit{STSB} and \textit{SBSC}, based on the correctly extracted balloons (Section~\ref{sub:balloon_evaluation}) and  text lines (see section~\ref{sub:text_localisation_recognition_evaluation}).
The expert system was able to retrieve 47.6\% of the $STSB$ and 18\% of the $SBSC$ relations, which represents more than 25\% of what could possibly be detected using the proposed model.

It should be stressed that these numbers represent the efficiency of the last process of the whole framework pipeline.
Individual errors at each recognition and validation step of the pipeline are propagated to the final semantic association between elements ($SBSC$).
Therefore a single improvement in the detection or the validation of any kind of element would have an impact at the semantic association level.


\paragraph{General evaluation} % (fold)
\label{par:global_evaluation}
It is difficult to combine individual metrics of a different nature into a single global metric.
As an indicative metric, we considered the hexagon area as being the amount of information that could be retrieved from the eBDtheque dataset~\cite{Guerin2013}.
Our model is theoretically able to model 88.83\% of this information (dashed line), despite the diversity of the images in this dataset, and 54.89\% of this information can be retrieved automatically (solid line) using our framework.




% subsection knowledge_driven_analysis_evaluation (end)

% \section{Global evaluation} % (fold)
% \label{sub:ex:global_evaluation}

% TODO ???

%big table to compare the 3 level of contribution of this thesis

      %                 | sequential|Independent|Knowledge-driven |
      %Panel            | R | P | F | R | P | F |  R  |  P  |  F  |
      %Text             |
      %Balloon          |
      %Tail                   ? 
      %Comic character
      %Semantic link STSB                 X             
      %Semantic link SBSC                 X

% subsection global_evaluation (end)
% section evaluation (end)


% Conclusion --------------------------------------------------------------------------------------------------------------------------------------
\section{Conclusions}
\label{sub:ex:conclusion}

\modif{TODO}
