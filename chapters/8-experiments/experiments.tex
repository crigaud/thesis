\chapter{Experimental Evaluation}
\chaptermark{Experiments}
\label{chap:experiments}
\graphicspath{{./chapters/8-experiments/figs/}}
The main purpose of this chapter is to provide an overall experimental evaluation and comparison of all the proposed methodologies and some of the state-of-the-art methods proposed in the literature. Among the state-of-the-art methods we have considered those which work with graph based techniques and apply their algorithms for symbol spotting in graphical documents. To have a unified experimental framework we configure a symbol spotting system on graphical documents especially on architectural floorplans. The evaluation of a particular method is basically done depending on the capability of spotting symbols on the graphical document.
\section{Introduction}
\label{sec:experiments:intro}
In the previous four chapters we have provided individual experiments corresponding to the different contributions of the thesis. In this chapter we provide an integrated evaluation scheme. Experimental evaluation is an important step to judge the behaviour of algorithms. An experimental study should reveal the strengths and weaknesses of the methods under test. The analysis of these strong points and drawbacks should determine which method is the most suitable for a certain use case and predict its behaviour when using it in applications with real data. In this chapter, we experimentally evaluate the proposed and some of the state-of-the-art algorithms and provide an overall comparisons among them.

Before going to the results part, in the next section we have provided a brief description of the state-of-the-art algorithms on symbol spotting considered for the comparison. For referencing the algorithms we use some abbreviations of the names which are listed in Table~\ref{tab:experiments:references}.

\begin{table}[h!]
\centering
\caption{Summary, abbreviations of the methods}
\begin{tabular}{r|m{9.5cm}}
\toprule
\hline
\textbf{Abbreviation} & \textbf{Method}\\
\hline
SG & Symbol spotting by hashing of serialized subgraphs (\ch{chap:hssg}).\\
PG & Product graph based subgraph matching (\ch{chap:pg}).\\
NCRAG & Near convex region adjacency graph (\ch{chap:ncrag}).\\
HGR & Hierarchical graph representation (\ch{chap:hgr}).\\
SSGR & Symbol spotting using graph representations~\cite{Qureshi2007}\\
FGE & Subgraph spotting through fuzzy graph embedding~\cite{Luqman2011}.\\
ILPIso & Integer linear program for substitution tolerant subgraph isomorphism~\cite{LeBodic2012}.\\
\hline
\end{tabular}
\label{tab:experiments:references}
\end{table}

The rest of the chapter is divided into four sections, where \sect{sec:experiments:desc-sota} is dedicated for describing in brief the state-of-the-art methods that we have considered for experimental evaluation and comparison. The experimentation is described in \sect{sec:experiments:res}. \sect{sec:experiments:disc} contains discussions on the obtained results by various methods. After that in \sect{sec:experiments:concl} the chapter is concluded.

\section{Description of state-of-the-art methods}
\label{sec:experiments:desc-sota}
This section contains brief descriptions of the three state-of-the-art methods:

\textbf{Symbol spotting using graph representations~\cite{Qureshi2007}:} The algorithm was proposed by Qureshi~\etal~\cite{Qureshi2007}. The proposed strategy has two main steps. In the first step, a graph based representation of a document image is generated that includes selection of description primitives (nodes of the graph) and relation of these features (edges) (see \fig{fig:experiments:ssgr}). In the second step the graph is used to spot interesting parts of the image that potentially correspond to symbols. The sub-graphs associated to selected zones are then submitted to a graph matching algorithm in order to take the final decision and to recognize the class of the symbol. The experimental results obtained on different types of documents demonstrates that the system can handle different types of images without any modification.
\begin{figure}[t!]
\begin{center}
\subfloat[]{\includegraphics[width=0.45\textwidth]{qureshi-a}}
\hspace{5mm}
\subfloat[]{\includegraphics[width=0.45\textwidth]{qureshi-b}}\\
\subfloat[]{\includegraphics[width=0.25\textwidth]{qureshi-c}}
\hspace{2mm}
\subfloat[]{\includegraphics[width=0.4\textwidth]{qureshi-d}}
\hspace{2mm}
\subfloat[]{\includegraphics[width=0.3\textwidth]{qureshi-e}}\\
\subfloat[]{\includegraphics[width=0.8\textwidth]{qureshi-f}}
\caption{(a) Initial image, (b) Vectorization results, (c) Zone of influence of a quadrilateral, (d) Influence zone of the quadrilaterals and their corresponding sub-graphs respectively, (e) and (f) Graph representation. (Figure credit: Qureshi~\etal~\cite{Qureshi2007}).}
\label{fig:experiments:ssgr}
\end{center}
\end{figure}

%\textbf{Rusi{\~n}ol \etal \cite{Rusinol2010}} In this work the authors present a spotting method based on vector primitives. Graphical symbols are represented by a set of vectorial primitives which are described by an off-the-shelf shape descriptor. A relational indexing strategy aims to retrieve symbol locations into the target documents by using a combined numerical-relational description of 2D structures. The zones which are likely to contain the queried symbol are validated by a Hough-like voting scheme.

\textbf{Subgraph spotting through fuzzy graph embedding~\cite{Luqman2011}:} Here the author present a method for spotting subgraph in a graph repository. Their proposed method accomplishes subgraph spotting through graph embedding. They have done an automatic indexation of a graph repository during an off-line learning phase; where they (i) split the graphs into 2-node subgraphs, which are primitive building-blocks of a graph, (ii) embed the 2-node subgraphs into feature vectors by employing proposed explicit graph embedding technique, (iii) cluster the feature vectors in classes by employing a classic agglomerative clustering technique, (iv) build an index for the graph repository and (v) learn a Bayesian network classifier. The subgraph spotting is achieved during the on-line querying phase; where they (i) further split the query graph into 2-node subgraphs, (ii) embed them into feature vectors, (iii) employ the Bayesian network classifier for classifying the query 2-node subgraphs and (iv) retrieve the respective graphs by looking-up in the index of the graph repository. The graphs containing all query 2-node subgraphs form the set of result graphs for the query. Finally, they employ the adjacency matrix of each resultant graph along with a score function, for spotting the query graph in it. The proposed subgraph spotting method is equally applicable to a wide range of domains; offering ease of query by example (QBE) and granularity of focused retrieval.

\textbf{Integer linear program for substitution tolerant subgraph isomorphism~\cite{LeBodic2012}:} This paper tackles the problem of substitution-tolerant subgraph isomorphism which is a specific class of error-tolerant isomorphism. This problem aims at finding a subgraph isomorphism of a pattern graph $S$ in a target graph $G$. This isomorphism only considers label substitutions and forbids vertex and edge insertion in $G$. This kind of subgraph isomorphism is often needed in pattern recognition problems when graphs are attributed with real values and no exact matching can be found between attributes due to noise.

The proposal to solve the problem of substitution-tolerant subgraph isomorphism relies on its formulation in the Integer Linear Program (ILP) formalism. Using a general ILP solver, the approach is able to find, if one exists, a mapping of a pattern graph in to a target graph such that the topology of the searched graph is kept and the editing operations between the label shave a minimal cost. The proposed subgraph matching has been applied for spotting symbols in graphical documents where document and symbol images are represented by vector-attributed Region Adjacency Graphs built from a segmentation process.
\begin{figure}[h!]
\begin{center}
\includegraphics[width=0.25\textwidth]{lebodic}
\caption{An example of matching. $\mathcal{S}$ and $\mathcal{G}$ both contain a single edge, respectively $ij$ and $kl$. The following solution is represented on this figure: $x_{i,k} = 1$ (resp. $x_{j,l} = 1$, $y_{ij,kl} = 1$), \ie~$i$ (resp. $j$, $ij$) is matched with $k$ (resp. $l$, $kl$). Conversely, since $i$ (resp. $j$) is not matched with $l$ (resp. $k$), $x_{i,l} = 0$ (resp. $x_{j,k} = 0$). (Figure credit: Le Bodice~\etal~\cite{LeBodic2012}).}
\end{center}
\label{fig:experiments:ilpiso}
\end{figure}

\section{Experimental results}
\label{sec:experiments:res}
For the experiments we choose two different subsets from the SESYD (floorplans) dataset, viz. \emph{floorplans16-05} and \emph{floorplans16-06}. One can find a details on SESYD (floorplans) in \sect{sec:datasets:sesyd}. These two particular subsets have been chosen keeping in mind the execution ability of different methods. This is because some of the graph based methods use special kind of vectorization algorithms which can not handle all kind of graphical structures such as thick walls etc.

Since in each of the individual chapters a detailed experimentations have already been documented for each of the proposed methods with different parameter settings, in this chapter we only mention the best results obtained by a particular method (by a particular parameter settings). This implies the best results from each of the method is considered for the experimental comparison. The results obtained by the methods SSGR~\cite{Qureshi2007} and FGE~\cite{Luqman2011} are directly taken from the paper~\cite{Luqman2011}, where the authors had performed a comparison. And for the ILPIso method proposed by Le Bodic~\etal, the implementation was downloaded from the project web page\footnote{\url{http://litis-ilpiso.univ-rouen.fr/ILPIso}}.

Three of the proposed methods viz. SG, PG, NCRAG and ILPIso are designed to provide a ranked list of retrievals. The HGR method does not provide a ranked list in this way. This is because this method works with a node attributes which take into account the angles and does not reflect similarity of shapes (see \ch{chap:hgr}). To have an idea how a particular method can rank the true positives with respect to the false positives, we draw the receiver operating characteristic (ROC) curves obtained from the ranked list of retrievals of different symbols (see \fig{fig:experiments:roc}). The ROC curves usually reveal how well a method can rank the true positives before the false positives. These curves are basically a test of the similarity/dissimilarity functions designed inside each of the algorithms and supposed to provide a similarity/dissimilarity measures for a retrieval.

\begin{figure}[!bh]
\begin{center}
\subfloat[\emph{bed}]{\includegraphics[width=0.32\textwidth]{roc_bed}}
\hspace{0.1cm}
\subfloat[\emph{door1}]{\includegraphics[width=0.32\textwidth]{roc_door1}}
\hspace{0.1cm}
\subfloat[\emph{door2}]{\includegraphics[width=0.32\textwidth]{roc_door2}}\\
\subfloat[\emph{sink1}]{\includegraphics[width=0.32\textwidth]{roc_sink1}}
\hspace{0.1cm}
\subfloat[\emph{sink2}]{\includegraphics[width=0.32\textwidth]{roc_sink2}}
\hspace{0.1cm}
\subfloat[\emph{sink3}]{\includegraphics[width=0.32\textwidth]{roc_sink3}}\\
\subfloat[\emph{sink4}]{\includegraphics[width=0.32\textwidth]{roc_sink4}}
\hspace{0.1cm}
\subfloat[\emph{sofa1}]{\includegraphics[width=0.32\textwidth]{roc_sofa1}}
\hspace{0.1cm}
\subfloat[\emph{sofa2}]{\includegraphics[width=0.32\textwidth]{roc_sofa2}}
\end{center}
\end{figure}
\begin{figure}[!th]
\begin{center}
\subfloat[\emph{table1}]{\includegraphics[width=0.32\textwidth]{roc_table1}}
\hspace{0.1cm}
\subfloat[\emph{table2}]{\includegraphics[width=0.32\textwidth]{roc_table2}}
\hspace{0.1cm}
\subfloat[\emph{table3}]{\includegraphics[width=0.32\textwidth]{roc_table3}}\\
\subfloat[\emph{tub}]{\includegraphics[width=0.32\textwidth]{roc_tub}}
\hspace{0.1cm}
\subfloat[\emph{window1}]{\includegraphics[width=0.32\textwidth]{roc_window1}}
\hspace{0.1cm}
\subfloat[\emph{window2}]{\includegraphics[width=0.32\textwidth]{roc_window2}}
\end{center}
\caption{Receiver operating characteristic (ROC) curves for different pattern graphs obtained by the method based on hashing of serialized graphs.}
\label{fig:experiments:roc}
\end{figure}

The quantitative results obtained by different methods are listed in the Table~\ref{tab:experiments:results}. The performance evaluation protocol followed in this experimentation is explained in \app{app:perf-eval}. For having an idea on the time complexity of each of the methods we have also provided a mean time measurement (\textbf{T}) for spotting instances of a query symbol in a single target document. Here it is to be mentioned that the mentioned time duration only includes the time taken in the online phase. The time taken for the necessary feature extraction, preprocessing, construction of graphs etc is not considered in this study as they are done in offline phase. One can consider the F-measure value for a very high level overview of the performance of the methods. But we believe that deciding the winner or looser is not the only aim of an experimental study. There are separate advantages and disadvantages of each methods.

\begin{table}[h!]
\centering
\caption{Results}
\begin{tabular}{cccccc}
\toprule
\hline
\textbf{Methods} & \textbf{P} & \textbf{R} & \textbf{F} & \textbf{AveP} & \textbf{T}\\
\hline
SG (\ch{chap:hssg}) & 54.19 & 83.98 & 65.87 & 65.43 & \textbf{0.07s}\\
PG (\ch{chap:pg}) & \textbf{70.56} & 86.29 & \textbf{80.10} & \textbf{82.98} & 33.37s\\
NCRAG (\ch{chap:ncrag}) & 61.89 & 82.87 & 70.85 & 70.65 & 0.72s\\
HGR (\ch{chap:hgr}) & 30.11 & 33.76 & 31.83 & - & 48m24s\\ \hline
SSGR~\cite{Qureshi2007}& 41.00 & 80.00 & 54.21 & 64.45 & -\\
%Rusi{\~n}ol~\etal~\cite{Rusinol2010} & 47.89 & 90.00 & & 64.51 & - \\
FGE~\cite{Luqman2011}& 56.00 & \textbf{100.00} & 71.79 & 75.50 & -\\
ILPIso~\cite{LeBodic2012} & 65.44 & 58.23 & 59.11 & 57.75 & 27.29s\\
\hline
\end{tabular}
\label{tab:experiments:results}
\end{table}

\section{Discussions}
\label{sec:experiments:disc}
%TODO SG
The SG method performs quite well for symbols with complex structures such as \emph{bed}, \emph{door2}, \emph{sink1}, \emph{sink2}, \emph{sink4}, \emph{table2}, \emph{table3} etc. This is quite justified since the complex curved parts provide enough discrimination to a particular class. As we will observe in the next part, that this phenomenon is quite common for the other methods too. The SG method returns false positives while the query symbol contains a subpart of the other symbol. For example, it retrieves false \emph{door1} because \emph{door1} also occurs inside \emph{door2} and it detects part of \emph{door2}. For the same reason it retrieves false \emph{sink3} as it belongs as a subpart in \emph{sink2}. This problem is also mentioned in \ch{chap:hssg} and this is because the paths are indexed independently and there is no higher level organization of the serialized structures. This method also performs worse for very simple and frequently occurring structure such as \emph{sofa1}, \emph{sofa2}, \emph{table1}, \emph{tub} and \emph{window1}. One of the advantages of this method is the execution time in the online phase which is quite less and can be considered as a benefit of the indexation technique. However, the indexation phase in the offline stage usually takes nearly two hours to create the hash table for this particular dataset.

%TODO PG
The overall results obtained by the PG method is quite good. It has achieved the highest precision, F-measure and average precision values. A problem of this method results in from the computation of the bounding box to decide the position of the occurrence of an instance which is presently done by grouping the adjacent dual nodes. This way occurrence of a false dual node often creates bigger bounding box, as shown in \fig{sfig:experiments:door1-err} (the red bounding box) while spotting \emph{door1}. In our performance evaluation bigger bounding boxes are classified as false positives, this explains the bad results for \emph{door1}. This method also performs worse in case of \emph{sofa1} but this is due to the occurrence of similar structure in different parts other than the actual occurrences as shown in \fig{sfig:experiments:sofa1-err}.
\begin{figure}[h!]
\centering
\subfloat[\emph{door1}: erroneous results obtained by PG method]{\label{sfig:experiments:door1-err}\includegraphics[width=0.4\textwidth]{door1_erroneous_PG}}
\hspace{1cm}
\subfloat[\emph{sofa1}: erroneous results obtained by PG method]{\label{sfig:experiments:sofa1-err}\includegraphics[width=0.4\textwidth]{sofa1_erroneous_PG}}
\caption{Erroneous results.}
\label{fig:experiments:erroneous-results}
\end{figure}

%TODO NCRAG
The results obtained by the NCRAG method is also good. Even the method worked very well for the difficult symbol \emph{sofa1}. It is observed that this method is bit sensitive to the selection of the key regions. A region, which is adjacent to the most of the other regions in a symbol, can be a good candidate for a key region. Otherwise, a wrong expansion often results in with a lower cost. This problem is observed for the symbol \emph{sink4} and \emph{table3}. A similar problem also occurs for symmetric symbol such as \emph{table1} where a finding a discriminating region is difficult. The issues concerning the variation of the regions of the pattern and target graphs have been reported in \ch{chap:ncrag}, have not been observed in this dataset.

%TODO HGR
The HGR method worked only with six query symbols and they are \emph{bed}, \emph{door1}, \emph{door2}, \emph{sofa1}, \emph{sofa2} and \emph{table1}. The reason of failure for the other symbols might be due to the node attributes which are not stable in many scenario and also getting robust node attributes for this kind of unlabelled graphs is not easy. But for the successful symbols the method works quite good, the false retrievals are substantially less for this method. We can not provide a ranked list of retrieved symbols because in this case obtaining a similarity value for each of the retrievals is not straight forward for the nature of the node attributes.

We are not able to comment on the detailed results obtained by the method SSGR and FGE as the results are taken from the paper for quantitative comparisons. The ILPIso method proposed by Le Bodic~\etal~ performed quite well with most of the scenario, as it obtained $100\%$ F-measure for seven pattern graphs (\emph{sink1}, \emph{sink2}, \emph{sink3}, \emph{sofa2}, \emph{tub}, \emph{window1} and \emph{window2}). As the other methods there is a usual problem with the occurrence of the same symbol as part of the other such as \emph{sofa1} occurs in \emph{table2}. Apart from that, the method can not provide any true retrievals for \emph{door1}, \emph{door2} etc. This is because of the kind graph representation used by the method, as region adjacency graph can not provide robust representation for these two symbols due to the existence of open regions. The method does not find any true instances for \emph{sink4}, \emph{table3}, may be this is because of the discrepancy of the regions in pattern and target graphs as mentioned for NCRAG method (\ch{chap:ncrag}). The method does not finish the search procedure with \emph{table2} and the search procedure has to be aborted manually after 60 minutes.

We have also provided some of the qualitative results for all the five methods, which are shown from \fig{sfig:experiments:1} to \fig{sfig:experiments:5}.

\begin{sidewaysfigure}[thp]
\begin{center}
\subfloat[\emph{SG}]{\includegraphics[width=0.18\textwidth,height=0.15\textwidth]{bed_SG}}
\hspace{0.1cm}
\subfloat[\emph{NCRAG}]{\includegraphics[width=0.18\textwidth,height=0.15\textwidth]{bed_NCRAG}}
\hspace{0.1cm}
\subfloat[\emph{HGR}]{\includegraphics[width=0.18\textwidth,height=0.15\textwidth]{bed_HGR}}
\hspace{0.1cm}
\subfloat[\emph{PG}]{\includegraphics[width=0.18\textwidth,height=0.15\textwidth]{bed_PG}}
\hspace{0.1cm}
\subfloat[\emph{ILPIso}]{\includegraphics[width=0.18\textwidth,height=0.15\textwidth]{bed_ILPIso}}\\
\subfloat[\emph{SG}]{\includegraphics[width=0.18\textwidth,height=0.18\textwidth]{door1_SG}}
\hspace{0.1cm}
\subfloat[\emph{NCRAG}]{\includegraphics[width=0.18\textwidth,height=0.18\textwidth]{door1_NCRAG}}
\hspace{0.1cm}
\subfloat[\emph{HGR}]{\includegraphics[width=0.18\textwidth,height=0.18\textwidth]{door1_HGR}}
\hspace{0.1cm}
\subfloat[\emph{PG}]{\includegraphics[width=0.18\textwidth,height=0.18\textwidth]{door1_PG}}
\hspace{0.1cm}
\subfloat[\emph{ILPIso}]{\includegraphics[width=0.18\textwidth,height=0.18\textwidth]{door1_ILPIso}}\\
\subfloat[\emph{SG}]{\includegraphics[width=0.18\textwidth,height=0.15\textwidth]{door2_SG}}
\hspace{0.1cm}
\subfloat[\emph{NCRAG}]{\includegraphics[width=0.18\textwidth,height=0.15\textwidth]{door2_NCRAG}}
\hspace{0.1cm}
\subfloat[\emph{HGR}]{\includegraphics[width=0.18\textwidth,height=0.15\textwidth]{door2_HGR}}
\hspace{0.1cm}
\subfloat[\emph{PG}]{\includegraphics[width=0.18\textwidth,height=0.15\textwidth]{door2_PG}}
\hspace{0.1cm}
\subfloat[\emph{ILPIso}]{\includegraphics[width=0.18\textwidth,height=0.15\textwidth]{door2_ILPIso}}
\caption{Qualitative results: (a)-(e) \emph{bed}, (f)-(j) \emph{door1} and (k)-(o) \emph{door2}.}
\label{sfig:experiments:1}
\end{center}
\end{sidewaysfigure}

\begin{sidewaysfigure}[thp]
\begin{center}
\subfloat[\emph{SG}]{\includegraphics[width=0.18\textwidth,height=0.15\textwidth]{sink1_SG}}
\hspace{0.1cm}
\subfloat[\emph{NCRAG}]{\includegraphics[width=0.18\textwidth,height=0.15\textwidth]{sink1_NCRAG}}
\hspace{0.1cm}
\subfloat[\emph{HGR}]{\includegraphics[width=0.18\textwidth,height=0.15\textwidth]{sink1_HGR}}
\hspace{0.1cm}
\subfloat[\emph{PG}]{\includegraphics[width=0.18\textwidth,height=0.15\textwidth]{sink1_PG}}
\hspace{0.1cm}
\subfloat[\emph{ILPIso}]{\includegraphics[width=0.18\textwidth,height=0.15\textwidth]{sink1_ILPIso}}\\
\subfloat[\emph{SG}]{\includegraphics[width=0.18\textwidth,height=0.18\textwidth]{sink2_SG}}
\hspace{0.1cm}
\subfloat[\emph{NCRAG}]{\includegraphics[width=0.18\textwidth,height=0.18\textwidth]{sink2_NCRAG}}
\hspace{0.1cm}
\subfloat[\emph{HGR}]{\includegraphics[width=0.18\textwidth,height=0.18\textwidth]{sink2_HGR}}
\hspace{0.1cm}
\subfloat[\emph{PG}]{\includegraphics[width=0.18\textwidth,height=0.18\textwidth]{sink2_PG}}
\hspace{0.1cm}
\subfloat[\emph{ILPIso}]{\includegraphics[width=0.18\textwidth,height=0.18\textwidth]{sink2_ILPIso}}\\
\subfloat[\emph{SG}]{\includegraphics[width=0.18\textwidth,height=0.15\textwidth]{sink3_SG}}
\hspace{0.1cm}
\subfloat[\emph{NCRAG}]{\includegraphics[width=0.18\textwidth,height=0.15\textwidth]{sink3_NCRAG}}
\hspace{0.1cm}
\subfloat[\emph{HGR}]{\includegraphics[width=0.18\textwidth,height=0.15\textwidth]{sink3_HGR}}
\hspace{0.1cm}
\subfloat[\emph{PG}]{\includegraphics[width=0.18\textwidth,height=0.15\textwidth]{sink3_PG}}
\hspace{0.1cm}
\subfloat[\emph{ILPIso}]{\includegraphics[width=0.18\textwidth,height=0.15\textwidth]{sink3_ILPIso}}
\caption{Qualitative results: (a)-(e) \emph{sink1}, (f)-(j) \emph{sink2} and (k)-(o) \emph{sink3}.}
\label{sfig:experiments:2}
\end{center}
\end{sidewaysfigure}

\begin{sidewaysfigure}[thp]
\begin{center}
\subfloat[\emph{SG}]{\includegraphics[width=0.18\textwidth,height=0.15\textwidth]{sink4_SG}}
\hspace{0.1cm}
\subfloat[\emph{NCRAG}]{\includegraphics[width=0.18\textwidth,height=0.15\textwidth]{sink4_NCRAG}}
\hspace{0.1cm}
\subfloat[\emph{HGR}]{\includegraphics[width=0.18\textwidth,height=0.15\textwidth]{sink4_HGR}}
\hspace{0.1cm}
\subfloat[\emph{PG}]{\includegraphics[width=0.18\textwidth,height=0.15\textwidth]{sink4_PG}}
\hspace{0.1cm}
\subfloat[\emph{ILPIso}]{\includegraphics[width=0.18\textwidth,height=0.15\textwidth]{sink4_ILPIso}}\\
\subfloat[\emph{SG}]{\includegraphics[width=0.18\textwidth,height=0.15\textwidth]{sofa1_SG}}
\hspace{0.1cm}
\subfloat[\emph{NCRAG}]{\includegraphics[width=0.18\textwidth,height=0.15\textwidth]{sofa1_NCRAG}}
\hspace{0.1cm}
\subfloat[\emph{HGR}]{\includegraphics[width=0.18\textwidth,height=0.15\textwidth]{sofa1_HGR}}
\hspace{0.1cm}
\subfloat[\emph{PG}]{\includegraphics[width=0.18\textwidth,height=0.15\textwidth]{sofa1_PG}}
\hspace{0.1cm}
\subfloat[\emph{ILPIso}]{\includegraphics[width=0.18\textwidth,height=0.15\textwidth]{sofa1_ILPIso}}\\
\subfloat[\emph{SG}]{\includegraphics[width=0.18\textwidth,height=0.15\textwidth]{sofa2_SG}}
\hspace{0.1cm}
\subfloat[\emph{NCRAG}]{\includegraphics[width=0.18\textwidth,height=0.15\textwidth]{sofa2_NCRAG}}
\hspace{0.1cm}
\subfloat[\emph{HGR}]{\includegraphics[width=0.18\textwidth,height=0.15\textwidth]{sofa2_HGR}}
\hspace{0.1cm}
\subfloat[\emph{PG}]{\includegraphics[width=0.18\textwidth,height=0.15\textwidth]{sofa2_PG}}
\hspace{0.1cm}
\subfloat[\emph{ILPIso}]{\includegraphics[width=0.18\textwidth,height=0.15\textwidth]{sofa2_ILPIso}}
\caption{Qualitative results: (a)-(e) \emph{sink4}, (f)-(j) \emph{sofa1} and (k)-(o) \emph{sofa2}.}
\label{sfig:experiments:3}
\end{center}
\end{sidewaysfigure}

\begin{sidewaysfigure}[thp]
\begin{center}
\subfloat[\emph{SG}]{\includegraphics[width=0.18\textwidth,height=0.15\textwidth]{table1_SG}}
\hspace{0.1cm}
\subfloat[\emph{NCRAG}]{\includegraphics[width=0.18\textwidth,height=0.15\textwidth]{table1_NCRAG}}
\hspace{0.1cm}
\subfloat[\emph{HGR}]{\includegraphics[width=0.18\textwidth,height=0.15\textwidth]{table1_HGR}}
\hspace{0.1cm}
\subfloat[\emph{PG}]{\includegraphics[width=0.18\textwidth,height=0.15\textwidth]{table1_PG}}
\hspace{0.1cm}
\subfloat[\emph{ILPIso}]{\includegraphics[width=0.18\textwidth,height=0.15\textwidth]{table1_ILPIso}}\\
\subfloat[\emph{SG}]{\includegraphics[width=0.18\textwidth,height=0.15\textwidth]{table2_SG}}
\hspace{0.1cm}
\subfloat[\emph{NCRAG}]{\includegraphics[width=0.18\textwidth,height=0.15\textwidth]{table2_NCRAG}}
\hspace{0.1cm}
\subfloat[\emph{HGR}]{\includegraphics[width=0.18\textwidth,height=0.15\textwidth]{table2_HGR}}
\hspace{0.1cm}
\subfloat[\emph{PG}]{\includegraphics[width=0.18\textwidth,height=0.15\textwidth]{table2_PG}}
\hspace{0.1cm}
\subfloat[\emph{ILPIso}]{\includegraphics[width=0.18\textwidth,height=0.15\textwidth]{table2_ILPIso}}\\
\subfloat[\emph{SG}]{\includegraphics[width=0.18\textwidth,height=0.18\textwidth]{table3_SG}}
\hspace{0.1cm}
\subfloat[\emph{NCRAG}]{\includegraphics[width=0.18\textwidth,height=0.18\textwidth]{table3_NCRAG}}
\hspace{0.1cm}
\subfloat[\emph{HGR}]{\includegraphics[width=0.18\textwidth,height=0.18\textwidth]{table3_HGR}}
\hspace{0.1cm}
\subfloat[\emph{PG}]{\includegraphics[width=0.18\textwidth,height=0.18\textwidth]{table3_PG}}
\hspace{0.1cm}
\subfloat[\emph{ILPIso}]{\includegraphics[width=0.18\textwidth,height=0.18\textwidth]{table3_ILPIso}}
\caption{Qualitative results: (a)-(e) \emph{table1}, (f)-(j) \emph{table2} and (k)-(o) \emph{table3}.}
\label{sfig:experiments:4}
\end{center}
\end{sidewaysfigure}

\begin{sidewaysfigure}[th]
\begin{center}
\subfloat[\emph{SG}]{\includegraphics[width=0.18\textwidth,height=0.15\textwidth]{tub_SG}}
\hspace{0.1cm}
\subfloat[\emph{NCRAG}]{\includegraphics[width=0.18\textwidth,height=0.15\textwidth]{tub_NCRAG}}
\hspace{0.1cm}
\subfloat[\emph{HGR}]{\includegraphics[width=0.18\textwidth,height=0.15\textwidth]{tub_HGR}}
\hspace{0.1cm}
\subfloat[\emph{PG}]{\includegraphics[width=0.18\textwidth,height=0.15\textwidth]{tub_PG}}
\hspace{0.1cm}
\subfloat[\emph{ILPIso}]{\includegraphics[width=0.18\textwidth,height=0.15\textwidth]{tub_ILPIso}}\\
\subfloat[\emph{SG}]{\includegraphics[width=0.18\textwidth,height=0.15\textwidth]{window1_SG}}
\hspace{0.1cm}
\subfloat[\emph{NCRAG}]{\includegraphics[width=0.18\textwidth,height=0.15\textwidth]{window1_NCRAG}}
\hspace{0.1cm}
\subfloat[\emph{HGR}]{\includegraphics[width=0.18\textwidth,height=0.15\textwidth]{window1_HGR}}
\hspace{0.1cm}
\subfloat[\emph{PG}]{\includegraphics[width=0.18\textwidth,height=0.15\textwidth]{window1_PG}}
\hspace{0.1cm}
\subfloat[\emph{ILPIso}]{\includegraphics[width=0.18\textwidth,height=0.15\textwidth]{window1_ILPIso}}\\
\subfloat[\emph{SG}]{\includegraphics[width=0.18\textwidth,height=0.15\textwidth]{window2_SG}}
\hspace{0.1cm}
\subfloat[\emph{NCRAG}]{\includegraphics[width=0.18\textwidth,height=0.15\textwidth]{window2_NCRAG}}
\hspace{0.1cm}
\subfloat[\emph{HGR}]{\includegraphics[width=0.18\textwidth,height=0.15\textwidth]{window2_HGR}}
\hspace{0.1cm}
\subfloat[\emph{PG}]{\includegraphics[width=0.18\textwidth,height=0.15\textwidth]{window2_PG}}
\hspace{0.1cm}
\subfloat[\emph{ILPIso}]{\includegraphics[width=0.18\textwidth,height=0.15\textwidth]{window2_ILPIso}}
\caption{Qualitative results: (a)-(e) \emph{tub}, (f)-(j) \emph{window1} and (k)-(o) \emph{window2}.}
\label{sfig:experiments:5}
\end{center}
\end{sidewaysfigure}

\section{Conclusions}
\label{sec:experiments:concl}
In this chapter we have provided an overall experimental evaluation of all the proposed methods and some of the state-of-the-art methods. We have tried to figure out the advantages and disadvantages of different methods. The discussions on different methods can reveal in which scenario which kind of methodology fits better. There is not any single method which can resolve all the problems. This fact indicates the need of certain future work for different methodologies, at the same time, it points out a direction to investigate on combining more than one symbol spotting systems.