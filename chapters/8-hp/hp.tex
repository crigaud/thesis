\chapter{System for document understanding} % 20 pages (see Clement's work)
\chaptermark{System for document understanding}
\label{chap:hp}
\graphicspath{{./chapters/8-hp/figs/}}

% Abstract------------------------------------------------------------------
In literature, there are methods that formulate...

\section{Introduction}
\label{sec:hp:intro}


%------------------------------------------------------------------------------------------------------------------------------------------------------------------------------------------------------------
\section{Methodology}
\label{sec:hp:method}


\subsection{Framework}
\label{sub:hp:rw}


\subsection{Application to comics} % (fold)
\label{sub:hp:application_to_comics}


\begin{equation}
SW_X=\lim_{n\rightarrow \infty}\sum_{k=0}^n \lambda^k W_{X_k}
\label{eqn:pg:rw1}
\end{equation}

where
\begin{equation}
W_{X_k} = W_X^k
\end{equation}

Here $\lambda$ is a weighting factor to discount the longer walks, as they often contain redundant or repeated information. In this chapter we always choose $\lambda=\frac{1}{a}$, where $a=\min(\Delta^+(W_X),\Delta^-(W_X))$. Here $\Delta^+(W_X)$, $\Delta^-(W_X)$ are respectively the maximum outward and inward degree of $W_X$~\cite{Gartner2003a}.

\section{Experimental results} % (fold)
\label{sec:hp:experimental_results}

% section experimental_results (end)

%------------------------------------------------------------------------------------------------------------------------------------------------------------------------------------------------------------
\section{Conclusions}
\label{sec:hp:conclusion}
