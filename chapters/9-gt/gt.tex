\chapter{Dataset and ground truth} %15 pages
\chaptermark{Dataset and ground truth}
\label{chap:gt}
\graphicspath{{./chapters/9-gt/figs/}}
% Abstract-----------------------------------------------------
In this chapter we propose and detail the construction of the first publicly available comics image dataset and ground truth.
The comics images were selected to cover the huge diversity of comic styles with the agreement of the consenting authors and publishers that had the objective to foster innovation in this domain trough academic research.
The ground truth is defined in accordance to existing formalism in order to fulfil the needs of a large amount of researchers related to comics material.It integrates low and high level information such as the spatial position of the elements in the image, their semantic links and also bibliographic information.

% -------------------------------------------------------------
\section{Introduction}
\label{sec:gt:intro}

% \begin{itemize}
% 	% \item Contact authors, publishers, they perception of the project
% 	\item Aim to represent the diversity of comics
% 	\item Add comic characters and semantic links
% 	\item Version 2013 and 2014
% \end{itemize}


With the analysis and processing of data comes the need of the output results evaluation.
Traditionally, this evaluation is made by validating the results of an algorithm with a ground truth that represents what an ideal output should be\cite{pascal-voc-2012, smeaton2006evaluation, griffinHolubPerona}.
Ideally, such a ground truth is made publicly available so anyone can challenge his own algorithm to the community \cite{lamiroy:inria-00537035}.
This can be applied to any kind of results from image segmentation to classification or information retrieval.
Being in need of comic books material and an associated ground truth to evaluate our work, we noticed that there is not such dataset publicly available for scientific purpose. 
Therefore, we decided to gather the first publicly available comic books dataset in association with several comic books authors and publishers and to build up the corresponding ground truth according to document analysis and understanding concerns.% which are image segmentation and semantic analysis.

It took almost one year to define which type of comics to use, meet and convince comics authors and publishers, get copyright authorizations for the scientific community, develop a specific annotation tool and finally to hire people to do the ground truth.

A selection of hundred comic pages were annotated in one day by twenty volunteers affiliated to the L3i lab.
In order to provide a common basis for evaluating research work, the ground truth have been published in 2013~\cite{Guerin2013} and made available to the scientific community via the project website~\footnote{http://ebdtheque.univ-lr.fr}.
It has been enhanced in 2014 by adding semantic information to the already annotated elements.

The content of the dataset, the ground truth construction protocol and its quality assessment are detailed in the next section.


% \section{Structure and indexation}
% \label{sec:gt:structure_indexation}
\section{Dataset description} % (fold)
\label{sec:dataset_description}

% section dataset_description (end)

Scott McCloud defined comics as ``juxtaposed pictorial and other images in deliberate sequence, intended to convey information and/or to produce an aesthetic response in the viewer''~\cite{mccloud1994understanding}.
This definition is intentionally broad enough to encompass the spectrum of the majority of works produced so far.
The dataset composition should reflects this heterogeneity to give everyone the opportunity
to compare its algorithms to a globally representative dataset of the comics world.
We contacted authors with different comic styles and have selected a corpus of one hundred images, representing single or double comics page.

The images were partly processed by the French company A3DNum (\url{http://www.a3dnum.fr}) which was commissioned to digitize 14 albums.
Among all the files, scanned at a resolution of 300 dots per inch and encoded in uncompressed Portable Network Graphic (PNG) format, we used 46 pages to integrate the eBDtheque corpus.
The remaining 54 images were selected from webcomics, public domain comics\footnote{http://digitalcomicmuseum.com} and unpublished artwork with different styles from 72 to 300 dots per inch.
We encoded all the images of the eBDtheque dataset in Joint Photographic Experts Group (JPEG) format with a lossy compression to facilitate file exchange.
%, the non compressed images are available on request.

Hereafter we describe the characteristics of the selected albums and their content.


\paragraph{Albums} % (fold)
 \label{par:albums}
 published between 1905 and 2012.
 29 pages were published before 1953 and 71 after 2000.
 Quality paper, colour saturation and textures related to printing technique changes can vary a lot from one page to another.
 The artworks are mainly from France (81\%), United States (13\%) and Japan (6\%).
 Their styles varies from classical Franco-Belgium ``bandes dessinées'' to Japanese manga through webcomic and American comics.

 % paragraph albums (end)

\paragraph{Pages} % (fold)
\label{par:pages}
 themselves have very diverse characteristics.
Among all, 72 are printed in colours and according to the authors and periods, there are a majority of the tint areas, watercolours and hand-coloured areas.
Among the remaining 28, 16 have are greyscale and 12 are simply black and white.
One album has two versions of each page, one in colour and the other one in black and white.
We have integrated an examples of each of them in order to allow performance comparison of algorithms on the same graphic style by using colour information or not.
Five of the 100 images are double page, others are single page and 20\% are not A4 format.
%We therefore, strictly speaking, 105 and 100 pages not in our database, each with a distinct structure
% paragraph pages (end)

\paragraph{Panels} % (fold)
\label{par:panels}
 contained in the pages are of various shapes.
Although most of them are bounded by a black line, a significant proportion has at least one part of the panel which is indistinguishable from the background of the page (frame less panel).
Two pages consist only of frame less panels, the visual delimitation uses background contrast difference between the panel and image.
Nine images contain overlapping panels, twelve contain only panels without border and several has panels connected by an straddle object.

% paragraph panels (end)

\paragraph{Balloons} % (fold)
\label{par:balloons}
 also contain a great diversity.
Some of them are completely surrounded by a black stroke, some partially and others not at all.
They have a bright background with a rectangular, oval or non geometric shape with ``smooth'', ``wavy'' or ``spiky'' contour in general.
Most of them has a tail pointing towards the speaker, but some do not.
There is text without any surrounding balloons on 33 images of the corpus.
% and 8 of them simply contain no bubble.

% paragraph balloons (end)

\paragraph{Text} % (fold)
\label{par:text}
 is either typewritten (61\% of the image) or handwritten, mainly upper-case.
The text lines contains 12 elements in average (see figure~\ref{fig:gt:textline_lenth_distribution}) and there are more than hundred text lines that are composed by only one letter which are punctuation or single letter words such as ``I'' or ``A''.

Most pages are from French artworks, where the text is written in French.
Only 13 pages contain English text and 6 images are in Japanese.
Onomatopoeia appears in 18 pages.

    %%%%%%%%%%%%%%%%%%%%%%%%%%%%%%%%%%%%%%%%%%%%%%%%%%%%%%%%
    \begin{figure}[ht]%trim=l b r t  width=0.5\textwidth,  
      \centering
      \includegraphics[trim= 100px 130px 120px 120px, clip, width=0.65\textwidth]{number_of_letter_per_textline.pdf}
      \caption{Distribution of the number of elements per text lines.
      }
      \label{fig:gt:textline_lenth_distribution}
    \end{figure}  
    %%%%%%%%%%%%%%%%%%%%%%%%%%%%%%%%%%%%%%%%%%%%%%%%%%%%%%%%

% paragraph text (end)

\paragraph{Comic characters} % (fold)
\label{par:comic_characters}
or protagonist are specific to each album.
They all have eye, harm and leg but at least 50\% are not humanoid, depending on the interpretation.
% paragraph comic_characters (end)

\subsection{Term of use} % (fold)
\label{sub:term_of_use}
We obtain the minimum rights for sharing and publishing image material from the right holders but we had to make sure the user accept it before using the data.
In collaboration with the intellectual property department of the University of La Rochelle, we established the following:

\textit{In order to use this database, you must firstly agree to the terms.
You may not use the database if you don't accept the terms.
The use of this database is limited to scientific and non-commercial purpose only, in the computer science domain.
For instance, you are allowed to split the images, through the use of segmentation algorithms.
You can also use pieces of this database to illustrate your research in publications and presentations.
Any other use case must be validated by our service.
If you do agree to be bound by all of these Terms of Use, please fill and email the request form and then use the login and password provided to download the selected version below.}

The concerned request form requires the identity, affiliation, address and intended use of the person who wishes to use data.

% subsection term_of_use (end)

\section{Ground truth construction} % (fold)
\label{sec:ground_truth_construction}
In order to cover a wide range of possible research matters, it has been decided to extract three different type of objects from the corpus: text lines, balloons and panels.
We decided to do this first ground truth by drawing horizontal bounding boxes as close as possible from the feature and including all its pixels.
% in order to proceed a maximum of pages in the allowed time. 
We chose this level of granularity in order to limit the subjectiveness of the person making the annotation.

The comics art is extremely heterogeneous and our dataset voluntarily integrates albums that can be classified as
unconventional.
This leaves room for interpretation on the form which increase annotation variations by different people and decrease the uniformity of the ground truth.
This precision level is used in several, widely used datasets~\cite{pascal-voc-2012, yao2007introduction}.
%{\color{blue}This groundtruthing level is also used for well known dataset widely used~\cite{pascal-voc-2012, yao2007introduction}.}

Hereafter we detail the two levels of annotation (visual and semantic) that form the ground truth and how they are indexed in a file.
The combination of visual and semantic annotation provides the advantage of making this ground truth relevant for document analysis and semantic evaluation which are both part of the comics understanding process.

% each element to annotate and the protocol to follow.
% follows specific guidelines according to the rules below:

\subsection{Visual annotation} % (fold)
\label{sub:visual_annotation}
The first annotation consist in defining the spacial region where elements are located in the image.
We describe hare how the visual annotation have been performed for the panels, balloons, texts and comic characters.

% subsection visual_annotation (end)
\paragraph{Panels} % (fold)
\label{par:panels}
The frame or panels are defined as an image area, generally rectangular, representing a single scene in the story. 
There is always at least one panel per page, the entire page region can be used as panel if necessary.
When a panel has a black border, the bounding box is placed as close as possible to its frame.
Sometimes, images have not been scanned perfectly horizontally, it is then impossible to have an horizontal bounding box sticking exactly to the border.
When the panel border is partially absent or suggested by the neighbourhood, the bounding box just defines the content of the panel.
In all cases, the other elements (balloon, text, drawings) extending from the frame are truncated, see figure~\ref{fig:gt:segPanel}.

%%%%%%%%%%%%%%%%%%%%%%%%%%%%%%%%%%%%%%%%%%%%%%%%%%%%%%%%%
\begin{figure}[h!]
\begin{center}
\includegraphics[width=0.8\textwidth]{segPanel.png}
\caption[Panel annotation]{Example of three panel annotations. The bounding box (transparent red) is defined without taking into account of non panel elements in all cases.}
\label{fig:gt:segPanel}
\end{center}
\end{figure}
%%%%%%%%%%%%%%%%%%%%%%%%%%%%%%%%%%%%%%%%%%%%%%%%%%%%%%%%%

%They can be frame-less, in that case the bounding box will be set according to the contained drawings, see Fig. \ref{fig:segmentation}c. %, ignoring text and overlapping features.
%In both cases, text and overlapping features are ignored.
%There is necessarily at least one panel in a page. 
% paragraph panels (end)

% The panel reading order is also annotated in a metadata called \texttt{rank}, it is in integer  between $1$ and $n$ which is increased incrementally from the first to the last panel $n$ in an image.%, le premier de la séquence prenant la valeur $1$ et le dernier la valeur $n$ pour une page contenant $n$ cases.

\paragraph{Balloons} % (fold)
\label{par:balloons}
We define a balloon (phylactery or bubble) as the region of an image including one or several lines of text, graphically defined by an identifiable physical boundary or suggested by the presence of an arrow pointing to the speaker (the tail).
Although rare, empty balloons (not containing lines of text) are also annotated if they are clearly identifiable by their shape or
the tail representation.
Pixel level annotation follows the contour of the balloon (see figure~\ref{fig:gt:segBalloons}c), while bounding box annotation does not consider the contour of phylactery and truncates the tail.
Sometimes it crosses the entire panel and generates an unrepresentative position of the desired balloon (see figure~\ref{fig:gt:segBalloons}a).
When the balloon is not closed (e.g. open contour) the annotated contour has to stick as close as possible to the contained text (see figure~\ref{fig:gt:segBalloons}b).
Note, the first version of the ground truth (2013) have been defined at bounding box level ignoring the tail and the second version (2014) at pixel level following the contour variations and the tail region.

%%%%%%%%%%%%%%%%%%%%%%%%%%%%%%%%%%%%%%%%%%%%%%ù
\begin{figure}[h!]
\begin{center}
\begin{tabular}{ccc}
a) \includegraphics[width=80px]{segBalloon1.png} 
& 
b) \includegraphics[width=80px]{segBalloon2.png}
&
c) \includegraphics[width=80px]{segBalloon3.png}
\end{tabular}
\caption[Speech balloon contour annotation]{Example of balloon clipping: a) using bounding box excluding the tail, b) bounding box of non closed balloon, c) pixel level contour annotation.} 
\label{fig:gt:segBalloons}
\end{center}
\end{figure}
%%%%%%%%%%%%%%%%%%%%%%%%%%%%%%%%%%%%%%%%%%%%%%ù

% paragraph balloons (end)

\paragraph{Text lines} % (fold)
\label{par:text_lines}
The text lines are defined as a sequence of text characters aligned in the same direction (see figure~\ref{fig:gt:segLines}a).
This definition encompasses both speech texts and narrative text, often located inside balloon, onomatopoeia (graphic sound) that are written or drawn directly in the panel without particular container.
%The text of the latter, although they are occasionally parallel to the edges of the box is still clipped by a horizontal bounding box for consistency across the ground truth.
Comics are static graphics, the expression of emotions of a comic character is the joint action of drawing and text, sometimes in the form of a single punctuation symbol.
For instance, an exclamation mark for surprise or a question mark for a misunderstanding.
These isolated symbols convey information and are segmented as text line as well (see figure~\ref{fig:gt:segLines}b).
Similarly, we have chosen to include in this category the illustrative text, such as a road sign or storefront (see figure~\ref{fig:gt:segLines}c).
Although at the boundary between text and graphic, these elements are still invariably read by the reader and their annotation is potentially interesting for multiple purposes, including story and scene analysis.

%%%%%%%%%%%%%%%%%%%%%%%%%%%%%%%%%%%%%%%%%%%%%%ù
\begin{figure}[h!]
\begin{center}
\begin{tabular}{ccc}
a) \includegraphics[width=80px]{segLine1.png} 
& 
b) \includegraphics[width=80px]{segLine2.png}
&
c) \includegraphics[width=80px]{segLine3.png}
\end{tabular}
\caption[Text line location annotation]{Different examples of text line position annotation: a) a classical text line in a speech balloon, b) a unique symbol, c) illustrative text, non horizontal.} 
\label{fig:segLines}
\end{center}
\end{figure}
%%%%%%%%%%%%%%%%%%%%%%%%%%%%%%%%%%%%%%%%%%%%%%ù

% paragraph text_lines (end)

\paragraph{Comic characters} % (fold)
\label{par:comic_characters}
The comic characters position have been included in the second version of the ground truth only (2014).
The concept of ``character'' may have different interpretations when used for comics and must be specified.
Characters in a comic have not necessarily a human-like, or even living beings appearance.
Even so, it would be appropriate to annotate every human being appearing in a box while some are nothing but a part of the scenery.
Therefore, we have chosen to limit the annotation to the comic characters that emits at least one speech balloon in the album (minimal impact in the story).
Their bounding box has been defined to maximize the region occupied by the comic character inside the box region.
Therefore, some parts of the character such as arms or legs, are clipped sometimes (see figure~\ref{fig:gt:segCharater}).

%%%%%%%%%%%%%%%%%%%%%%%%%%%%%%%%%%%%%%%%%%%%%%
\begin{figure}[h!]
\begin{center}
\includegraphics[width=0.75\textwidth]{segCharacter.png}
\caption[Comic character position annotation]{Example of comic character annotation: sniper's arms are not included in the bounding box in order to maximize the region occupied by the sniper in its bounding box. The two snipers and the two characters in the car are annotated because they emit a speech balloon in a different panel.}
\label{fig:gt:segCharacter}
\end{center}
\end{figure}
%%%%%%%%%%%%%%%%%%%%%%%%%%%%%%%%%%%%%%%%%%%%%%

% paragraph comic_characters (end)


% section ground_truth_construction (end)

\subsection{Semantic annotation} % (fold)
\label{sub:gt:semantic_annotation}
This second level of annotation complete each spacial region with additional information about its semantic.
Also, the image itself is annotated with extra information about its origin (e.g. album, collection, author, publisher).
%Once segmented, each object is annotated with a set of predefined metadata as follows. 

\paragraph{Images}
The image, often assimilated as pages, has been annotated with bibliographical information, so that anyone using this ground truth is free to get is own paper copy of the comic books for extra uses.
The first annotation is the page number (\texttt{{pageNumber}}) then the comic book title, from which the page has been picked up, and its release date (\texttt{{albumTitle, releaseDate}}), the series it belongs to (\texttt{{collectionTitle}}), the authors and editor names (\texttt{{writerName, drawerName, editorName}}) and, finally, the website and/or ISBN (\texttt{{website, ISBN}}).
The album title is not mandatory for webcomics.
Structural information about the page content has been added as well, such as resolution (\texttt{{resolution}}), reading direction (\texttt{{readingDirection}}), main language of the text (\texttt{{language}}) and single or double page information (\texttt{{doublePage}}).
% These annotation are stored in a tag named \texttt{{Pages}}.

\paragraph{Panels} 
The panes are annotated with a \texttt{{rank}} metadata which stand for its position in the reading sequence. 
The first panel to be read on a given page has its rank property set to 1, while the last one is set to \textit{n}, where \textit{n} is the number of panels in the page.

\paragraph{Balloons}
Balloons are also annotated with a \texttt{{rank}} property that defines their reading order relatively to the image because balloons are not always included in panels.
%, their rank is set according to the page as a whole. 
For a page containing $m$ balloons, the first balloon's rank will be 1 and the last will be $m$.
%Moreover, two additional metadata are given. 
A second information concerns the \texttt{{shape}} of the balloon.
This feature conveys an information about how the contained text is spoken (tone).
The type of \texttt{{shape}} is given from the following list \{\texttt{smooth, wavy, spiky, suggested}\} as pictured in figure~\ref{fig:gt:balloonShape}.
Finally, the tail tip position (extremity) and its pointing direction have been added into the second version of the ground truth.
There are given through the \texttt{{tailTip}} and \texttt{{tailDirection}} properties.
The possible values of the direction are reduced to the eight cardinal directions plus a ninth additional value for the lack of tail: \{\texttt{N, NE, E, SE, S, SW, W, NW, none}\}.
In the second version of the ground truth (2014), we added the identifier of the comic character which is emitting the balloon \texttt{{idCharacter}}. 

%%%%%%%%%%%%%%%%%%%%%%%%%%%%%%%%%%%%%%%%%%%%%%%%%%%
\begin{figure}[h!]
\begin{center}
\includegraphics[width=0.65\textwidth]{balloonShape.png}
\caption[Balloon shapes]{Les différents types de contour de bulle, de gauche à droite et de haut en bas : nuage, hérissé, suggéré, rectangle, ovale.}
\label{fig:gt:balloonShape}
\end{center}
\end{figure}
%%%%%%%%%%%%%%%%%%%%%%%%%%%%%%%%%%%%%%%%%%%%%%%%%%%

\paragraph{Text lines}
The text lines are associated with their transcription and the identifier of the corresponding balloon is added in \texttt{{idBalloon}} if the text line is included in a balloon.

\paragraph{Comic characters} % (fold)
\label{par:comic_characters}
The comic character are identified by \texttt{{idCharacter}} in order to be easily referred.

% paragraph comic_characters (end)

% \paragraph{Speaking character and speech balloon} % (fold)
% \label{par:speaking_character_and_speech_balloon}
% In the second version of the ground truth, we added a new type of relation which is between regions.
%  balloons and comic characters regions that are related in the story (balloon spoken by a character).
% The information is stored in a structure called \texttt{{LinkSBSC}} that consist a pair of identifier \texttt{{idBalloon}} and \texttt{{idCharacter}}.
% paragraph speaking_character_and_speech_balloon (end)


%%%%%%%%%%%%%%%%%%%%%%%%%%%%%%%%%%%%%%%%%%%
% \begin{figure}
% \begin{center}
% \includegraphics[width=0.45\textwidth]{balloons_shape3.png}
% \caption{Different speech balloon shapes. Top-down, from left to right: cloud, peak, suggested, rectangular and oval.}
% \label{fig:balloons_shape}
% \end{center}
% \end{figure}
%%%%%%%%%%%%%%%%%%%%%%%%%%%%%%%%%%%%%%%%%%%

% subsection semantic_annotation (end)

\subsection{File structure} % (fold)
\label{sub:file_structure}

The ground truth file structure have been thought according to comics related formalism such as Comics Markup Language (ComicsML)~\cite{McIntosh2011}, Comic Book Markup Language (CBML)~\cite{Walsh2012a}, Periodical Comics\footnote{\url{http://www.w3.org/wiki/WebSchemas/PeriodicalsComics}}, A Comics Ontology~\cite{Rissen2012}, Advanced Comic Book Format (ACBF)\footnote{\url{https://launchpad.net/acbf}} and the Grand Comics Database (GCD)\footnote{\url{http://www.comics.org}}.
See the Ph.D. thesis of Gu{\'e}rin~\cite{phdthesisGuerin14} for an extended review.

As we wanted to keep the ground truth file system simple and easy to share, visual and semantic annotations about a given page are gathered in a single full-text file following the specifications of Scalable Vector Graphics (SVG). 
Besides being an open-standard developed by the World Wide Web Consortium (W3C) since 1999, the SVG format fulfils two essential needs for this database.

First, using a recent Internet browser or your favourite image viewer, it provides a simple, fast and elegant way to display the visual annotation of any desired object over a comic book page using layers.
No need to install software such as Matlab, Adobe Illustrator or equivalent open source to visualize the ground truth information.

It is XML-based vector image format that allows to display an animate the annotated region, stored as polygon object in the SVG file, as desired using the Cascading Style Sheets (CSS) properties, see figure~\ref{fig:gt:svgImage}. 

%%%%%%%%%%%%%%%%%%%%%%%%%%%%%%%%%%%%%%%%%%%%%%%%%

\begin{figure}[h!]
\begin{center}
\begin{tabular}{ccc}
a) \includegraphics[width=100px]{svgPanel.png} 
& 
b) \includegraphics[width=100px]{svgBalloon.png}
&
c) \includegraphics[width=100px]{svgTextlines.png}
\end{tabular}
\caption[Annotation rendering in a browser]{Example of rendering for each class of element. For example, red for panels (a), cyan for balloons (b) and green for text (c). The opacity is set to 50\% to allow seeing the corresponding image by transparency.} 
\label{fig:gt:svgImage}
\end{center}
\end{figure}
%%%%%%%%%%%%%%%%%%%%%%%%%%%%%%%%%%%%%%%%%%%%%%%%%%

Each layer can be displayed or not in order to enhance the clearness of the annotations when browsing the database.
Secondly, SVG being a XLM-based language, it makes the integration of semantic annotation very easy via the use of the predefined \texttt{metadata} element.

One ground truth file contains the complete description of one comics image. 
There is no hierarchical link between pages from a same comic book. 
Following the basic XML encoding information, a SVG file starts with a root \texttt{<svg>} element containing the title of the document, \texttt{<title>}, and five \texttt{<svg>} children with different class attributes.
These contain annotations collected on five types of elements which are the page, panels, balloons, text lines and comic characters.
The type of element in a tag is specified by its \texttt{class} attribute.
The first tag, \texttt{class = ``Page''} contains description on the image and has two daughters.
The first one, \texttt{image} has several attributes which specifies a link to the image file in the dataset \texttt{xlink: href} and two specifying the \texttt{width} and \texttt{height} of the image.
The second, \texttt{metadata}, contains bibliographic information about the album and page properties described in \ref{sub:gt:semantic_annotation}.
%A SVG file begins with information about the XML version and encoding system and then a root \texttt{<svg>} element containing the title of the document \texttt{<title>} and four others \texttt{<svg>} elements with different class attributes. % that describe the page, the panels, the text lines and the text areas the entire set of annotations for the attached page.
%The annotations are the same for all the \texttt{<svg>} nodes, each of them describing one kind of element (e.g. panel, balloon) according to its {\tt class} attribute. 
%The first \texttt{<svg>} element, \texttt{<svg class=``Page''>}, has two children.
%The first one is \texttt{<image>} and contains a link to the corresponding image file and the size it has to be displayed.
%The next child is a \texttt{<metadata>} element containing the bibliographical information described in \ref{sub:gt:semantic_annotation}.
The four following \texttt{<svg>} siblings, \texttt{<svg class=``Panel''>}, \texttt{<svg class=``Balloon''>}, \texttt{<svg class=``Line''>} and \texttt{<svg class=``Character''>} respectively contain the annotations about panels, balloons, text lines and comic characters. 
They all contain SVG \texttt{<polygon>} elements with a list of five or more points in a \texttt{point} attribute that define the position of the bounding box corners or the pixel-level contour. %four corners position coordinates of a bounding box. 
Note that the last point is always equal the first one to ``close'' the polygon according to the SVG specifications. 
Those points are used by the viewer to draw polygons with the corresponding CSS style. 
Each \texttt{<polygon>} has a \texttt{<metadata>} child to store information on the corresponding polygon, according to the attributes list described in \ref{sub:gt:semantic_annotation}.

An example of ground truth file is given figure~\ref{list:svg}.% présente un exemple du contenu de l'un de ces fichiers.

\begin{lstlisting}[language=XML, frame=single, caption=Example of SVG file of the eBDtheque ground truth, captionpos=b, label=list:svg]
<?xml version="1.0" encoding="UTF-8" standalone="no"?>
<svg>
  <title>CYB_BUBBLEGOM_T01_005</title>
  <svg class="Page">
    <image 
    	x="0" 
    	y="0" 
    	width="750" 
    	height="1060" 
    	href="CYB_BUBBLEGOM_T01_005.jpg"
    />
    <metadata 
    	collectionTitle="Bubblegom_Gom"
    	editorName="Studio_Cyborga" 
    	doublePage="false"
    	website="http://bubblegom.over-blog.com"
    	albumTitle="La_Legende_des_Yaouanks"
    	drawerName="Cyborg_07"
    	language="french"
    	resolution="300"
    	ISBN="979-10-90655-01-0"
    	readingDirection="leftToRight" 
    	writerName="Cyborg_07"
    	releaseDate="2009"
    	pageNumber="5"
    />
  </svg>
  <svg class="Panel">
    <polygon points="53,95 268,95 268,292 53,292 53,95">
      <metadata rank="1"/>
    </polygon>
    ...
  </svg>
  <svg class="Balloon">
    <polygon points="61,103 143,103 143,172 61,172 61,103">
      <metadata
      	idBalloon="B01" 
      	idCharacter="C01" 
      	shape="smooth"
      	tailDirection="SE" 
      	rank="1"
      />
    </polygon>
    ...
  </svg>
  <svg class="Line">
    <polygon points="373,121 432,121 432,132 373,132 373,121">
      <metadata idBalloon="B01">
      	LIKE YOU.
      </metadata>
    </polygon>
    ...
  </svg>
  <svg class="Character">
    <polygon points="84,153 261,153 261,298 84,298 84,153"/>
    	<metadata idCharacter="C01"/>
    ...
  </svg>
</svg>
\end{lstlisting}

%{\bf Add SVG code figure here?} Arnaud suggested it too but, we don't have room and it will be ugly...

%
%\begin{lstlisting}
%<svg>
%</svg>
%\end{lstlisting}

%{\color{red}short file as an illustration?}

% \begin{figure}
% \begin{center}
% \begin{tabular}{ccc}
% \includegraphics[width=0.14\textwidth]{fig/bbg_panel2.png} & 
% \includegraphics[width=0.14\textwidth]{fig/bbg_balloon2.png} &
% \includegraphics[width=0.14\textwidth]{fig/bbg_textlines2.png}
% %\includegraphics[width=0.20\textwidth]{fig/bbg_all.png} &
% \end{tabular}
% \caption{Left-to-right: segmentation of a panel, a speech balloon and text lines. Credits: \cite{Bubble09}}
% \label{fig:css}
% \end{center}
% \end{figure}
% subsection file_structure (end)

\begin{itemize}
	\item Comic Book Markup Language \url{http://digitalhumanities.org/dhq/vol/6/1/000117/000117.html}
\end{itemize}

\section{Ground truth quality assessment}
\label{sec:gt:ground_truth_quality_assessment}

When several persons are involved in the creation of a graphical ground truth, it is very difficult to obtain a perfectly homogeneous segmentation.
Indeed, it could vary from one person to another because each person has a different sensitivity at reading comics and at integrating instructions.
Therefore, in addition to the package of pages he was in charge of, each participant has been asked to annotate the panels of a extra page. 
This extra page was the same for everybody and was chosen for its graphical components heterogeneity. 
It contained ten panels from which, four were full-framed, five half-framed and one was frame less.
%, see Fig. \ref{fig:test_page}. 
% This heterogeneity is somehow representative of the whole corpus.
We defined an acceptable error for the position of a corner given by several persons.
The images of dataset being of different definitions, using a percentage of the page size makes more sense than using a specific number of pixels.
%A percentage of the page width and height is more relevant.
%Let set an acceptable error of 0.5\% of the page for the position on each corner. 
We set this percentage $p$ at 0.5\% of the page height and width in $x$ and $y$. 
Given the definition of the test image of 750x1060 pixels, this makes a delta of +/- 5 pixels in $y$ axis and +/- 4 pixels in $x$ axis. 

We asked to each one of the twenty involved persons to draw the four points bounding box of the panels ignoring text area.
A mean position from the twenty different values has been calculated for each of them.
Then, the distance of each point to its mean value is computed. 
Figure~\ref{fig:gt:graphiqueStdVT} shows the amount of corners for a distance, centred on zero.


%%%%%%%%%%%%%%%%%%%%%%%%%%%%%%%%%%%%%%%%%%%%%ù
\begin{figure}[h!]
\begin{center}
\includegraphics[width=0.7\textwidth]{stdVT.png}
\caption[Distance to the mean position]{Number of corners for a given standard deviation value. This has been calculated on $y$ axis and $x$ axis and produces similar plot.}
\label{fig:gt:graphiqueStdVT}
\end{center}
\end{figure}
%%%%%%%%%%%%%%%%%%%%%%%%%%%%%%%%%%%%%%%%%%%%%ù

Given the threshold $p=0.5$, 87.5\% of pointed corners can be considered as being homogeneous over the group of labelling people. 
%{\bf what is p?} p is the threshold
The overall mean standard deviation on this page reaches 0,15\%
 (1.13 pixels) for the width, and 0.12\% (1.28 pixels) for the height.
The two bumps, at -40 and 15, are related to the missegmentation of 13 of the 80 panels. 
Indeed, instructions have been misunderstood by some people who included text area outside of the panels or missed some panel's parts.
Figure~\ref{fig:gt:diffVT} shows the difference between areas labelled as a panel by at least one person and areas labelled as a panel by every participant.
However, such mistakes have been manually corrected before publishing the ground truth.

%%%%%%%%%%%%%%%%%%%%%%%%%%%%%%%%%%%%%%%%%%%%%%%%%
\begin{figure}[h!]
\begin{center}
\includegraphics[width=0.7\textwidth]{segDifference.png}
\caption[Error measurement image]{Error measurement page. Red hatched areas are the difference between areas labelled as panels by at least one person, and areas labelled by everybody. Image credit:~\cite{Bubble09}.}
\label{fig:gt:diffVT}
\end{center}
\end{figure}
%%%%%%%%%%%%%%%%%%%%%%%%%%%%%%%%%%%%%%%%%%%%%%%%%

Even though the error criterion has only been estimated on panels, it is reasonable to extend it to balloons and text lines as well.
Indeed, the segmentation protocol being quite similar for all features (bounding box as close as possible to the drawing), the observed standard deviation of panel corner positions has no reason to be different from balloons and text lines.
The pixel-level balloon and the comic characters visual annotation have been carried out by a single person, the homogeneity is only subject to the regularity of the person over time and is, therefore, difficult to assess quantitatively.

% \section{Discussions}
% \label{sec:gt:siscussions}


\section{Conclusions}
\label{sec:gt:concl}
We presented the eBDtheque, a representative database of comics, which is composed by comics images, visual (spatial) and semantic annotations.
The one hundred image corpus has been introduced as well as its ground truth construction and quality assessment protocols.

The material have been published in 2013~\cite{Guerin2013} and made available to the scientific community via the project website~\footnote{http://ebdtheque.univ-lr.fr}.

\modif{At the time this thesis is written, five requests from India, China and Japan have been received to use this dataset and ground truth.}

% In order to provide a common basis for evaluating research work, the ground truth have been published
% It has been enhanced in 2014 by adding semantic information to the already annotated elements.

