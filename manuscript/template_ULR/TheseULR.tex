%%%%%%%%%%%%%%%%%%%%%%%%%%%%%%%%%%%%%%%%%%%%%%%%%%%%%%%%%%%%%%%%%%%%%
%THESES DE LA ROCHELLE : STYLES POUR LES PAGES DE COUV et DE DOS obligatoires
%Construit a partir des styles d'Orleans, helene.jouguet@univ-orleans.fr / SCD

%Classe du document : book :
\documentclass[a4paper,14pt]{book}

%%%%%%%%%%%%%%%%%%%%%%%%%%%%%%%%%%%%%%%%%%%%%%%%%%%%%%%%%%%%%%%%%%%%%
%Packages :
\usepackage[T1]{fontenc}			%codage de caractères T1
\usepackage[latin1]{inputenc}	%caractères accentués
% \usepackage[utf8]{inputenc}
% \usepackage[english]{babel}
\usepackage[francais]{babel}	%adaptation de LaTeX au français
\usepackage{xspace}						%Gestion des espaces
\usepackage[pdftex]{graphicx}	%inclure des graphiques
\usepackage{vmargin}					%Definition des marges
%\usepackage[scaled]{uarial}   %arial pour les pages de couv et 4ème
\usepackage{times} 

% Autres packages non utilisés pour les couv.
%\usepackage{amsmath}					%complement mathematique
%\usepackage{amssymb}					%ajoute des symboles
%\usepackage{array}						%Outils supplémentaires pour les tableaux
%\usepackage{multicol}					%Plusieurs colonnes possibles
%\usepackage{indentfirst} 			%indenter le premier paragraphe après une nouvelle section
%\usepackage{latexsym}					%ajoute des symboles
%\usepackage{setspace}					%Pour les Interlignes
%\usepackage{algorithm}				%Pour les algorithmes
%\usepackage{algorithmic}			%Pour les algorithmes
%\usepackage{makeidx}					%Pour l'index

%%%%%%%%%%%%%%%%%%%%%%%%%%%%%%%%%%%%%%%%%%%%%%%%%%%%%%%%%%%%%%%%%%%%%
%en tete et pied de page :
\usepackage{fancyhdr}
\fancyhead{}

\fancyhead[L]{\rightmark}			%calle à gauche le titre de la section courante en tete de page
\fancyfoot[C]{\thepage}				%centre LE numero de page en pied de page

%%%%%%%%%%%%%%%%%%%%%%%%%%%%%%%%%%%%%%%%%%%%%%%%%%%%%%%%%%%%%%%%%%%%%

%Definition cmd :
%center verticalement dans une page
\newenvironment{vcenterpage}
{\newpage\vspace*{\fill}}
{\vspace*{\fill}\par\pagebreak}

%%%%%%%%%%%%%%%%%%%%%%%%%%%%%%%%%%%%%%%%%%%%%%%%%%%%%%%%%%%%%%%%%%%%%
%%%%%%%%%%%%%%%%%%%%%%%%%%%%%%%%%%%%%%%%%%%%%%%%%%%%%%%%%%%%%%%%%%%%%
%DEBUT DU DOCUMENT
\begin{document}
\shorthandoff{:}
%%%%%%%%%%%%%%%%%%%%%%%%%%%%%%%%%%%%%%%%%%%%%%%%%%%%%%%%%%%%%%%%%%%%%
%Propriete du document :
\author{
\textbf{Christophe RIGAUD}\\
~\\ ~\\
Laboratoire Informatique, Image et Interaction \textsc{France}\\
~\\
\textit{christophe.rigaud@univ-lr.fr}\\
}

\title{
\textbf{\huge Segmentation et indexation d'objets complexes dans les images de bandes d{\'e}ssin{\'e}es}
\\
Version 
}

%%%%%%%%%%%%%%%%%%%%%%%%%%%%%%%%%%%%%%%%%%%%%%%%%%%%%%%%%%%%%%%%%%%%%
%Page de garde :
\thispagestyle{empty}

\setmarginsrb{20mm}{0mm}{15mm}{15mm}{0mm}{0mm}{0mm}{0mm}

%\begin{sffamily}

\begin{tabular}{ p{3cm} p{12cm}}
		\begin{minipage}{3cm}
			\includegraphics[width=3cm]{./images/UnivLaRochelle.png} 
		\end{minipage}
	&
		\begin{minipage}{12cm}
			\begin{center}
				\textbf{\LARGE\textbf{UNIVERSIT\'{E} DE LA ROCHELLE}}
			\end{center}
		\end{minipage}
\end{tabular}
	
\vspace{1.3cm}
	
\begin{minipage}{16cm}
	\begin{center}
	\LARGE \textit{\textbf{
		\'ECOLE DOCTORALE S2IM \\ \vspace{0.2cm}}}
		$\ $ Laboratoire Informatique, Image et Interaction (L3i)
	\end{center}
\end{minipage}
	
\vspace{1.3cm}
	
\begin{minipage}{16cm}
	\begin{center}
		\LARGE \textbf{TH\`ESE}
		\normalsize pr\'esent\'ee par : \\ \vspace{0.5cm}
		\LARGE \textbf{Christophe RIGAUD}\\ \vspace{0.5cm}
		\large soutenue le : \textbf{[XX mois en lettres 2014]}
	\end{center}
\end{minipage}
	
\vspace{0.2cm}
	
\begin{minipage}{16cm}
	\begin{center}
		\large pour obtenir le grade de : \textbf{Docteur de l'universit{\'e} de La Rochelle \\ \vspace{0.2cm}}
		Discipline : \textbf{informatique et applications}
	\end{center}
\end{minipage}
	
\vspace{1cm}

\fbox{
	\begin{minipage}{16cm}
		\vspace{0.2cm}
		\begin{center}
			\LARGE \textbf{Segmentation et indexation d'objets complexes dans les images de bandes d{\'e}ssin{\'e}es}\\ \vspace{0.3cm}
			\LARGE{[Sous titre {\'e}ventuel]}
		\end{center}
		\vspace{0.2cm}
	\end{minipage}
}

\vspace{2cm}

\hbox{\raisebox{0.4em}{\vrule depth 0pt height 0.4pt width 16cm}}

\vspace{0.5cm}

\textsc{\textbf{JURY : }}  $\ $ \vspace{0.2cm} \\
\begin{tabular}{l p{2cm} p{9cm}}
  \textbf{Pr{\'e}nom \textsc{NOM}} & $\ $ &  Professeur, Université xxxxxx, Président du jury\\
  \textbf{Pr{\'e}nom \textsc{NOM}} & $\ $ &  Directeur de recherche CNRS, Université xxxx, Rapporteur\\
  \textbf{Pr{\'e}nom \textsc{NOM}} & $\ $ &  Professeur, Université xxxxxxx, Rapporteur\\
  \textbf{Pr{\'e}nom \textsc{NOM}} & $\ $ &  Professeur, Université de La Rochelle, Directeur de thèse\\
  \textbf{Pr{\'e}nom \textsc{NOM}} & $\ $ &  Professeur, Université xxxxxxxxxxx\\
  \textbf{Pr{\'e}nom \textsc{NOM}} & $\ $ &  Maître de conférences, Université xxxxxxxxxxxx\\
  \textbf{Pr{\'e}nom \textsc{NOM}} & $\ $ &  Titre, établissement\\
\end{tabular}

%\end{sffamily}

\setlength{\voffset}{0pt}

\thispagestyle{empty}

%%%%%%%%%%%%%%%%%%%%%%%%%%%%%%%%%%%%%%%%%%%%%%%%%%%%%%%%%%%%%%%%%%%%%
\pagestyle{fancy}

%%%%%%%%%%%%%%%%%%%%%%%%%%%%%%%%%%%%%%%%%%%%%%%%%%%%%%%%%%%%%%%%%%%%%

%LA THESE



%%%%%%%%%%%%%%%%%%%%%%%%%%%%%%%%%%%%%%%%%%%%%%%%%%%%%%%%%%%%%%%%%%%%%
%La page de dos :
\newpage
\thispagestyle{empty}
\newpage $\ $
\newpage
\thispagestyle{empty}
\thispagestyle{empty}

\setmarginsrb{20mm}{15mm}{15mm}{15mm}{0mm}{0mm}{0mm}{0mm}

%\begin{sffamily}

\begin{center}
%		\LARGE{[Pr�nom NOM]}\\ 
		\textbf{[Titre de la th�se (en fran�ais)]}
\end{center}
	
\vspace{0.8cm}

\fbox{
	\begin{minipage}{16cm}
	R�sum� : (1700 caract�res max.)\\
	\\Lorem ipsum dolor sit amet, consectetur adipiscing elit. Proin volutpat ipsum id purus ultrices lobortis. Maecenas ornare enim quis eros. Nunc eget mauris ut quam malesuada mattis. Vestibulum ante ipsum primis in faucibus orci luctus et ultrices posuere cubilia Curae; Integer vel tellus. Nam rutrum, purus non sodales rhoncus, quam magna imperdiet eros, sit amet euismod justo metus at orci. Suspendisse neque turpis, feugiat interdum, faucibus vel, aliquet quis, risus. Etiam est elit, eleifend a, consequat sit amet, scelerisque nec, odio. Quisque id odio quis libero iaculis tincidunt. Sed non mi. Morbi aliquam commodo nibh. Integer justo purus, pulvinar a, suscipit vel, iaculis a, justo. Morbi ut orci. Maecenas fringilla orci. Phasellus auctor, enim vitae tempus egestas, justo mi cursus sem, vel blandit leo turpis vitae quam. Etiam sit amet felis vitae eros ornare porttitor.\\
Proin orci ligula, vehicula non, ultrices at, ultrices ut, massa. Vestibulum ac est. Curabitur at erat. Mauris gravida. Praesent vestibulum. Curabitur eget orci ac massa cursus condimentum. Integer sapien dui, ultricies eget, dapibus a, dapibus id, mauris. Curabitur felis velit, aliquam at, aliquet in, iaculis vitae, velit. Nunc lobortis magna id ligula. Vestibulum ante ipsum primis in faucibus orci luctus et ultrices posuere cubilia Curae; Integer congue ultrices mi.
Isdem diebus Apollinaris Domitiani gener, paulo ante agens palatii Caesaris curam, ad Mesopotamiam missus a socero per militares numeros immodice scrutabatur, an quaedam altiora meditantis iam Galli secreta susceperint scripta, qui conpertis Antiochiae gestis per minorem Armeniam lapsus Constantinopolim petit.\\
	\\
Mots cl�s : mot 1, mot 2, ...
	\end{minipage}
}

\vspace{0.8cm}


	\begin{center}
		\large \textbf{[Titre de la th�se (en anglais)]}
	\end{center}

\vspace{0.8cm}

\fbox{
	\begin{minipage}{16cm}
	Summary : (1700 caract�res max.)\\
	\\Lorem ipsum dolor sit amet, consectetur adipiscing elit. Proin volutpat ipsum id purus ultrices lobortis. Maecenas ornare enim quis eros. Nunc eget mauris ut quam malesuada mattis. Vestibulum ante ipsum primis in faucibus orci luctus et ultrices posuere cubilia Curae; Integer vel tellus. Nam rutrum, purus non sodales rhoncus, quam magna imperdiet eros, sit amet euismod justo metus at orci. Suspendisse neque turpis, feugiat interdum, faucibus vel, aliquet quis, risus. Etiam est elit, eleifend a, consequat sit amet, scelerisque nec, odio. Quisque id odio quis libero iaculis tincidunt. Sed non mi. Morbi aliquam commodo nibh. Integer justo purus, pulvinar a, suscipit vel, iaculis a, justo. Morbi ut orci. Maecenas fringilla orci. Phasellus auctor, enim vitae tempus egestas, justo mi cursus sem, vel blandit leo turpis vitae quam. Etiam sit amet felis vitae eros ornare porttitor.\\
	Proin orci ligula, vehicula non, ultrices at, ultrices ut, massa. Vestibulum ac est. Curabitur at erat. Mauris gravida. Praesent vestibulum. Curabitur eget orci ac massa cursus condimentum. Integer sapien dui, ultricies eget, dapibus a, dapibus id, mauris. Curabitur felis velit, aliquam at, aliquet in, iaculis vitae, velit. Nunc lobortis magna id ligula. Vestibulum ante ipsum primis in faucibus orci luctus et ultrices posuere cubilia Curae; Integer congue ultrices mi.
Isdem diebus Apollinaris Domitiani gener, paulo ante agens palatii Caesaris curam, ad Mesopotamiam missus a socero per militares numeros immodice scrutabatur, an quaedam altiora meditantis iam Galli secreta susceperint scripta, qui conpertis Antiochiae gestis per minorem Armeniam lapsus Constantinopolim petit.\\
	\\
Keywords : word 1, word 2,...
	\end{minipage}
}

\vspace{0.8cm}

\begin{tabular}{ p{3cm} p{9cm} p{3cm}}
		\begin{minipage}{3cm}
			[logo partenaire]
		\end{minipage}
	&
		\begin{minipage}{9cm}
			\begin{center}
				SIGLE DU LABORATOIRE\\
				\vspace{0.3cm}
				\textbf{[Nom et adresse du laboratoire]}\\
				\vspace{0.3cm}
				17xxx LA ROCHELLE
			\end{center}
		\end{minipage}
	&
		\begin{minipage}{3cm}
			[logo partenaire]
		\end{minipage}
\end{tabular}

%\end{sffamily}

\end{document}%%%%%%%%%%%%%%%%%%%%%%%%%%%%%%%%%%%%%%%%%%%%%%%%%%%%%%%%%%%%%%